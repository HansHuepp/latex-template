\chapter{Gesetzliche Anforderungen an Messanger in medizinischen Einrichtungen}\label{chapter:ganforderungen}
Trotz eines großen Angebotes an Messenger Diensten für den privaten, wie auch den professionellen gebrauch, gibt es zum aktuellen Stand keine anerkannte Lösung für einen Dienst welcher im Krankenhausbereich eingesetzt werden kann, da sie den hohen Anforderungen der DSGVO an die Verarbeitung von Gesundheitsdaten nicht gerecht werden. Dies zeigt wie wichtig eine genaue Analyse der Gesetzeslage ist und welche Implikationen der Datenschutz auf die Umsetzung des Dienstes hat.

\section{DSGVO Anforderungen}\label{section:dsgvo}
Das folgende Kapitel behandelt die für die Entwicklung des Krankenhaus Messenger Dienstes relevanten DSGVO Regelungen. Die relevaten Gesetzestexte werden dabei im folgenden zu Themenbereichen zusammengeordnet und ihr Effekt auf die finale Messenger Plattform erläutert. Dabei wird sich an der vom Bundesministerium für Wirtschaft und Energie herausgegebenen "Orientierungshilfe zum Gesundheitsdatenschutz" orientiert.

Dass die Gesundheitsdaten zum Zweck der Gesundheits vorsorge verarbeitet werden sollen, reicht dagegen in der Regel nicht aus, um eine Zulässigkeit zu begründen. Denn die Gesundheitsvorsorge ist nur dann als Ausnahme vom Verbot der Verarbeitung von Gesundheitsdaten einschlägig, wenn z. B. ein (Berufs-)Geheimnisträger die Verarbeitung verantwortet (Artikel 9 Abs. 3 DSGVO).

\subsection{Maßnahmen zur Wahrung der Nutzerrechte}\label{subsection:mzwdn}

\subsection{Vorgaben zur Datensicherheit}\label{subsection:mzwdn}

\subsection{Gesundheitsdatenspezifische Anforderungen}\label{subsection:ga}

\section{Sonderfälle z.B. Röntgenaufnahmen}\label{section:sr}

\section{Haftung}\label{section:haftung}

\section{Zusammenfassung gesetzlicher Anforderungen}\label{section:zga}

