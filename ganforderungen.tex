\chapter{Anforderungsanalyse an einen Messanger für den Einsatz in medizinischen Einrichtungen}\label{chapter:ganforderungen}
Trotz eines großen Angebotes an Messenger Diensten für den privaten, wie auch den professionellen gebrauch in Unternehmen, gibt es zum aktuellen Stand keine anerkannte Lösung für einen Dienst welcher im Krankenhausbereich eingesetzt werden kann, da sie den hohen Anforderungen der DSGVO an die Verarbeitung von Gesundheitsdaten nicht gerecht werden. Dies zeigt wie wichtig eine genaue Analyse der Gesetzeslage ist und welche Implikationen der Datenschutz auf die Umsetzung des Dienstes hat.

\section{Thematische Grundlagen}\label{chapter:hintergrund}
Um einen Einstiegspunkt in die Thematik der technischen Herausforderungen, des Datenschutzgesetz der Nutzeranforderungen an Messenger Dienste im Krankenhausbereich zu ermöglichen, müssen zunächst die einzelnen Anforderungssteller erörtert werden. Ebenso müssen auch die Problemstellungen rund um herkömmlichen Messenger Systeme und der gegenwärtige Zustand innerhalb von Krankenhäusern in Bezug auf digitale Kommunikation erörtert werden.
//Verknüpfung mit Forschungsmethodik

\subsection{Übersicht der Anforderungssteller}\label{section:ueda}
Im folgenden Kapitel werden die Anforderungssteller, welche bei der Entwicklung eines Messenger-Dienstes im Krankenhausbereich zu beachten sind gelistet und die Relevants der Vorderungen auf Basis der für den Umfang dieser Arbeit vorgenommen Einschränkungen analysiert. Diese werden in späteren Kapiteln referenziert und ihr Einfluss auf die Konzeption des Dienstes weiter analysiert und ausgewertet. Der vom Bundesverband Gesundheits-IT e. V. herausgegebene Report "Austausch von Gesundheitsdaten - Datenschutzrechtliche Anforderungen an Datenaustauschplattformen im Gesundheitswesen" nennt für eine dem Messenger Dienst entsprechende Datenaustauschplattform folgende Akteure: Hersteller der Datenaustauschplattform, Betreiber der Datenaustauschplattform, Datenlieferant, Nutzer der Datenaustauschplattform und Betroffene bei Nutzung der Datenaustauschplattform.\footnote{Vgl. \cite[S. 13]{Schubert2014}} Diese werden im Folgenden vorgestellt.
//Verknüpfung mit Forschungsmethodik

\subsubsection{Betroffene bei Nutzung des Messangers}\label{subsection:bbndd}
Bei den Betroffenen handelt es sich um das Personal, welches den Dienst nutzt und die Patienten, deren Daten über diese Plattform ausgetauscht werden, denn bei beiden Personengruppen findet eine Datenverarbeitung durch den Messenger Dienst statt. Die Anforderungen dieser beiden Personengruppen werden dabei über das Datenschutzrecht festgelegt, welches im folgenden kurz vorgestellt wird:

Datenschutz-Grundverordnung (DSGVO): Die DSGVO ist eine Verordnung der Europäischen Union, welche die private wie auch die öffentliche Verarbeitung personenbezogener Daten EU-weit vereinheitlichend reglementiert. Dadurch soll einerseits der Schutz personenbezogener Daten innerhalb der Europäischen Union sichergestellt und andererseits der freie Datenverkehr innerhalb des europäischen Binnenmarktes gewährleistet werden. Die Einhaltung der DSGVO Verordnungen ist eine der obersten Prioritäten bei der Entwicklung des Messenger Dienstes. Ein Verstoß wird mit erheblichen Bußgeldern geahndet. Da sich die DSGVO auf die Verarbeitung von personenbezogenen Daten in ihrer Gesamtheit bezieht, sind nicht alle Regelungen gleichermaßen relevant für die Entwicklung des Messangers. Besonders zu beachtende Regelungen sind "das Recht auf Löschung", "Das Recht auf Einverständnis" und "das Recht auf Datenübertragung". Die Einhaltung aller für den Dienst relevanten Anforderungen ist eine der zentralen Zielsetzungen dieser Arbeit, weshalb eine Auswertung der zu beachtenden Gesetze durchgeführt werden muss. Diese ist zu finden in Kapitel x

Länderrecht: Neben der europaweit geltenden DSGVO gibt es innerhalb der einzelnen Bundesländer zusätzliche Datenschutzbestimmungen, welche im jeweiligen Bundesrecht verankert sind und in einzelnen Fällen Einfluss auf die Verarbeitung von Gesundheitsdaten haben.
Zuständigkeiten zwischen Bund und den Ländern werden in Artikel 70 bis 75 im Grundgesetz aufgeteilt. Artikel 74 des Grundgesetzes besagt dabei, dass im Gesundheitswesen eine konkurrierende Gesetzgebung existiert. Dies bedeutet, dass länderspezifische Rechte bei der Verarbeitung von personenbezogenen Daten zu beachten ist.\footnote{Vgl. \cite[S. 8]{Schubert2014}} Besonders bei Krankenhäusern, welche als wichtige Akteure bei der Bereitstellung von Gesundheitsdaten angesehen werden, gelten landes- und kirchenrechtliche Bestimmungen. Diese beschreiben, wie der Umgang mit den Gesundheitsdaten der Patienten zu handhaben ist.\footnote{Vgl. \cite[S. 21]{OrientierungshilfezumGesundheitsdatenschutz2018}}
Trotz der teilweise länderspezifischen Regelungen für den Umgang mit Gesundheitsdaten wird für diese Arbeit nur das europäische, beziehungsweise Deutschland weit gültige Datenschutzrecht betrachtet, denn eine Betrachtung sämtlicher, innerhalb der Bundesländer spezifischen Regelungen den Umfang dieser Arbeit übersteigen würde. Für den realen Einsatz, des in dieser Arbeit beschriebenen Dienstes, wäre somit eine zusätzliche Prüfung des Länderrechts notwendig.

\subsubsection{Betreiber der Datenaustauschplattform}\label{subsection:bdd}
Bei den Betreibern des Messenger Dienstes handelt es sich um die Krankenhäuser. Für Krankenhäuser sind neben der Sicherheit und Robustheit des Dienstes vor allem Punkte wie Hosting und Integration in die bestehende IT-Infrastruktur relevant.
Mögliche Beispiele für Anforderungen der Krankenhäuser wären, dass das selbstständige Hosten des Dienstes, aufgrund der Sensibilität der über den Dienst transferierten Daten, für Krankenhäuser die bevorzugte Art und Weise des Betriebs ist. Die funktionalen Anforderungen an einen solchen Messanger Dienst werden in Kapitel x gelistet, dabei wird sich auf auf die für den Betrieb des Dienstes essentiellen Funktionen fokussiert, welche sich durch die Experteninterviews ergaben. Funktionen, welche diesen essentiellen Kern übersteigen können dem Dienst zwar einen erheblichen Mehrwert bieten, diese sind allerdings nicht Fokus dieser Arbeit und sollten bei einer Vortführung dieser gemeinsam mit den Krankenhäusern erörtert werden.
//Wird durch Interview noch ergänzt

\subsubsection{Nutzer der Datenaustauschplattform}\label{subsection:ndd}
Das Krankenhauspersonal ist als Nutzer des Messangers ebenso von enormer Wichtigkeit, da der Dienst nur dann einen Mehrwert bringt, wenn er vom Personal angenommen und genutzt wird und zu einer Verbesserung der krankenhausinternen Kommunikation beitragen kann. Aus diesem Grund muss bei der Implementierung und Entwicklung der Plattform eng mit dem Personal zusammengearbeitet werden, um sicher zu stellen dass der Dienst allgemein angenommen wird und einen mehrwert bieten kann. Da der Fokus dieser Arbeit auf der Erstellung der technischen Basis einens solchen Dienstes Anhand der gesetzlichen Vorlagen liegt, wird dieser Themenbereich nur begrenzt in dieser Arbeit behandelt. 

Durch die geführten Experteninterviews wird dennoch sichergestellt, das die Grundlegenden Ansprüche des Krankenhauspüersonals an den Messanger eingehalten werden (mehr dazu in Kapitel x) und bei einen Weiterführung der Forschung auf diese aufgebaut werden kann.

\subsubsection{Hersteller der Datenaustauschplattform}\label{subsection:hdd}
Der Hersteller des Messengerdienstes hat sicherzustellen, dass die notwendigen Anforderungen an die Software eingehalten werden.
Der in dieser Arbeit erstellte Anforderungskatalog und Prototype soll dabei für mögliche Hersteller eines solchen Dienstes als Unterstützung und Leitfaden zu einhaltung der gesetzlichen Anforderungen dienen.

\section{DSGVO Anforderungen}\label{section:dsgvo}
Das folgende Kapitel behandelt die für die Entwicklung des Krankenhaus Messenger Dienstes relevanten DSGVO Regelungen. Die relevanten Gesetzestexte werden dabei im folgenden zu Themenbereichen zusammen geordnet und ihr Effekt auf die Messenger Plattform erläutert. Dabei wird sich an der vom Bundesministerium für Wirtschaft und Energie herausgegebenen Orientierungshilfe zum Gesundheitsdatenschutz orientiert.

\subsection{Maßnahmen zur Wahrung der Nutzerrechte}\label{subsection:mzwdn}
Der Schutz der Nutzer einer Plattform gegenüber ihrer Betreiber ist ein zentraler Pfeiler der DSGVO. So besagt die DSGVO, dass Nutzer zunächst durch eine Datenschutzerklärung über die Verarbeitung der Daten informiert werden muss und ihm die Möglichkeit geboten werden muss dieser einzuwilligen oder ihr zu widersprechen.\footnote{Vgl. \cite[S. 3]{OrientierungshilfezumGesundheitsdatenschutz2018}} Zudem muss der Nutzer in der Lage sein, Auskünfte über die ihm zugeordneten, gespeicherten Daten zu erhalten. Der Nutzer hat darüber hinaus das Recht darauf, der weiteren Verarbeitung der Daten zu wiedersprechen, wie auch deren Löschung zu beantragen. Auch muss es dem Nutzer möglich sein, die Daten zu einem anderen Anbieter zu portieren.\footnote{Vgl. \cite[S. 30 ff.]{OrientierungshilfezumGesundheitsdatenschutz2018}} Diese Rechte werden in Artikel 13 oder Artikel 14 der DSGVO behandelt.

Die Wahrung der Nutzerrechte gilt für den Messanger Dienst in doppelter Hinsicht.
Zum einen müssen die genannten Rechte gegenüber der Patienten eingehalten werden. Besonders das Recht auf Löschung der personenbezogenen Daten ist dabei relevant für den Messanger, da bei einer Forderung auf Löschung sichergestellt werden muss, dass die betroffenen personenbezogenen Daten auch aus sämtlichen über den Dienst geführten Chats entfernt werden. Zum Anderen sind die Maßnahmen zur Wahrung der Nutzerrechte aber auch für die Nutzer der Plattform selbst, also das Krankenhauspersonal zu beachten.

Daraus folgt, dass der Messanger Dienst technisch in der Lage sein muss, Nutzerdaten und Chatverläufe dediziert ausgeben und wenn gewollt, die Daten permanent von allen mit dem Dienst verbundenen Geräten löschen zu können.\footnote{Vgl. \cite[S. 36 ff.]{OrientierungshilfezumGesundheitsdatenschutz2018}}

\subsection{Vorgaben zur Datensicherheit}\label{subsection:vzd}
Der Schutz der Gesundheitsdaten gegenüber unbefugten Dritten gilt als besonders wichtig, da bei Fällen von Datenklau, Datenverlust oder Datenmanipulation der Betreiber des Dienstes für dies Haften muss und dies zudem einen direkten Einfluss auf die Behandlung von Patienten haben könnte.\footnote{Vgl. \cite[S. 3]{OrientierungshilfezumGesundheitsdatenschutz2018}} Aus diesem Grund wird das erforderliche Schutzniveau der Gesundheitsdaten von der DSGVO verbindlich konkretisiert.
Um den von der DSGVO geforderten, angemessenen Schutz sicherzustellen, muss zunächst das Schutzniveaus der Daten ermittelt werden.
Dies geschieht anhand eines risikobasierten Ansatzes. Die beurteilung liegt dabei in der Eigenverantwortung der Betreiber der Plattform. Die "Orientierungshilfe zum Gesundheitsdatenschutz" des Bundesministeriums für Wirtschaft und Energienennt nennt dabei folgende kriterien:
\footnote{Vgl. \cite[S. 41 ff.]{OrientierungshilfezumGesundheitsdatenschutz2018}}

- "Die Schwere eines möglichen Schadens beurteilt sich nach dem Gewicht des bedrohten Rechts bzw. der bedrohten Freiheit sowie danach, welche Schäden ihnen aus der Verarbeitung erwachsen können. Dabei sind sowohl materielle als auch immaterielle Schäden von Bedeutung. Je sensibler die Daten sind, desto größer ist die mögliche Schadenshöhe. Bei Gesundheitsdaten ist grundsätzlich von einer besonderen Schadenshöhe auszugehen."

- "Die zudem zu berücksichtigende Eintrittswahrscheinlichkeit meint den statistischen Erwartungswert, mit dem ein bestimmtes Schadensereignis eintreten wird."

- "Art der Verarbeitung: Artikel 4 Nr. 2 DSGVO nennt insbesondere das Erheben, das Erfassen, die Übermittlung, das Ordnen, die Speicherung, das Löschen und die Vernichtung."

- "Umfang der Verarbeitung: Menge der Personen, deren Daten in die Verarbeitung einfließen, sowie Menge der Daten, die dabei über eine Person erhoben werden. Dabei ist zu berücksichtigen, dass Daten umso engmaschiger miteinander verknüpft werden können, je größer die Datenmenge ist. Der Aussagegehalt einer Datenanalyse steigt zudem mit der wachsenden Anzahl von Personen, die in eine vergleichende Betrachtung einbezogen werden."

- "Umstände der Verarbeitung: Gemeint sind alle tatsächlichen und rechtlichen Gegebenheiten, die die Einzelheiten des Verarbeitungsprozesses bestimmen."

- Zwecke der Verarbeitung: Hiermit sind die Ziele gemeint, die mit der Verarbeitung verfolgt werden. Diese legt der Verantwortliche selbst fest."

Das angemessene Schutzniveau wird neben dem Risiko für die Rechte des Betroffenen zudem anhand der wirtschaftlichen Interessen des Unternehmens, wie zum Beispiel den Implementierungskosten, festgelegt. Unter den Implementierungskosten versteht man die wirtschaftlichen Ressourcen, welche aufgebracht werden müssen, um die erörterten Anforderungen an den Dienst zu integrieren.\footnote{Vgl. \cite[S. 42 f.]{OrientierungshilfezumGesundheitsdatenschutz2018}} Somit soll die Höhe der Implementierungskosten als Grenze der zumutbaren und zu erwartenden Sicherheitsmaßnahmen berücksichtigt werden.

Auf diese Weise soll sichergestellt werden, dass die Maßnahmen zum Schutz der Gesundheitsdaten dem gegebenen Stand der Technik und den bestehenden Marktstandards entspricht. Hierbei kann sich an den BSI-Grundsätzen, beziehungsweise der ISO 27000-Normreihe orientiert werden.\footnote{Vgl. \cite[S. 42 f.]{OrientierungshilfezumGesundheitsdatenschutz2018}}

Die anhand des vorher bestimmten Schutzniveaus festgelegten Anforderungen gilt es somit technisch und organisatorisch umzusetzen und somit einen angemessenen Schutz der Gesundheitsdaten zu gewährleisten. Beispiele für die angemessene Umsetzung technischer Maßnahmen wären unter anderem die Zugriffs- und Weitergabekontrolle der Daten, sowie deren Verschlüsselung oder eine Passwortsicherung. Unter technischen Maßnahmen werden aber auch bauliche Entscheidungen aufgefasst, welche vor dem Zugriff unbefugter Personen schützen.\footnote{Vgl. \cite[S. 41 ff.]{OrientierungshilfezumGesundheitsdatenschutz2018}}

Unter organisatorischen Maßnahmen verstehen sich vor allem die Rahmenbedingungen unter welchen die technischen Verarbeitungsprozesse ablaufen. Beispiele hierfür wären unter anderem das Vieraugenprinzip, Schulungen für die Betreiber des Dienstes, die Verpflichtung zu Mitarbeitererklärungen oder die Protokollierungen von Zugriffen. Es gilt dabei die Maßnahmen regelmäßig zu überprüfen und wenn notwendig zu aktualisieren. Das Unternehmen hat dabei nachzuweisen, dass die Sicherheit der Verarbeitung gewährleistet ist. 

Um die technischen Anforderungen für die Konzeption des Messenger Dienstes bestimmen zu können ist es also Notwendig, das angemessenen Schutzniveau für einen solchen Dienst zu bestimmen. Für diese Arbeit werden die von der Gesellschaft für Datenschutz im Dokument "Austausch von Gesundheitsdaten -  Datenschutzrechtliche Anforderungen an Datenaustauschplattformen im Gesundheitswesen" heraus gegebenen Empfehlungen für Anforderungen an eine Plattform, welche dem Schutzniveau des Messenger Dienstes entspricht, übernommen. Diese werden in Kapitel ... weiter besprochen und ihre praktische Umsetzbarkeit für die Konzeption des Dienstes diskutiert. 


\subsection{Anonymisierung der Daten}\label{subsection:add}
Durch das anonymisieren von personenbezogenen Daten entfallen die Vorgaben der DSGVO, da auf diese Weise keine Rückverfolgung der Daten zur Person, welcher diese ursprünglich zugeordnet waren, mehr möglich ist. Bei Gesundheitsdaten ist jedoch zu beachten, dass wirklich eine vollständige Anonymisierung vorliegt, da diese Art von Daten oft eine Vielzahl von Anhaltspunkten über die zugehörige Person enthalten. Hierzu zählen auch Information, welche von Dritten eingeholt werden können. Es sollte somit darauf geachtet werden, dass es durch die Anonymisierung nicht möglich ist, mit vertretbarem Aufwand die zu verarbeitenden Daten auf die zugehörige Person zurückzuführen, denn nur dann entfallen die Vorgaben der DSGVO.\footnote{Vgl. \cite[S. 5 f.]{OrientierungshilfezumGesundheitsdatenschutz2018}}

Bei der Umsetzung des Messenger Dienstes sollten somit Funktionen implementiert werden, welche es den Nutzern der Plattform erlauben, Daten auf eine einfache Weise zu anonymisieren, denn falls der Personenbezug nicht Relevant beim Teilen der Gesundheitsdaten über den Dienst ist und eine Anonymisierung möglich ist, sollte diese auch von den Nutzern durchgeführt werden. Auf diese Weise lässt sich das Risiko der Verletzung der DSGVO und der Schutz der personenbezogenen Daten am besten gewährleisten.

\subsection{Rechtfertigungsgründe für die Verarbeitung von Gesundheitsdaten}\label{subsection:rfdvvg}
Eine zentrale Grundlage des Datenschutzrechtes besteht darin, dass personenbezogene Daten nur dann übermittelt, gespeichert oder anderweitig verarbeitet werden dürfen, wenn es hierfür eine gesetzliche Grundlage existiert. Mögliche Rechtsgrundlagen werden in Artikel 6 der DSGVO aufgelistet.\footnote{Vgl. \cite[S. 5 f.]{OrientierungshilfezumGesundheitsdatenschutz2018}} Gesundheitsdaten fallen allerdings in eine besondere Kategorie der personenbezogenen Daten und gelten als besonders schützenswert. Aus diesem Grund gelten für Gesundheitsdaten neben den allgemeinen Datenschutzrechtlichen, zusätzliche, weitaus strengere Regelungen. Diese besonderen Regelungen beinhalten ein allgemeines Verbot zur Verarbeitung von Gesundheitsdaten, solange kein gesetzlich geregelter Ausnahmetatbestand vorliegt, aufgelistet in Artikel 9 Abs. 2–4 der DSGVO.\footnote{Vgl. \cite[S. 20 ff.]{OrientierungshilfezumGesundheitsdatenschutz2018}}

a) Die betroffene Person hat in die Verarbeitung der genannten personenbezogenen Daten für einen oder mehrere festgelegte Zwecke ausdrücklich eingewilligt, es sei denn, nach Unionsrecht oder dem Recht der Mitgliedstaaten kann das Verbot nach Absatz 1 durch die Einwilligung der betroffenen Person nicht aufgehoben werden,(...)

Um das Verarbeiten von Gesundheitsdaten über den Messenger Dienst zu ermöglichen, muss dementsprechend geprüft werden, ob eine Einwilligung des Patienten zum Austausch der Daten innerhalb des Krankenhauses bereits vorliegt. Wenn dies nicht der Fall ist, müsste eine solche Klausel in den Patienten AGBs des Krankenhauses ergänzt werden. Dieser Teil des Datenschutzgesetzes hat zwar keinen direkten Einfluss auf die technische Umsetzung des Messanger Dienstes, wurde an dieser Stelle aber dennoch behandelt, da es sich bei den Maßnahmen zur Wahrung der Nutzerrechte und das Einverständnis des Nutzers zur Verarbeitung der eigenen Daten ein so grundlegender Pfeiler für die Datenverarbeitung nach DSGVO ist.

\section{Anforderungen an die Plattform auf Basis der bestehenden Gesetzeslage}\label{section:aadpabsbg}
Nachdem im vorherigen Kaptitel bereits die für die Verarbeitung von Gesundheitsdaten relevanten Abschnitte der DSGVO herausgearbeitet wurden, werden diese allgemein gehaltenen Gesetzestexte der DSGVO im folgenden Kapitel in konkrete Anforderungen an den Messenger Dienst überführt. 

Der zu entwicklende Messanger Dienst fällt in die Kategorie einer Datenaustauschplattform für Gesundheitsdaten und dabei in die Unterkategorie einer E-Collaborationsplattform. Die Kategorisierung gibt dabei das in Kapitel x bereits aufgeführte Schutzniveau des Dienstes vor und dient als Orientierung für die Auswertung der DSGVO Richtlinien. 

Somit ist es für die Entwicklung des Dienstes notwendig, die breits herausgearbeiteten DSGVO Richtlinien auf ihre Implikationen für eine E-Collaborationsplattform wie den Messenger auszuwerten und die sich daraus ergebenden Anforderungen aufzustellen. 
DIe Basis für diese Auswertung bilden dabei die im Austausch von Gesundheitsdaten - Datenschutzrechtliche Anforderungen an Datenaustauschplattformen im Gesundheitswesen aufgestellten technischen und organisatorischen Anforderungen an eine Datenaustauschplattform für medizinische Daten. Diese Anforderungen werden im folgenden danach sortiert, in wie Weit sie die konkrete Entwicklung der Messanger Plattform betreffen und ihre genaue Bedeutung für die konkrete Umsetzung des Dienstes werden erläutert.
Das Ziel dieses Kapitels ist es somit die Fertigstellung eines Anforderungskatalogs, der die gesetzlich relevanten Anforderungen eines der Zielsetzung dieser Arbeit entsprechenden Messanger Dienstes beinhaltet. Auf dessen Basis soll anschließend die Umsetzung des Messanger Dienstes erarbeitet werden.

Das Dokument Austausch von Gesundheitsdaten - Datenschutzrechtliche Anforderungen an Datenaustauschplattformen im Gesundheitswesen nennt 99 Anfroderungen an eine E-Collaborationsplattform im Gesundheitswesen auf Basis der in der DSGVO festgelegten Richtlinien. EIne vollständige Liste aller Anforderungen findet sich im Anhang auf Seite x. Im folgenden werden nur die Anforderungen genannt und diskutiert, welche für technische Umsetzung des Messanger Dienstes relevant sind.


T Anforderung 1
„Jegliche Erhebung, Verarbeitung und Nutzung personenbezogener oder personenbeziehbarer Daten - insbesondere durch den Einsatz von Datenaustauschplattformen, die an das Internet angebunden sind - bedarf einer datenschutzrechtlich wirksamen Einwilligung des Betroffenen.“
T Anforderung 3
„Eine Einwilligung muss für den Betroffenen jederzeit mit Wirkung für die Zukunft widerrufbar sein.“

Anforderung 1 und 3 sind relevant, da sie für den Messanger bedeuten, dass dem Krankenhauspersonal die Möglichkeit gegeben werden muss, der Verarbeitung ihrer Nutzerdaten durch die Plattform zuzustimmen. Das gleiche gilt auch für die Patienten, deren Einwilligung über die für das Krankenhaus gültigen AGBs eingeholt werden muss, welche auch die allgemeine Verarbeitung der Patientendaten inerhalb des Krankenhauses beschreiben.


T Anforderung 4
„Jede Datenaustauschplattform muss Datenschutzhinweise veröffentlichen und darin die getroffenen Datensicherheitslösungen allgemeinverständlich beschreiben.“
T Anforderung 5
„Datenschutzhinweise müssen unmittelbar von der Startseite der Datenaustauschplattform aus aufrufbar bzw. erreichbar sein.“
T Anforderung 6
„Datenschutzhinweise müssen für den Betroffenen jederzeit abrufbar sein.“
T Anforderung 7
„Bei nachträglicher Änderung ist der Betroffene zu informieren und sein Einverständnis erneut einzuholen.“
T Anforderung 8
„Wird die Einwilligung des Betroffenen elektronisch eingeholt, so muss der Vorgang protokolliert werden und der Inhalt der Einwilligung für den Betroffenen jederzeit abrufbar sein.“
Der Messanger muss die Möglichkeit bieten, die Datenschutzhinweise Anforderung 4-8 entsprechend den Nutzern zur Verfügung zu stellen.
Die eigentliche Formulierung der Datenschutzhinwiese ist nicht Teil dieser Arbeit, die technische Umsetzung, um diese auf die geforderte Weise zur Verfügung zu stellen jedoch schon.

T Anforderung 9
„Personenbezogene oder personenbeziehbare Daten müssen pseudonymisiert werden, soweit dies nach dem Verwendungszweck möglich ist und keinen im Verhältnis zu dem angestrebten Schutzzweck unverhältnismäßigen Aufwand erfordert.“
Der Messanger Dienst sollte den Nutzern aufgrund dieser Anforderung funktionen zur Verfügung stellen, Daten wie zum Beispiel Bilddateien, die über den Dienst geteilt werden sollen auf eine einfache Art und Weise zu pseudonymisieren.

T Anforderung 13
„Der Betroffene muss der Nutzung von Cookies explizit zustimmen.“
T Anforderung 14
„Die Verwendung von Cookies durch Drittanbieter muss auf die Erstellung anonymer Nutzungsstatistiken beschränkt sein.“
T Anforderung 15
„Das Handling der Drittanbieter-Cookies muss durch den Betroffenen steuer- und nachvollziehbar sein.“
Die gesetzliche Anforderungen 13-15 an den Umgang mit Cookies ist vorallem für die Umsetzung der im Funktionsumfang festgehaltenen Anforderung des Webzugangs zum Messanger Dienst relevant. Der Messanger Dienst muss dementsprechend technisch die Möglichkeit bieten, über mögliche Cookies zu informieren und dem Nutzer die Zustimmung zur Nutzung von Cookies ermöglichen.

T Anforderung 18
„Die Erhebung, Verarbeitung und Nutzung von Daten zur Erstellung von Nutzungsstatistiken von Webportalen muss dokumentiert werden.“
T Anforderung 19
„Die Erstellung von Nutzungsstatistiken von Datenaustauschplattformen muss auf Basis anonymisierter Daten erfolgen.“
T Anforderung 20
„Eine Analyse des Nutzerverhaltens (einschließlich Geolokalisierung) auf der Basis gekürzter IP-Adressen ist ohne Einwilligung des Betroffenen möglich, wenn durch die Kürzung der IP-Adresse ein Personenbezug ausgeschlossen wurde.“

Aus Gründen der Privatsphäre ist von einem individuellen Tracken des Nutzerverhaltens auf der Plattform ab zu sehen, da es sich hierbei nicht um eine relevante Funktion für den einsatz des Messaganer Dienstes handelt. Bei der nicht personenbezogenen Auswertungen des Nutzerverhaltens, zum Beispiel zur Überprüfung der Anahme des Dienstes seitens der Nutzer, sind Anforderungen 18-20 zu beachten. Allgemein fällt die Analyse nicht in die Zielsetung der Arbeit und wird deshalb im folgenden nicht weiter besprochen.

T Anforderung 31
„Alle Personen, welche auf die gespeicherten Gesundheitsdaten zugreifen können, müssen vor dem erstmaligen Zugriff auf die Wahrung des Datengeheimnisses verpflichtet worden sein.“
Für den Messanger Dienst muss eine technische Lösung implementiert werden, welche eine solche es erlaubt, dem Nutzer eine solche Wahrung zu präsentieren.

T Anforderung 32
„Ein anonymer Zugriff des Plattformbetreibers sowie der Nutzer der Plattform auf personenbezogene oder personenbeziehbare Gesundheitsdaten ist zu unterbinden.“
Es muss sicher gestellt werden das der Zugriff auf den Dienst zu jeder Zeit einer realen Person zugeordnet werden kann, dies gilt auch für den direkten Zugriff auf die mit dem Dienst verbundenen Datenbanken durch Administratoren.

T Anforderung 33
„Daten über den Ablauf des Zugriffs oder der sonstigen Nutzung sind unmittelbar nach dem Zugriff bzw. ggfs. nach erfolgter Abrechnung der Dienstenutzung zu sperren und nur zu entsperren, wenn eine gesetzliche Bestimmung (z.B. Auskunftsersuchen eines Betroffenen) dies erlaubt.“
Der Messanger muss es erlauben, Daten zeitlich begränzt zu Teilen, damit sicher gestellt werden kann, dass  Daten nur für die Dauer ihres bnötigten Verwendungszwecks über den Messanegr abrufbar sind.

T Anforderung 34
„Die Nutzung von Telemedien muss gegen die Kenntnisnahme unberechtigter Dritter geschützt werden.“
Eine Anmeldung und Verifizierung des Nutzers muss vor dem Zugriff auf den Messanger erfolgen. Ebenso muss der Zugriff auf die dem Messanger zugeornete Software und Datenbanken geschützt werden. Dies muss durch eine teschnische Lösung sichergestellt werden.

T Anforderung 35
„Es muss protokolliert werden, wer wann auf welche personenbezogenen oder personenbeziehbaren Daten mit welcher Berechtigung zugegriffen hat.“

T Anforderung 36
„Es muss protokolliert werden, wer wann welche personenbezogenen oder personenbeziehbaren Daten mit welcher Berechtigung exportiert oder ausgedruckt hat. Zu jedem Datenexport oder Ausdruck ist die Eingabe einer Begründung notwendig, damit nachvollziehbar und überprüfbar ist, zu welchem Zweck ein Datenexport oder ein Ausdruck erfolgte.“

T Anforderung 37
„Es muss protokolliert werden, wer wann welche personenbezogenen oder personenbeziehbaren Daten gespeichert, verändert, gesperrt oder gelöscht hat. Bei jeder Löschung ist die Eingabe einer Begründung notwendig, damit nachvollziehbar und überprüfbar ist, wer aus welchem Grund welche Daten löschte.“

T Anforderung 38
„Es muss protokolliert werden, wer wann auf welche personenbezogenen oder personenbeziehbaren Daten durch die Nutzung erweiterter Systemprivilegien (z.B. durch den sogenannten „Notfall-Zugriff“) zugriff. Zu jedem Datenzugriff unter Nutzung von durch das System bereitgestellten Mechanismen zur Erweiterung der Zugriffsberechtigung ist die Eingabe einer Begründung notwendig, damit nachvollziehbar und überprüfbar ist, zu welchem Zweck ein Zugriff, ein Datenexport oder ein Ausdruck erfolgte.“


T Anforderung 41
„Der Betroffene muss die Möglichkeit haben, jederzeit Einblick in alle zu seiner Person gespeicherten Daten zu erhalten. Dies umfasst auch die Möglichkeit, Änderungen seiner gespeicherten Daten nachzuvollziehen.“
T Anforderung 42
„Der Betroffene muss die Möglichkeit haben, einen Ausdruck oder einen für ihn verwertbaren Export aller zu seiner Person gespeicherten Daten zu erhalten.“
T Anforderung 43
„Es muss eine Möglichkeit geben, auf Aufforderung des Betroffenen Daten zu seiner Person zu korrigieren.“

T Anforderung 45
„Wurden Daten vom Betreiber der Datenaustauschplattform an andere Stellen übermittelt, so sollte der Betreiber der Datenaustauschplattform, sofern zumutbar, diese Stellen über erfolgte Berichtigungen informieren, sofern diese Benachrichtigung im Interesse des Betroffenen liegt.“

T Anforderung 46
„Es muss eine Löschfunktion implementiert werden, welche eine Rekonstruktion gelöschter Informationen ausschließt.“

T Anforderung 48
„Liegt keine gesetzliche Grundlage zur Speicherung der Daten vor, sind die Daten auf Anweisung des Betroffenen unverzüglich zu löschen.“

T Anforderung 52
„Die Löschung der Daten ist unter Angabe des Löschgrunds sowie des Anwenders, der die Löschung vornahm, zu protokollieren.“

T Anforderung 58
„Da im Internet eine potenziell größere Gefährdung für einen unbefugten Zugriff auf personenbezogene Daten existiert, muss mindestens eine 2-Faktor-Authentifizierung erfolgen.“

T Anforderung 59
„Werden statische Passwörter zur Authentisierung eingesetzt, so müssen die Empfehlungen des BSI bzgl. der Generierung und des Umgangs eingehalten werden. D.h. es müssen technische und organisatorische Maßnahmen zu der Einhaltung der Empfehlungen des BSI getroffen werden.“

T Anforderung 60
„Authentisierungsgeheimnisse dürfen nur gesichert in Netzwerken übertragen werden, d.h. es muss eine verschlüsselte Datenübertragung entsprechend dem Stand der Technik eingesetzt werden.“

T Anforderung 61
„Passwörter und/oder entsprechende Formulareingaben dürfen nicht auf dem Client oder in seiner Umgebung unverschlüsselt gespeichert werden, eine Speicherung im Browser ist zu verhindern.“

T Anforderung 62
„Nach wiederholter fehlerhafter Authentisierung muss der Zugang gesperrt werden.“

T Anforderung 63
„Die Sitzung muss gesperrt oder beendet werden, wenn der Anwender eine definierte Zeitspanne in der Sitzung keine Aktivitäten durchführte (Sitzungs-Zeitlimit, Session Timeout).“

T Anforderung 64
„Ein Prozess zur Rücksetzung bzw. Entsperrung von gesperrten Zugangskennungen ist einzurichten, zu beschreiben und anzuwenden.“

T Anforderung 66
„Benutzerkennungen, welche über einen definierten Zeitraum nicht benutzt wurden, sind zu sperren bzw. auf inaktiv zu setzen.“

T Anforderung 70
„Im Berechtigungs- und Rollenkonzept muss beschrieben sein, welche Funktionsrollen nicht miteinander vereinbar sind und somit nicht von einer Person gleichzeitig wahrgenommen werden dürfen. Es sind technische und organisatorische Maßnahmen zu ergreifen, um diese Trennung sicherzustellen.“

T Anforderung 71
„Eine Kombination von Rollen bzw. Zugriffsrechten für eine Person, welche der Person mehr Rechte auf Datenzugriffe erteilt, als für ihre Aufgabe nötig ist, ist zu verhindern. Es sind technische und organisatorische Maßnahmen zu ergreifen, um dies sicherzustellen.“

T Anforderung 73
„Protokolldaten sind gegen unbefugten Zugriff in geeigneter Weise entsprechend dem Stand der Technik zu schützen.“

T Anforderung 81
„Die Übertragung personenbezogener oder personenbeziehbarer Gesundheitsdaten zwischen Clients und Servern wie auch zwischen Servern selbst muss entsprechend dem jeweiligen Stand der Technik generell verschlüsselt erfolgen.“

T Anforderung 84
„Dieser Prozess muss datenschutzgerechte Löschverfahren beinhalten.“

T Anforderung 86
„Gesundheitsdaten sind entsprechend dem Stand der Technik verschlüsselt in der Datenbank zu speichern, so dass bei administrativen Zugriffen Wartungspersonal keinen unbefugten Zugriff auf die gespeicherten Daten erhalten kann.“

T Anforderung 87
„Nur Personen, die laut Berechtigungskonzept zu einer Eingabe berechtigt sind, dürfen personenbezogene oder personenbeziehbare Daten in eine Datenaustauschplattform eingeben.“

T Anforderung 88
„Nur Personen, die laut Berechtigungskonzept zu einem Datenimport berechtigt sind, dürfen einen Import personenbezogener oder personenbeziehbarer Daten in eine Datenaustauschplattform durchführen oder veranlassen.“

T Anforderung 89
„Sowohl die Eingabe wie auch der Import personenbezogener oder personenbeziehbarer Daten muss protokolliert werden.“

T Anforderung 93
„Es muss ein Backup-Konzept vorhanden sein, welches gewährleistet, dass die Daten nach einem Vorfall in angemessener Zeit wieder zur Verfügung gestellt werden können. In diesem Backup-Konzept muss berücksichtigt werden, dass nur berechtigte Personen Zugriff auf Backup-Daten erlangen können.“

T Anforderung 95
„Entsprechend der festzulegenden Anforderung an die Verfügbarkeit der Datenaustauschplattform müssen Notfalleinrichtungen vorhanden sein.“

//Abschliende Prüfung ob punkte umgesetzt sind, nach der Konzeption:
Tabelle mit Punkten und Hacken auf (Umgestezt, teilweise Umgestezt, Nicht umgestezt+ Kommentar)

\section{Anforderungen an den Funktionumfang}\label{section:pdimsbd}
Neben den gesetzlichen Anforderungen an den Messanger Dienst muss auch der eigentliche Funktionsumfang definiert werden. Die für diese Arbeit geführten Experteninterviews haben gezeigt, dass es eine Vielzahl an möglichen Funktionen gibt, welche den Funktionsumfang von herkömmlichen Messangern weit überschreiten, aber speziel für den einsatz in Krankenhäusern einen Mehrwert bieten können.
Da der Fokus dieser Arbeit allerdings auf der Umsetzung der gesätzlichen Anforderungen liegt, werden im folgenden nur Funktionen gesammelt, welche bei einem allgemein auf den professionellen Gebrauch ausgelegten Messanger Dienst zu erwarten sind und für den einsatz in Krankenhäusern hilfreich sind. Zudem soll aber auch bei der noch folgenden Konzeption des Messangers sichergestellt werden, das funktionale erweiterungen dennoch möglich sind.

Die folgende Tabelle listet Funktionen professioneller Kommunkationsdienste, welche für den einsatz des Messanger Dienstes notwendig sind, basierend auf den geführten Experteninterviews:
- Einzel und Gruppen Chats 
- Themenspezifische Chaträume
- Administrative Funktionen zur Vergabe von Rechten innerhalb des Dienstes 
- Video und Audio telefonie
- Anbindung an das Personalregister des Krankenhauses
- Zugriff auf den Dienst per App, Desktop und Web Oberfläche
- Durchsuchbares Personenregister 

//

\section{Prüfung der im medizinischen Sektor bestehenden Dienste}\label{section:pdimsbd}
//Kapitel wird noch hinter den entwickelten Anforderungskatalog gerückt...// 

Nachdem sich die vorherigen Kapitel mit den Anforderungen an einen Messanger Dienst für Krankenhäuser beschäftigt haben, behandelt das folgende Kapitel das bereits auf dem Markt vorhandene Angebote an Messenger Diensten und Prüft diese anhand des entwickelten Anforderungskatalog. Hierdurch soll zum einen sicher gestellt werden, dass nicht bereits ein, die Zielsetzung der Arbeit erfüllender,Dienst existiert, zum anderen sollen hierbei die Probleme der bereits existierenden Dienste für diesen Anwendungsfall verdeutlicht werden. Die Auswahl der im folgenden analysierten Dienste basieren auf den in den Experteninterviews genannten Lösungen, die in Krankenhäusern bereits vereinzelt für organisatorische, nicht aber für Gesundheitsdaten betreffende Kommunikation genutzt werden.

WhatsApp: Bei WhatsApp handelt es sich um einen der populärsten Messenger Dienste Deutschlands. Der dem Facebook Konzern zugehörige Dienst ist vor allem für den privaten gebrauch ausgelegt, bietet aber dennoch einen großen Funktionsumfang, mit der Integration von Audio und Video Telefonaten, der Bedienung des Dienstes außerhalb der App über ein Web Interface und der Möglichkeit Nachrichten End zu End zu verschlüsseln.

-Speicherung von Daten direkt auf dem Handy

-Keine Administration möglich 

-An die Telefonnummer gebunden

-Kein Selfhosting möglich 

-Online Speicherung der Daten intransparent

-Programmcode nicht einsehbar

-Keine zwingende End zu End Verschlüsselung

-Eigentum eines amerikanischen Unternehmens

Slack: Slack ist eine proprietäre Geschäftskommunikationsplattform, die vom amerikanischen Softwareunternehmen Slack Technologies entwickelt wurde. Slack bietet viele Funktionen im IRC-Stil, einschließlich Chatrooms, die nach Themen, privaten Gruppen und Direktnachrichten organisiert sind. Zudem unterstützt auch Slack Video und Audio Telefonate, bietet aber gegenüber WhatsApp eine Reihe von Funktionen, welche speziell auf Enterprise Kunden ausgelegt sind. Hierzu zählen unter anderem eine Reihe von administrativer Funktionen, welche Unternehmen eine zentrale Verwaltung des Dienstes ermöglicht, oder die Option bestehende Nutzerverwaltungen an die Software anzubinden.
-Kein Selfhosting möglich

-Programmcode nicht einsehbar

-Keine zwingende End zu End Verschlüsselung

-Eigentum eines amerikanischen Unternehmens

Microsoft Teams: Microsoft Teams ist eine proprietäre Geschäftskommunikationsplattform, die von Microsoft als Teil der Microsoft 365-Produktfamilie entwickelt wurde. Teams konkurriert hauptsächlich mit dem vergleichbaren Dienst Slack, da beide Softwareprodukte primär auf den Einsatz in Unternehmen ausgelegt sind. Teams bietet ebenfalls Möglichkeiten für Chat- und Videokonferenzen und das Teilen von Dateien. Der Hauptunterschied zwischen Microsofts Lösung und Slack liegt in der Präsentation, somit unterscheiden sich zwar UI und UX, die Kernfunktionen einer modernen Kommunikationsplattform decken aber beide Dienste gleichermaßen ab.

-Kein Selfhosting möglich 

-Programmcode nicht einsehbar

-Keine zwingende End zu End Verschlüsselung

-Eigentum eines amerikanischen Unternehmens

Siilo 
Bei Siilo handelt es um den speziell für den medizinischen Sektor entwickelten Messenger Dienst aus Amsterdam.
Siilo bietet dabei eine Reihe von speziell für den medizinischen Sektor angepassten Funktionen, wie die Möglichkeit Bildinhalte zu anonymisieren und wirbt mit einem besonders hohen Fokus auf Datenschutz und Datensicherheit.

-Kein Selfhosting möglich

-Programmcode nicht einsehbar
