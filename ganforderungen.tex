\chapter{Gesetzliche Anforderungen an Messanger in medizinischen Einrichtungen}\label{chapter:ganforderungen}
Trotz eines großen Angebotes an Messenger Diensten für den privaten, wie auch den professionellen gebrauch in Unternehmen, gibt es zum aktuellen Stand keine anerkannte Lösung für einen Dienst welcher im Krankenhausbereich eingesetzt werden kann, da sie den hohen Anforderungen der DSGVO an die Verarbeitung von Gesundheitsdaten nicht gerecht werden. Dies zeigt wie wichtig eine genaue Analyse der Gesetzeslage ist und welche Implikationen der Datenschutz auf die Umsetzung des Dienstes hat.

\section{DSGVO Anforderungen}\label{section:dsgvo}
Das folgende Kapitel behandelt die für die Entwicklung des Krankenhaus Messenger Dienstes relevanten DSGVO Regelungen. Die relevaten Gesetzestexte werden dabei im folgenden zu Themenbereichen zusammengeordnet und ihr Effekt auf die finale Messenger Plattform erläutert. Dabei wird sich an der vom Bundesministerium für Wirtschaft und Energie herausgegebenen "Orientierungshilfe zum Gesundheitsdatenschutz" orientiert.

Dass die Gesundheitsdaten zum Zweck der Gesundheits vorsorge verarbeitet werden sollen, reicht dagegen in der Regel nicht aus, um eine Zulässigkeit zu begründen. Denn die Gesundheitsvorsorge ist nur dann als Ausnahme vom Verbot der Verarbeitung von Gesundheitsdaten einschlägig, wenn z. B. ein (Berufs-)Geheimnisträger die Verarbeitung verantwortet (Artikel 9 Abs. 3 DSGVO).

\subsection{Rechtfertigungsgründe für die Verarbeitung von Gesundheitsdaten}\label{subsection:rfdvvg}
Eine zentrale Grundlage des Datenschutzrechtes besteht darin, dass personenbezogene Daten nur dann übermittelt, gespeichert oder anderweitig verarbeitet werden dürfen, wenn es hierfür eine gesetzliche Grundlage existiert.

Mögliche Rechtsgrundlagen werden in Artikel 6 der DSGVO aufgelistet. Ein Beispiel für eine mögliche Rechtsgrundlage wäre die Einwilligung oder ein Vertrag mit der betroffenen Person, welcher die Nutzung und Verarbeitung der Daten erlaubt.

Gesundheitsdaten fallen allerdings in eine besondere Kategorie der personenbezogenen Daten und gelten als besonders schützensewert. Aus diesem Grund gelten für Gesundheitsdaten neben den allgemeinen Datenschutzrechtlichen, zusätzliche, weitaus strengere Regelungen.

Diese besonderen Regelungen beinhalten ein allgemeines Verbot zur Verarbeitung von Gesundheitsdaten, sollange kein gesetzlich geregelter Ausnahmetatbestand vorliegt, aufgelistet in Artikel 9 Abs. 2–4  der DSGVO.

"a) Die betroffene Person hat in die Verarbeitung der genannten personenbezogenen Daten für einen oder mehrere festgelegte Zwecke ausdrücklich eingewilligt, es sei denn, nach Unionsrecht oder dem Recht der Mitgliedstaaten kann das Verbot nach Absatz 1 durch die Einwilligung der betroffenen Person nicht aufgehoben werden,"

Um das Verarbeiten ist von Gesundheitsdaten über den Messenger Dienst zu ermöglichen, muss dementsprechend geprüft werden, ob eine Einwilligung des Patienten zum Austausch der Daten innerhalb des Krankenhauses bereits vorliegt. Wenn dies nicht der Fall ist, müsste eine solche Klausel in den Patienten AGBs des Krankenhauses ergänzt werden. Dieser Teil des Datenschutzgesetzes hat zwar keinen direkten EInfluss auf die technische Umsetzung des Messanger Dienstes, wurde an dieser Stelle aber dennoch behandelt, da es sich bei den Maßnahmen zur Wahrung der Nutzerrechte und des Einverständnis des Nutzers zur Verarbeitung der eigenen Daten ein so grundlegender Pfeiler für die Datenverarbeitung nach DSGVO ist.

\subsection{Maßnahmen zur Wahrung der Nutzerrechte}\label{subsection:mzwdn}
Der Schutz der Nutzer einer Plattform gegenüber ihrer Betreiber ist ein zentraler Pfeiler der DSGVO. So besagt die DSGVO, dass Nutzer zunächst durch eine durch eine Datenschutzerklärung über die Verarbeitung seiner Daten informiert werden muss und ihm die Möglichkeit geboten werden muss dieser ein zu wolligen oder ihr zu wiedersprechen. Zudem muss der Nutzer in der Lage sein, Auskünfte über die ihm zugeordneten, gespeicherten Daten zu erhalten. Der Nutzer hat darüber hinaus das Recht darauf, der weiteren Verarbeitung der Daten zu wiedersprechen, wie auch die Löschung dieser und die Portierung der Daten zu einem anderen Anbieter.

Die Wahrung der Nutzerrechte gilt für den Messanger Dienst in doppelter Hinsicht. 
Zum einen müssen die genannten Rechte gegenüber der Patienten eingehalten werden. Besonders das Recht auf Löschung der personenbezogenen Daten ist dabei Relevant für den Messanger, da bei einer Forderung auf Löschung sichergestellt werden muss, dass die betroffenen personenbezogenen Daten auch aus den Chat Verläufen des Dienstes entfernt werden. Zum Anderen sind die Maßnahmen zur Wahrung der Nutzerrechte aber auch für die Nutzer der Plattform selbst, also das Krankenhauspersonal zu beachten.

Der Messanger Dienst muss also technisch in der Lage sein, Nutzerdaten und Chatverläufe dediziert Ausgeben und wenn gewollt, diese Daten permanent von allen, mit dem Dienst verbundenen Geräten löschen zu können. 

\subsection{Vorgaben zur Datensicherheit}\label{subsection:vzd}
Der Schutz der Gesundheitsdaten gegenüber unbefugten Dritten gilt als besonders wichtig, da Fälle von Datenklau, Datenverlust oder Datenmanipulation einen direkten Einfluss auf die Behandlung von Patienten haben kann. Aus diesem Grund wird das erforderliche Schutzniveau der Gesundheitsdaten von der DSGVO verbindlich konkretisiert.
Um das von der DSGVO geforderte, angemessenen Schutz sicher zu stellen, muss zunächst das Schutzniveaus der Daten ermittelt werden.
Dies geschieht anhand eines risikobasierten Ansatzes. Die beurteilung liegt dabei in der Eigenverantwortung der Betreiber der Plattform. Die Kriterien für die Bewertung eines Dienstes lauten nach der "Orientierungshilfe zum Gesundheitsdatenschutz" wie folgt:
(S. 41)

-Die Schwere eines möglichen Schadens beurteilt sich nach dem Gewicht des bedrohten Rechts bzw. der bedrohten Freiheit sowie danach, welche Schäden ihnen aus der Verarbeitung erwachsen können. Dabei sind sowohl materielle als auch immaterielle Schäden von Bedeutung. Je sensibler die Daten sind, desto größer ist die mögliche Schadenshöhe. Bei Gesundheitsdaten ist grundsätzlich von einer besonderen Schadenshöhe auszugehen.

-Die zudem zu berücksichtigende Eintrittswahrscheinlichkeit meint den statistischen Erwartungswert, mit dem ein bestimmtes Schadensereignis eintreten wird.

-Art der Verarbeitung: Artikel 4 Nr. 2 DSGVO nennt insbesondere das Erheben, das Erfassen, die Übermittlung, das Ordnen, die Speicherung, das Löschen und die Vernichtung.

-Umfang der Verarbeitung: Menge der Personen, deren Daten in die Verarbeitung einfließen, sowie Menge der Daten, die dabei über eine Person erhoben werden. Dabei ist zu berücksichtigen, dass Daten umso engmaschiger miteinander verknüpft werden können, je größer die Datenmenge ist. Der Aussagegehalt einer Datenanalyse steigt zudem mit der wachsenden Anzahl von Personen, die in eine vergleichende Betrachtung einbezogen werden.

-Umstände der Verarbeitung: Gemeint sind alle tatsächlichen und rechtlichen Gegebenheiten, die die Einzelheiten des Verarbeitungsprozesses bestimmen.

-Zwecke der Verarbeitung: Hiermit sind die Ziele gemeint, die mit der Verarbeitung verfolgt werden. Diese legt der Verantwortliche selbst fest.

Das angemessene Schutzniveau wird neben dem Risiko für die Rechte des Betroffenen zudem an den wirtschaftlichen Interesse des Unternehmens, wie zum Beispiel die Implementierungskosten ausgedrückt. Unter den Implementierungskosten versteht man die wirtschaftlichen Ressourcen, welche aufgebracht werden müssen, um die Vorgaben in einem System zu integrieren. Somit soll die Höhe der Implementierungskosten als Grenze der zumutbaren und zu erwartetenden Sicherheitsmaßnahmen berücksichtigt werden.

Auf diese Weise soll sichergestellt werden, dass die Maßnahmen zum Schutz der Gesundheitsdaten dem gegebenen Stand der Technik und den bestehenden Marktstandards entsprechen. Hierbei kann sich an den BSI-Grundsätzen, beziehungsweise der ISO 27000-Normreihe orientiert werden.

Die anhand vorher bestimmten Schutzniveaus festgelegten Anforderungen gilt es somit technisch und organisatorisch umzusetzen und somit einen angemessenen Schutz der Gesundheitsdaten zu gewährleisten. Beispiele für die angemessene Umsetzung technischer Maßnahmen wären unter anderem die Zugriffs- oder Weitergabekontrolle der Daten, sowie deren Verschlüsselung oder eine Passwortsicherung. Unter technischen Maßnahmen werden aber auch bauliche Entscheidungen aufgefasst, welche vor dem Zugriff unbefugter Personen schützt.

Unter organisatorischen Maßnahmen verstehen sich vor allem die Rahmenbedingungen unter welchen die technischen Verarbeitungsprozesse ablaufen. Beispiele hierfür wären unter anderem das Vieraugenprinzip, Schulungen für die Betreiber des Dienstes, die Verpflichtung zu Mitarbeitererklärungen oder die Protokollierungen von Zugriffen.
Es gilt dabei die Maßnahmen regelmäßig zu überprüfen und wenn notwendig zu aktualisieren. 

Um die technischen Anforderungen für die Konzeption des Messenger Dienstes bestimmen zu können ist es also Notwendig das angemessenen Schutzniveau für einen solchen Dienst zu bestimmen. Für diese Arbeit werden die von der Gesellschaft für Datenschutz im Dokument "Austausch von Gesundheitsdaten -  Datenschutzrechtliche Anforderungen an Datenaustauschplattformen im Gesundheitswesen" heraus gegebenen Empfehlungen für die technischen Anforderungen an eine Plattform mit dem Schutzniveau des Messenger Dienstes übernommen. Diese werden in Kapitel ... weiter besprochen und ihre praktische Umsetzbarkeit für die Konzeption des Dienstes diskutiert. 

Abschließend ist fest zu halten, dass ein Unternehmen nachzuweisen hat, dass die Sicherheit der Verarbeitung gewährleistet ist. 
Der Nachweis der Erfüllung aller dem Schutzniveau entsprechenden Anforderungen an den Dienst kann dabei auch die Einhaltung genehmigter Verhaltensregeln (Artikel 40 DSGVO) oder die Prüfung über ein genehmigtes Zertifizierungsverfahren (Artikel 42 DSGVO) beinhalten.

\subsection{Anonymisierung der Daten}\label{subsection:add}
Durch das anonymisieren von personenbezogenen Daten entfallen die Vorgaben der DSGVO, da auf diese Weise keine Rückverfolgung der Daten zur Person welcher diese ursprünglich zugeordnet waren mehr möglich ist. Bei Gesundheitsdaten ist jedoch zu beachten, dass wirklich eine vollständige Anonymisierung vorliegt, da diese Art von Daten oft eine vielzahl von Anhaltspunkten über die zugehörige Person enthalten. Dazu zählen auch Information, welche von Dritten eingeholt werden können. Es sollte somit darauf geachtet werden, dass es durch die Anonymisierung nicht möglich ist, mit vertretbarem Aufwand die zu verarbeitenden Daten der zugehörigen Person zu zu Ordnen, denn nur dann entfallen die Vorgaben der DSGVO. 

Bei der Umsetzung des Messenger Dienstes sollten somit Funktionen implementiert werden, welche es den Nutzern der Plattform erlauben, Daten auf eine einfache Weise zu anonymisieren, denn falls der Personenbezug nicht Relevant beim Teilen der Gesundheitsdaten über den Dienst ist und eine Anonymisierung möglich ist, sollte diese auch vor dem Teilen durchgeführt werden. Auf diese Weise lässt sich das Risiko der Verletzung der DSGVO und der Schutz der personenbezogenen Daten am besten gewährleisten.

\section{Haftung}\label{section:haftung}

\section{Zusammenfassung gesetzlicher Anforderungen}\label{section:zga}

