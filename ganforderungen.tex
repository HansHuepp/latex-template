\chapter{Anforderungsanalyse an einen Messenger für den Einsatz in medizinischen Einrichtungen}\label{chapter:ganforderungen}
Trotz eines großen Angebotes an Messengerdiensten für den privaten, wie auch den professionellen Gebrauch in Unternehmen, gibt es zum aktuellen Stand keine anerkannte Lösung für einen Dienst, welcher im Krankenhausbereich eingesetzt werden kann, da sie den hohen Anforderungen der DSGVO an die Verarbeitung von Gesundheitsdaten nicht gerecht werden.\footnote{Vgl. \cite[S. 1 ff.]{Datenschutzkonferenz2019}} Dies zeigt wie wichtig eine genaue Analyse der Gesetzeslage ist und welche Bedeutung der Datenschutz für die Umsetzung des Dienstes hat.

\section{Übersicht der Anforderungssteller}\label{chapter:hintergrund}
Um einen Einstiegspunkt in die Thematik der gesetzlichen Anforderungen an Messengerdienste im Krankenhausbereich zu ermöglichen, müssen zunächst die einzelnen Anforderungssteller erörtert werden.

Hierzu wird mit Hilfe der Forschungsmethodik nach Mayring eine systematische Erschließung der Textmaterialien vorgenommen, um auf dieser Basis eine qualitative Inhaltsanalyse durchzuführen. Entsprechend der Methodik werden die gewonnen Informationen zusammengefasst, durch Explikation werden komplexe Sachverhalte mit zusätzlichem Material ergänzt und durch Strukturierung werden die für diese Arbeit entscheidenden Aspekte des Materials herausgearbeitet. 

Im folgenden Kapitel werden deshalb die Anforderungssteller, welche bei der Entwicklung eines Messengerdienstes im Krankenhausbereich zu beachten sind, gelistet und die Relevanz der Forderungen auf Basis der für den Umfang dieser Arbeit vorgenommen Einschränkungen analysiert. Der vom Bundesverband Gesundheits-IT e. V. herausgegebene Report \glqq Austausch von Gesundheitsdaten - Datenschutzrechtliche Anforderungen an Datenaustauschplattformen im Gesundheitswesen\grqq{} nennt für eine, dem Massenger Dienst entsprechende Datenaustauschplattform, folgende Akteure: Hersteller der Datenaustauschplattform, Betreiber der Datenaustauschplattform, Datenlieferant, Nutzer der Datenaustauschplattform und Betroffene bei Nutzung der Datenaustauschplattform.\footnote{Vgl. \cite[S. 13]{Bundesverband-Gesundheits-IT-e.V.2016}} Diese werden im Folgenden vorgestellt und erläutert, in wie fern sie für die Zielsetzung dieser Arbeit relevant sind.

\subsection{Betroffene bei Nutzung des Messengers}\label{subsection:bbndd}
Bei den Betroffenen handelt es sich im wesentlichen um das Personal, welches den Dienst nutzt und die Patienten, deren Daten über diese Plattform ausgetauscht werden, denn bei beiden Personengruppen findet eine Datenverarbeitung durch den Messengerdienst statt.\footnote{Vgl. \cite[S. 13]{Bundesverband-Gesundheits-IT-e.V.2016}} Die Grundlage für den Schutz dieser beiden Personengruppen wird dabei über das Datenschutzrecht festgelegt, welches im Folgenden kurz vorgestellt wird:

Datenschutz-Grundverordnung (DSGVO): Die DSGVO ist eine Verordnung der Europäischen Union, welche die private, wie auch die öffentliche Verarbeitung personenbezogener Daten EU-weit vereinheitlichend reglementiert. Dadurch soll einerseits der Schutz personenbezogener Daten innerhalb der Europäischen Union sichergestellt und andererseits der freie Datenverkehr innerhalb des europäischen Binnenmarktes gewährleistet werden. Die Einhaltung der DSGVO Verordnungen ist eine der obersten Prioritäten bei der Entwicklung des Messengerdienstes.\footnote{Vgl. \cite[S. 2 ff.]{Voigt2018}}Ein Verstoß wird mit erheblichen Bußgeldern geahndet. Da sich die DSGVO auf die Verarbeitung von personenbezogenen Daten in ihrer Gesamtheit bezieht, sind nicht alle Regelungen gleichermaßen relevant für die Entwicklung des Messengers. Besonders zu beachtende Regelungen sind \glqq das Recht auf Löschung\grqq{}, \glqq Das Recht auf Einverständnis\grqq{} und \glqq das Recht auf Datenübertragung\grqq. Die Einhaltung aller, für den Dienst relevanten Anforderungen, ist eine der zentralen Zielsetzungen dieser Arbeit, weshalb eine Auswertung der zu beachtenden Gesetze durchgeführt werden muss. Diese ist zu finden in Kapitel 2.3.

Länderrecht: Neben der europaweit geltenden DSGVO gibt es innerhalb der einzelnen Bundesländer zusätzliche Datenschutzbestimmungen, welche im jeweiligen Bundesrecht verankert sind und in einzelnen Fällen Einfluss auf die Verarbeitung von Gesundheitsdaten haben.
Zuständigkeiten zwischen Bund und den Ländern werden in Artikel 70 bis 75 im Grundgesetz aufgeteilt. Artikel 74 des Grundgesetzes besagt dabei, dass im Gesundheitswesen eine konkurrierende Gesetzgebung existiert. Dies bedeutet, dass länderspezifische Rechte bei der Verarbeitung von personenbezogenen Daten zu beachten sind.\footnote{Vgl. \cite[S. 8 f.]{Bundesverband-Gesundheits-IT-e.V.2016}} Besonders bei Krankenhäusern, welche als wichtige Akteure bei der Bereitstellung von Gesundheitsdaten angesehen werden, gelten landes- und kirchenrechtliche Bestimmungen. Diese beschreiben, wie der Umgang mit den Gesundheitsdaten der Patienten zu handhaben ist.\footnote{Vgl. \cite[S. 21]{Bundesaerztekammer2020}}
Trotz der teilweise länderspezifischen Regelungen für den Umgang mit Gesundheitsdaten wird für diese Arbeit nur das europäische, beziehungsweise deutschlandweit gültige Datenschutzrecht betrachtet, denn eine Betrachtung sämtlicher, innerhalb der Bundesländer geltenden Regelungen würde den Umfang dieser Arbeit übersteigen. Für den realen Einsatz des in dieser Arbeit beschriebenen Dienstes, wäre somit eine zusätzliche Prüfung des entsprechenden Länderrechts notwendig.

\subsection{Betreiber der Datenaustauschplattform}\label{subsection:bdd}
Bei den Betreibern des Messengerdienstes handelt es sich für diese Arbeit primär um Krankenhäuser.\footnote{Vgl. \cite[S. 13]{Bundesverband-Gesundheits-IT-e.V.2016}} Für Krankenhäuser sind neben der Sicherheit und Robustheit des Dienstes vor allem Punkte wie die Integration in die bestehende IT-Infrastruktur relevant.\footnote{Vgl. Experteninterview Georg Woditsch}
Mögliche Beispiele für Anforderungen der Krankenhäuser wären, dass das selbstständige Hosten des Dienstes, aufgrund der Sensibilität, der über den Dienst transferierten Daten, für Krankenhäuser die bevorzugte Art und Weise des Betriebs ist.\footnote{Vgl. Experteninterview Georg Woditsch} Die funktionalen Anforderungen an einen solchen Messengerdienst werden in Kapitel 2.4 gelistet, dabei wird sich auf die für den Betrieb des Dienstes essentiellen Funktionen fokussiert. Funktionen, welche diesen essentiellen Kern übersteigen, können dem Dienst zwar einen erheblichen Mehrwert bieten, diese sind allerdings nicht Fokus dieser Arbeit und sollten bei einer Fortführung der Forschung gemeinsam mit den Krankenhäusern erörtert werden.

\subsection{Nutzer der Datenaustauschplattform}\label{subsection:ndd}
Das Krankenhauspersonal ist als Nutzer des Messengers von enormer Wichtigkeit, da der Dienst nur dann einen Mehrwert bringt, wenn er vom Personal angenommen und genutzt wird und zu einer Verbesserung der krankenhausinternen Kommunikation beitragen kann.\footnote{Vgl. \cite[S. 13]{Bundesverband-Gesundheits-IT-e.V.2016}}\footnote{Vgl. Experteninterview Georg Woditsch} Aus diesem Grund muss bei der Implementierung und Entwicklung der Plattform eng mit dem Personal zusammengearbeitet werden, um sicherzustellen, dass der Dienst allgemein angenommen wird und einen Mehrwert bieten kann. Da der Fokus dieser Arbeit auf der Erstellung der technischen Basis einens solchen Dienstes, anhand der gesetzlichen Vorlagen, liegt, wird dieser Themenbereich nur begrenzt in dieser Arbeit behandelt. 

\subsection{Datenlieferant}\label{subsection:hdd}
Mit Datenlieferanten sind zum einen das Krankenhauspersonal, zum anderen die Patienten gemeint, deren Daten über den Dienst verarbeitet werden.\footnote{Vgl. \cite[S. 13]{Bundesverband-Gesundheits-IT-e.V.2016}} Welche Bedeutung diese zwei Parteien für den Messanger haben, wurde bereits unter dem Punkt \glqq Betroffene bei Nutzung des Messengers\grqq{} erläutert.

\subsection{Hersteller der Datenaustauschplattform}\label{subsection:hdd}
Der Hersteller des Messengerdienstes hat sicherzustellen, dass die notwendigen Anforderungen an die Software eingehalten werden.
Der in dieser Arbeit erstellte Anforderungskatalog soll dabei für mögliche Hersteller eines solchen Dienstes als Unterstützung und Leitfaden zur Einhaltung der gesetzlichen Anforderungen dienen.\footnote{Vgl. \cite[S. 13]{Bundesverband-Gesundheits-IT-e.V.2016}}

\section{Gesetzliche Anforderungen}\label{section:dsgvo}
Das folgende Kapitel behandelt die für die Entwicklung des Krankenhaus Messengerdienstes relevanten DSGVO Regelungen. Die relevanten Gesetzestexte werden dabei im Folgenden zu Themenbereichen zusammengeordnet und ihr Effekt auf die Messenger Plattform erläutert. Dabei wird sich an der vom Bundesministerium für Wirtschaft und Energie herausgegebenen Orientierungshilfe zum Gesundheitsdatenschutz orientiert. Im Folgenden werden deshalb die Maßnahmen zur Wahrung der Nutzerrechte, Vorgaben zur Datensicherheit, Maßnahmen zur Anonymisierung der Daten und die Rechtfertigungsgründe für die Verarbeitung von Gesundheitsdaten erläutert.

\subsection{Maßnahmen zur Wahrung der Nutzerrechte}\label{subsection:mzwdn}
Der Schutz der Nutzer einer Plattform gegenüber ihrer Betreiber ist ein zentraler Pfeiler der DSGVO. So besagt die DSGVO, dass der Nutzer zunächst durch eine Datenschutzerklärung über die Verarbeitung der Daten informiert werden muss und ihm die Möglichkeit geboten werden muss dieser einzuwilligen oder ihr zu widersprechen.\footnote{Vgl. \cite[S. 3]{Bundesaerztekammer2020}} Zudem muss der Nutzer in der Lage sein, Auskünfte über die ihm zugeordneten, gespeicherten Daten zu erhalten. Der Nutzer hat darüber hinaus das Recht darauf, der weiteren Verarbeitung der Daten zu widersprechen, wie auch deren Löschung zu beantragen. Auch muss es dem Nutzer möglich sein, die Daten zu einem anderen Anbieter zu portieren.\footnote{Vgl. \cite[S. 30 ff.]{Bundesaerztekammer2020}} Diese Rechte werden in Artikel 13 und Artikel 14 der DSGVO behandelt.

Die Wahrung der Nutzerrechte gilt für den Messengerdienst in doppelter Hinsicht.
Zum einen müssen die genannten Rechte gegenüber den Patienten eingehalten werden. Besonders das Recht auf Löschung der personenbezogenen Daten ist dabei relevant für den Messenger, da bei einer Forderung auf Löschung sichergestellt werden muss, dass die betroffenen personenbezogenen Daten auch aus sämtlichen über den Dienst geführten Chats entfernt werden. Zum anderen sind die Maßnahmen zur Wahrung der Nutzerrechte aber auch für die Nutzer der Plattform selbst, also das Krankenhauspersonal zu beachten.

Daraus folgt, dass der Messengerdienst technisch in der Lage sein muss, Nutzerdaten und Chatverläufe dediziert ausgeben und wenn gewollt, die Daten permanent von allen mit dem Dienst verbundenen Geräten löschen zu können.\footnote{Vgl. \cite[S. 36 ff.]{Bundesaerztekammer2020}}

\subsection{Vorgaben zur Datensicherheit}\label{subsection:vzd}
Der Schutz der Gesundheitsdaten gegenüber unbefugten Dritten gilt als besonders wichtig, da bei Fällen von Datenklau, Datenverlust oder Datenmanipulation der Betreiber des Dienstes für dies Haften muss und dies zudem einen direkten Einfluss auf die Behandlung von Patienten haben könnte.\footnote{Vgl. \cite[S. 3]{Bundesaerztekammer2020}} Aus diesem Grund wird das erforderliche Schutzniveau der Gesundheitsdaten von der DSGVO verbindlich konkretisiert.
Um den von der DSGVO geforderten, angemessenen Schutz sicherzustellen, muss zunächst das Schutzniveau der Daten ermittelt werden.
Dies geschieht anhand eines risikobasierten Ansatzes. Die Beurteilung liegt dabei in der Eigenverantwortung der Betreiber der Plattform. Die \glqq Orientierungshilfe zum Gesundheitsdatenschutz\grqq{} des Bundesministeriums für Wirtschaft und Energie nennt dabei folgende Kriterien:

\glqq Die Schwere eines möglichen Schadens beurteilt sich nach dem Gewicht des bedrohten Rechts bzw. der bedrohten Freiheit sowie danach, welche Schäden ihnen aus der Verarbeitung erwachsen können. Dabei sind sowohl materielle als auch immaterielle Schäden von Bedeutung. Je sensibler die Daten sind, desto größer ist die mögliche Schadenshöhe. Bei Gesundheitsdaten ist grundsätzlich von einer besonderen Schadenshöhe auszugehen.\grqq{}\footnote{\cite[S. 41]{Bundesaerztekammer2020}}

\glqq Die zudem zu berücksichtigende Eintrittswahrscheinlichkeit meint den statistischen Erwartungswert, mit dem ein bestimmtes Schadensereignis eintreten wird.\grqq{}\footnote{\cite[S. 41]{Bundesaerztekammer2020}}

\glqq Art der Verarbeitung: Artikel 4 Nr. 2 DSGVO nennt insbesondere das Erheben, das Erfassen, die Übermittlung, das Ordnen, die Speicherung, das Löschen und die Vernichtung.\grqq{}\footnote{\cite[S. 41]{Bundesaerztekammer2020}}

\glqq Umfang der Verarbeitung: Menge der Personen, deren Daten in die Verarbeitung einfließen, sowie Menge der Daten, die dabei über eine Person erhoben werden. Dabei ist zu berücksichtigen, dass Daten umso engmaschiger miteinander verknüpft werden können, je größer die Datenmenge ist. Der Aussagegehalt einer Datenanalyse steigt zudem mit der wachsenden Anzahl von Personen, die in eine vergleichende Betrachtung einbezogen werden.\grqq{}\footnote{\cite[S. 41]{Bundesaerztekammer2020}}

\glqq Umstände der Verarbeitung: Gemeint sind alle tatsächlichen und rechtlichen Gegebenheiten, die die Einzelheiten des Verarbeitungsprozesses bestimmen.\grqq{}\footnote{\cite[S. 42]{Bundesaerztekammer2020}}

\glqq Zwecke der Verarbeitung: Hiermit sind die Ziele gemeint, die mit der Verarbeitung verfolgt werden. Diese legt der Verantwortliche selbst fest.\grqq{}
\footnote{\cite[S. 42]{Bundesaerztekammer2020}}

Das angemessene Schutzniveau wird neben dem Risiko für die Rechte des Betroffenen zudem anhand der wirtschaftlichen Interessen des Unternehmens, wie zum Beispiel den Implementierungskosten, festgelegt. Unter den Implementierungskosten versteht man die wirtschaftlichen Ressourcen, welche aufgebracht werden müssen, um die erörterten Anforderungen an den Dienst zu integrieren.\footnote{Vgl. \cite[S. 42 f.]{Bundesaerztekammer2020}} Somit sollen die Implementierungskosten im Rahmen der zumutbaren Aufwände für Sicherheitsmaßnahmen bleiben. 

Über das entsprechende Schutzniveau soll also sichergestellt werden, dass die Maßnahmen zum Schutz der Gesundheitsdaten dem gegebenen Stand der Technik und den bestehenden Marktstandards entspricht.\footnote{Vgl. \cite[S. 740]{Kuhnl2018}}

Die anhand des vorher bestimmten Schutzniveaus festgelegten Anforderungen gilt es somit technisch und organisatorisch umzusetzen und somit einen angemessenen Schutz der Gesundheitsdaten zu gewährleisten. Beispiele für die angemessene Umsetzung technischer Maßnahmen wären unter anderem die Zugriffs- und Weitergabekontrolle der Daten, sowie deren Verschlüsselung oder eine Passwortsicherung. Unter technischen Maßnahmen werden aber auch bauliche Entscheidungen aufgefasst, welche vor dem Zugriff unbefugter Personen schützen.\footnote{Vgl. \cite[S. 41 ff.]{Bundesaerztekammer2020}}

Unter organisatorischen Maßnahmen verstehen sich vor allem die Rahmenbedingungen unter welchen die technischen Verarbeitungsprozesse ablaufen. Beispiele hierfür wären unter anderem das Vieraugenprinzip, Schulungen für die Betreiber des Dienstes, die Verpflichtung zu Mitarbeitererklärungen oder die Protokollierungen von Zugriffen. Es gilt dabei die Maßnahmen regelmäßig zu überprüfen und wenn notwendig zu aktualisieren. Das Unternehmen hat dabei nachzuweisen, dass die Sicherheit der Verarbeitung gewährleistet ist.\footnote{Vgl. \cite[S. 42 ff.]{Bundesaerztekammer2020}}

Um die technischen Anforderungen für die Konzeption des Messengerdienstes bestimmen zu können ist es also notwendig, das angemessenen Schutzniveau für einen solchen Dienst zu bestimmen. Für diese Arbeit werden die, von der Gesellschaft für Datenschutz im Dokument \glqq Austausch von Gesundheitsdaten -  Datenschutzrechtliche Anforderungen an Datenaustauschplattformen im Gesundheitswesen\grqq{} herausgegebenen Empfehlungen für Anforderungen an eine Plattform, welche dem Schutzniveau des Messengerdienstes entspricht, übernommen. Diese werden in Kapitel 2.3 weiter besprochen und ihre praktische Umsetzbarkeit für die Konzeption des Dienstes diskutiert. 


\subsection{Anonymisierung der Daten}\label{subsection:add}
Durch das anonymisieren von personenbezogenen Daten entfallen die Vorgaben der DSGVO zu personenbezogenen Daten, da auf diese Weise keine Rückverfolgung der Daten zur Person, welcher diese ursprünglich zugeordnet waren, mehr möglich ist. Bei Gesundheitsdaten ist jedoch zu beachten, dass wirklich eine vollständige Anonymisierung vorliegt, da diese Art von Daten oft eine Vielzahl von Anhaltspunkten über die zugehörige Person enthalten.\footnote{Vgl. \cite[S. 429 ff.]{Voigt2018}}
Hierzu zählen auch Informationen, welche von Dritten eingeholt werden können. Es sollte somit darauf geachtet werden, dass es durch die Anonymisierung nicht möglich ist, mit vertretbarem Aufwand die zu verarbeitenden Daten auf die zugehörige Person zurückzuführen, denn nur dann entfallen die Vorgaben der DSGVO.\footnote{Vgl. \cite[S. 5 f.]{Bundesaerztekammer2020}}

Aufgrund dieser Bestimmung, sollten, um die Patientendaten zu schützen und das Risiko einer Veruntreuung der Daten minimal zu halten, nur dann nicht anonymisierte Patientendaten über den Messanger geteilt werden, wenn dies unbedingt notwendig ist. Ein Beispiel hierfür wäre, wenn medizinisches Personal Patientendaten über den Dienst austauscht und diskutiert und ein Bezug zwischen Person und Daten unbedingt notwenig ist für eine Beurteilung.

Bei der Umsetzung des Messengerdienstes sollten somit Funktionen implementiert werden, welche es den Nutzern der Plattform erlauben, Daten auf eine einfache Weise zu anonymisieren, denn falls der Personenbezug nicht relevant beim Teilen der Gesundheitsdaten über den Dienst ist und eine Anonymisierung möglich ist, sollte diese auch von den Nutzern durchgeführt werden. Auf diese Weise lässt sich das Risiko der Verletzung der DSGVO und der Schutz der personenbezogenen Daten am besten gewährleisten.

\subsection{Rechtfertigungsgründe für die Verarbeitung von Gesundheitsdaten}\label{subsection:rfdvvg}
Eine zentrale Grundlage des Datenschutzrechtes besteht darin, dass personenbezogene Daten nur dann übermittelt, gespeichert oder anderweitig verarbeitet werden dürfen, wenn es hierfür eine gesetzliche Grundlage gibt. Mögliche Rechtsgrundlagen werden in Artikel 6 der DSGVO aufgelistet.\footnote{Vgl. \cite[S. 5 f.]{Bundesaerztekammer2020}} Gesundheitsdaten fallen allerdings in eine besondere Kategorie der personenbezogenen Daten und gelten als besonders schützenswert. Aus diesem Grund gelten für Gesundheitsdaten neben den allgemeinen datenschutzrechtlichen, zusätzliche, weitaus strengere Regelungen. Diese besonderen Regelungen beinhalten ein allgemeines Verbot zur Verarbeitung von Gesundheitsdaten, solange kein gesetzlich geregelter Ausnahmetatbestand vorliegt, aufgelistet in Artikel 9 Abs. 2–4 der DSGVO.\footnote{Vgl. \cite[S. 20 ff.]{Bundesaerztekammer2020}}

\glqq a) Die betroffene Person hat in die Verarbeitung der genannten personenbezogenen Daten für einen oder mehrere festgelegte Zwecke ausdrücklich eingewilligt, es sei denn, nach Unionsrecht oder dem Recht der Mitgliedstaaten kann das Verbot nach Absatz 1 durch die Einwilligung der betroffenen Person nicht aufgehoben werden,(...)\grqq{}\footnote{Art. 9 – EU-DSGVO 2a}

Um das Verarbeiten von Gesundheitsdaten über den Messengerdienst zu ermöglichen, muss dementsprechend geprüft werden, ob eine Einwilligung des Patienten zum Austausch der Daten innerhalb des Krankenhauses vorliegt. Wenn dies nicht der Fall ist, müsste eine solche Klausel in den Patienten AGBs des Krankenhauses ergänzt werden.\footnote{Vgl. Experteninterview Georg Woditsch} Dieser Teil des Datenschutzgesetzes hat zwar keinen direkten Einfluss auf die technische Umsetzung des Messengerdienstes, wurde an dieser Stelle aber dennoch behandelt, da es sich bei den Maßnahmen zur Wahrung der Nutzerrechte und des Einverständnis des Nutzers zur Verarbeitung der eigenen Daten, um einen zentralen Pfeiler für die Datenverarbeitung nach DSGVO handelt.

\section{Anforderungen an die Plattform auf Basis der bestehenden Gesetzeslage}\label{section:aadpabsbg}
Nachdem im vorherigen Kaptitel bereits die für die Verarbeitung von Gesundheitsdaten relevanten Abschnitte der DSGVO herausgearbeitet wurden, werden diese allgemein gehaltenen Gesetzestexte der DSGVO im folgenden Kapitel in konkrete Anforderungen an den Messengerdienst überführt.  Das Ziel dieses Kapitels ist somit die Fertigstellung eines Anforderungskatalogs, der die gesetzlich relevanten Anforderungen eines, der Zielsetzung dieser Arbeit entsprechenden Messengerdienstes, beinhaltet.

Der zu entwicklende Messengerdienst fällt in die Kategorie einer Datenaustauschplattform für Gesundheitsdaten und dabei in die Unterkategorie einer E-Collaborationsplattform.\footnote{Vgl. \cite[S. 14 ff.]{Bundesverband-Gesundheits-IT-e.V.2016}} Die Kategorisierung gibt dabei das in Kapitel 2.2.2 bereits aufgeführte Schutzniveau des Dienstes vor und dient als Orientierung für die Auswertung der DSGVO Richtlinien. Somit ist es für die Entwicklung des Dienstes notwendig, die bereits herausgearbeiteten DSGVO Richtlinien auf ihre Implikationen für eine E-Collaborationsplattform wie den Messenger auszuwerten und die sich daraus ergebenden Anforderungen aufzustellen. 

Die Basis für diese Auswertung bildet dabei das in zusammenarbeit von Bundesverband Gesundheits-IT e. V., der Arbeitsgruppe Datenschutz, der deutschen Gesellschaft für Medizinische Informatik, Biometrie und Epidemiologie e. V., der Arbeitsgruppe Datenschutz und IT-Sicherheit im Gesundheitswesen, der IHE Deutschland e.V., der Gesellschaft für Datenschutz und Datensicherheit e. V. und dem Arbeitskreis Datenschutz und Datensicherheit im Gesundheits- und Sozialwesen entwickelte Dokument \glqq Austausch von Gesundheitsdaten - Datenschutzrechtliche Anforderungen an Datenaustauschplattformen im Gesundheitswesen\grqq. Innerhalb dieses Dokumentes werden 99 allgemein gültige, gesetzliche Anforderungen an eine Datenaustauschplattform im Gesundheitswesen gelistet, welche aus der bestehenden Gesetzeslage durch die DSGVO abstrahiert wurden.\footnote{Vgl. \cite[S. 19 ff.]{Bundesverband-Gesundheits-IT-e.V.2016}}

Für diese Arbeit wurden die 99 allgemeinen Anforderungen danach sortiert, in wieweit sie für die Entwicklung eines, der Zielsetzung dieser Arbeit entsprechenden Messengerdienst, relevant sind. Die Anforderungen wurden dabei in die Kategorien relevant, teilweise relevant und nicht relevant eingeordnet. Die Einteilung entspricht dabei einem ordinal geordneten Kategoriensystem, wie es in der Forschungsmethodik nach Mayring für eine strukturierende Inhaltsanalyse vorgeschlagen wird. (S 548) 

Desweiteren wurden alle Anforderungen, welche in die Kategorie relevant oder teilweise relevant fielen im Zusammenhang mit den zugrundeliegenden DSGVO Gesetzten (Kapitel 2.2) analysiert und anhand der allgemein gefassten Anforderung, deren jeweils spezifische Bedeutung für die Umsetzung des Messengers, für die noch folgende Konzeption erläutert. 
Das Ergebnis dieser Analyse sind 37 gesetzlich relevante und teilweise relevante Anforderungen speziel an den Messengerdienst, die technisch umgesetzt werden müssen, um einen gesetzeskonformen Betrieb in Krankenhäusern zu erlauben. Dieser angepasste und jeweils um die Bedeutung für die Umsetzung des Messengers ergänzte Anforderungskatalog kann im Anhang eingesehen werden.

\section{Anforderungen an den Funktionsumfang}\label{section:pdimsbd}
Neben den gesetzlichen Anforderungen an den Messengerdienst muss auch der eigentliche Funktionsumfang definiert werden. Das für diese Arbeit geführte Experteninterview und die Fachliteratur haben ergeben, dass es eine Vielzahl an möglichen Funktionen gibt, welche den Funktionsumfang von herkömmlichen Messengern weit überschreiten, aber speziell für den Einsatz in Krankenhäusern einen Mehrwert bieten können.\footnote{Vgl. Experteninterview Georg Woditsch}\footnote{Vgl. \cite[S. 19 4.]{Khan2020}}
Da der Fokus dieser Arbeit allerdings auf der Umsetzung der gesetzlichen Anforderungen liegt, werden im folgenden nur Funktionen gesammelt, welche bei einem allgemein auf den professionellen Gebrauch ausgelegten Messengerdienst zu erwarten sind und für den Einsatz in Krankenhäusern hilfreich sind. Zudem soll aber auch bei der noch folgenden Konzeption des Messengers sichergestellt werden, das funktionale Erweiterungen dennoch möglich sind.

Um eine repräsentative Auflistung der benötigten Funktionen eines Messengers für den professionellen Gebrauch zu erhalten, listet die folgende Tabelle die Kernfunktionen der beiden populärsten Messengerdienste für den professionellen Einsatz, Microsoft Teams und Slack auf.
Mit 115 Millionen täglich aktiven Nutzern (Teams) und 44 Millionen täglich aktiven Nutzer (Slack) handelt es sich bei den beiden Diensten um die klaren Marktführer im Segment der Kommunikationsplattformen für Unternehmen. Im Folgenden ein kurzer Steckbrief der jeweiligen Dienste:

Slack: Slack ist eine proprietäre Geschäftskommunikationsplattform, die vom amerikanischen Softwareunternehmen Slack Technologies entwickelt wurde. Slack bietet viele Funktionen im IRC-Stil, einschließlich Chatrooms, die nach Themen, privaten Gruppen und Direktnachrichten organisiert sind. Zudem unterstützt auch Slack Video und Audio Telefonate, bietet aber gegenüber Diensten wie WhatsApp eine Reihe von Funktionen, welche speziell auf Enterprise Kunden ausgelegt sind. Hierzu zählen unter anderem eine Reihe von administrativen Funktionen, welche Unternehmen eine zentrale Verwaltung des Dienstes ermöglicht, oder die Option bestehende Nutzerverwaltungen an die Software anzubinden.\footnote{\cite{Slack2020}}

Microsoft Teams: Microsoft Teams ist eine proprietäre Geschäftskommunikationsplattform, die von Microsoft als Teil der Microsoft 365-Produktfamilie entwickelt wurde. Teams bietet ebenfalls Möglichkeiten für Chat- und Videokonferenzen und das Teilen von Dateien. Der Hauptunterschied zwischen Microsofts Lösung und Slack liegt in der Präsentation. Somit unterscheiden sich zwar UI und UX, die Kernfunktionen einer modernen Kommunikationsplattform decken aber beide Dienste gleichermaßen ab.\footnote{\cite{Microsoft2020}}

Die folgende Tabelle listet die Kernfunktionen eines professioneller Kommunikationsdienstes auf. Die Liste wurde beispielhaft auf Basis der von Slack und Teams für den jeweiligen Dienst beworbenen Funktionen erstellt, ergänzt mit den im Experteninterview genannten Anforderungen an den Dienst. Diese Liste soll dabei als Anforderungsminimum für einen entsprechenden Dienst verstanden werden, welche zusammen mit den gesetzliche Anforderungen umgesetzt werden muss.

\begin{itemize}
    \item Einzel und Gruppen Chats\footnote{\cite{Slack2020}}\footnote{\cite{Microsoft2020}}\footnote{Vgl. Experteninterview Georg Woditsch}
    \item Themenspezifische Chaträume\footnote{\cite{Slack2020}}\footnote{Vgl. Experteninterview Georg Woditsch}
    \item Administrative Funktionen zur Vergabe von Rechten innerhalb des Dienstes\footnote{\cite{Slack2020}}\footnote{\cite{Microsoft2020}}\footnote{Vgl. Experteninterview Georg Woditsch}
    \item Video und Audio telefonie\footnote{\cite{Slack2020}}\footnote{\cite{Microsoft2020}}\footnote{Vgl. Experteninterview Georg Woditsch}
    \item Anbindung an ein externes Personalregister zur Anmeldung\footnote{\cite{Slack2020}}\footnote{\cite{Microsoft2020}}\footnote{Vgl. Experteninterview Georg Woditsch}
    \item Zugriff auf den Dienst per App, Desktop und Web Oberfläche\footnote{\cite{Slack2020}}\footnote{\cite{Microsoft2020}}\footnote{Vgl. Experteninterview Georg Woditsch}
    \item Durchsuchbares Personenregister\footnote{\cite{Slack2020}}\footnote{\cite{Microsoft2020}}
\end{itemize}