\chapter{Schlussbetrachtung}\label{chapter:schlussbetrachtung}
Zum Abschluss dieser Arbeit werden im folgenden Kapitel die Ergebnisse dieser Arbeit noch einmal zusammengefasst, kritisch reflektiert und ein Ausblick darauf geben, welchen Einfluss die in dieser Arbeit gewonnen Erkenntnissen auf zukünftige Forschung in diesem Themenbereich haben können.

\section{Kritische Reflektion}\label{chapter:kr}
Um ein umfassendes Fazit aus dieser Arbeit ziehen zu können, ist es notwendig sich kritisch mit den gewonnenen Erkenntnissen der Forschung auseinander zu setzen. Die Arbeit hat ergeben, dass sich ein datenschutzgerechter Messenger Dienst für organisatorische und Gesundheitsdaten betreffende Kommunikation theoretische technisch Umsetzen lässt. Die Forschung, zusammen mit dem geführten Experteninterview haben allerdings gezeigt, dass es sich bei der technischen Umsetzung der rechtlichen Anforderungen nur um einen Bestandteil des großen Problemkomplexes handelt, welcher zu lösen ist bevor ein solcher Messenger Dienst Umsetzbar ist und dem Krankenhauspersonal einen realen Mehrwert bieten kann.  Hierzu zählen eine Vielzahl von organisatorischen Prozessen die erarbeitet und implementiert werden müssen damit der Messenger Datenschutzkonform betrieben werden kann. Beispiele hierfür sind die Art und Weise wie und mit wem das Personal Daten über den Messenger teilen darf, die Umsetzung von Off- und Onboarding Prozessen oder wie vergabe der Rolle von Gruppen-Administratoren innerhalb des Messengers gehandhabt werden soll. Ebenso müssen eine vielzahl an dokumentarischen Anforderungen erfüllt sein, wie die Anfertigung von Datenschutzvereinbarungen für die Nutzer des Messengers.
Für den Abschluss der Forschung zur technischen Umsetzbarkeit der rechtlichen Grundlage sollte zudem noch in einer weiterführenden Forschung auf Basis des in dieser Arbeit erstellten Konzepts ein Prototyp erstellt werden und von dritten juristisch auf seine DSGVO-Konformität geprüft werden, dies gilt auch für das gültige Länderrecht.

Da es zum Zeitpunkt der Fertigstellung dieser Arbeit keinen Messenger Dinest gibt, der den aufgestellten Anforderungen entspricht und von einem Krankenhaus verwendet wird, bietet das Thema Krankenhaus Messenger Dienst noch eine vielzahl offener Problemstellungen die gelöst werden müssen, bevor eine Implementierung eines solchen Dienstes in den realen Krankenhausbetrieb möglich ist.
Einen großer, offenen Punkt stellt dabei der Funktionsumfang dar. Während in dieser Arbeit ein minimaler Funktionumfang als Basis für die Erstellung des Konzeptes verwendet wurde, sollte dieser Punkt vor einer realen Implementierung noch weiter ausgearbeitet werden. Eine vollwertige Ausarbeitung dieses Themenpunktes sollte in enger Zusammenarbeit mit dem Krankenhauspersonal geschehen. Nur auf diese Wiese kann sicher gestellt werden dass der entwickelte Messenger einen mehrwert für das Krankenhaus bietet und vom Personal angenommen wird.

Aus technischer Sicht müssen ebenfalls eine Reihe von offenen Punkten besprochen werden, beovor der Messenger fertig gestellt werden kann.
Dazu zählt vorallem wie der Messenger innerhalb des Krankenhauses betrieben werden soll. Dabei ist zu klären wie eine fortlaufende Weiterentwicklung des Dienstes ermöglicht werden kann. Gerade wegen der sensiblen Daten die über den DIenst verarbeitet werden muss es, falls Sicherheitslücken auftreten, möglich sein diese schnell zu beheben. Auch die Frage, wie der Messenger auf den Servern des Krankenhauses betrieben werden kann, muss noch diskutiert werden, da sich in diesem Punkt die Anforderungen einzelner Krankenhäuser an den Messenger unterscheiden können.

Zusammenfassend lässt sich festhalten, dass auch wenn innerhalb dieser Arbeit das iniziale Forschungsziel erreicht wurde, noch ein erheblicher Forschungsaufwand betrieben werden muss, damit ein Messenger Dienst speziel für den Gebrauch in Krankenhäusern realisiert werden kann. Hiermit kann auch begründet werden, warum es zum jetzigen Zeitpunkt keinen entsprechenden Dienst gibt, trotz Forderungen von Bundesregierung und Ärzten. Die Komplexität der Umsetzung eines Messenger Dienstes speziel für den Einsatz in Krankenhäusern ist schlichtweg erheblich.

\section{Fazit}\label{chapter:fazit}
Zusammenfassend lässt sich festhalten, dass trotz einer einstimmigen Forderung von Bundesregierung, Ärzten und Krankenhaus IT es zum Zeitpunkt der Fertigstellung dieser Arbeit keinen Messenger Dienst gibt, welcher sowohl für die organisatorische, als auch Gesundheitsdaten betreffende, interne Kommunikation genutzt werden kann. Die analyse der geltenden Gesetze hat ergeben, dass für die Umsetzung eines solchen Dienstes, dem Gesetzt entsprechend enorm hohe Sicherheitsstandards gewahrt werden müssen, da der Dienst nicht nur die Daten von seinen Nutzern, dem Krankenhauspersonal verarbeitet, sondern auch die Gesundheitsdaten der Patienten. Die Wahrung des Datenschutz, vorallem der der Patienten muss dementsprechend höchste priorität haben. 

Um die Umsetzung eines solchen Dienstes zu realisieren, welcher eine ausreichend Funktionalen Umfang bietet und gleichzeitig alle geltenden gesetzlichen Anforderungen erfüllt wurde deshalb für diese Arbeit ein Anforderungskatalog, bestehend aus 37 rechtlichen und 7 funktionallen Anforderungen erstellt, welcher das Minimum an technisch umzusetzenden Anforderungen vorgibt, welches Erfüllt sein muss, damit der Messenger betrieben werden kann. Um die gesetzlichen Anforderungen zu erstellen wurden die die Umsetzung des DIenstes betreffenden Gesetzestexte der DSGVO herausgearbeitet und erläutert. Die Einhaltung dieser für der Umsetzung des Messengers bildet das Fundament des Anforderungskatalogs. Um auf den Messenger anwendbare Anforderungen zu erhalten wurde mit den Erkenntnissen aus den herausgearbeiteten Gesetzestexten und einem von der Bundesregierung herausgegebenen Anforderungskatalog zur Umsetzung von Datenaustauschplattformen im Gesundheitswesen, der Anforderungskatalog für die Umsetzung eines Messengerdienstes im Gesundheitswesen entwickelt.

Dieser Anforderungskatalog dient als Leitfaden für das in dieser Arbeit erstelle Konzept zur Umsetzung des Messenger Dienstes auf Basis des Matrix Frameworks. Im Falle, dass ein der Zielsetzung dieser Arbeit entsprechender Messenger entwickelt wird, der einen anderen Ansatz zur Umsetzung der Anforderungen wählt als das in dieser Arbeit beschriebene Konzept, kann der Anforderungskatalog dennoch als Einstiegspunkt für die Umsetzung des Dienstes genutzt werden und somit dessen Entwicklung unterstützen.

Um aufzuzeigen, das eine Umsetzung der aufgestellten Anforderungen theoretisch möglich ist, wurde zudem ein Konzept für eine beispielhafte Umsetzung auf Basis des Matrix Frameworks erstellt. Mit Hilfe des robusten und auf Sicherheit fokussierten Matrix Frameworks können alle im Katalog festgehaltenen Anforderungen an den Messenger realisiert werden. 

Die Arbeit zeigt auf, dass die Umsetzung eines solchen Messengers technisch Möglich ist, ohne bestehende Gesetze zu verletzen und die Daten von Krankenhauspersonal und Patienten angemessen zu schützen.
Damit ist ein erster Schritt zur realisierung des Messengers getan. Wie in der Einleitung dieser Arbeit bereits erläutert hat ein moderner Messanger Dienst, welcher es erlaubt Gesundheitsdaten auszutauschen, das Potential einen echten Mehrwert für Krankenhäuser zu beiten, der schnelle Austausch über patientenbezogene Daten, zum Beispiel zwischen Ärzten aktuell in digitaler Form nicht möglich ist. 


\section{Ausblick}\label{chapter:fazit}
Zum Abschluss soll noch einmal ein Ausblick gegeben werden, welche Implikationen die Erkenntnisse dieser Arbeit auf zukünftige Forschung und Entwicklung im Bereich der Kommunikationssysteme für den Gesundheitssektor haben kann. Wie bereits beschrieben soll diese Arbeit als Grundlage für die Entwicklung eines entsprechenden Krankenhaus internen Messangers diesen. Sowohl das Konzept zur Umsetzung der Anforderungskatalog als auch der Anforderungskatalog selbst kann dabei als Einstiegspunkt für die realisierung eines solchen Dienstes genutzt werden. Mit den sich häufenden Forderungen nach einem entsprechenden Kommunikationsdienst und der fortschreitenden digitalisierung im Gesundheitssystem ist es nicht unwahrscheinlich, dass ein ein solcher, besonderer Messanger Dienst Teil der Krankenhaus internen Kommunikation wird. 

An dieser Stelle soll auch besonderer Fokus auf die gewonnen Erkenntnisse bei der Umsetzung des Messangers mit dem Matrix Framework gelegt werden. Eine besondere Eigenschaft es Matrix Frameworks welche im Kontext der Zielsetzung dieser Arbeit nicht relevant ist, aber einen großen Einfluss bei der Implementierung eines entsprechenden Dienstes in Krankenhäusern haben könnte, ist der dezentralisierte Aufbau. Wenn Matrix die Basis des Messangers für mehrere Krankenhäuser biete wäre es dadurch Möglich, dass sich Personal über den Dienst Krankenhausübergreifend austauschen kann. Gleichzeitig wäre jedes Krankenhaus weiter in der Lage die eigene Matrix Instanz auf den internen Servern zu betreiben und somit eine Hoheit über die eigenen Daten beizubehalten, da nur die Daten der Instanzübergreifenden Chats ausgetauscht würden. Auf diese Weise könnte ein Netzwerk aus Matrix instanzen Aufgebaut werden, welches es Gesundheitsperson erlaubt, sich Klinikübergreifend auszutauschen. Noch weiter Gedacht könnte dieses Netzwerk sogar auf Arztpraxen und andere medizinische Einrichtungen ausgedehnt werden. Ein solches System könnte einen großen Einfluss auf den Informations- und Wissensaustausch im gesundheitssektor haben. Hierbei ist es wichtig anzumerken, dass solches, dezentralisierten Messanger Netzwerk nur als Beispiel und Denkanstoß ZU verstehen ist, der demonstrieren soll, was mit einem krankenhaus Messanger auf Basis des Matrix Frameworks theoretisch technisch Möglich wäre. 

Die Konzeption des Krankenhausinternen Messangers hat bereits gezeigt, dass noch viele Hürden genommen werden müssen, bevor ein solches System realisiert werden kann. Aus diesem Grund soll mit dieser Arbeit ein erster Schritt für die Umsetzung eines Krankenhausinternen Messangers getan werden, mit dem Ziel die digitale Kommunikation zwischen medizinischem personal zu vereinfachen und zu verbessern. 

