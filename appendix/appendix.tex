\chapter*{Anhang}
\addcontentsline{toc}{chapter}{Anhang}

\lstset{language=TeX,
    morekeywords={anhang, anhangteil}
}

% Definition des Anhangverzeichnis
\section*{Anhangverzeichnis}
\vspace{-8em}
\abstaendeanhangverzeichnis
\listofanhang
\clearpage

% Konfiguration der speziellen Kopfzeile für den Anhang
\spezialkopfzeile{Anhang}

% Hauptteil des Anhangs
% Mit \anhang{Abschnitt des Anhangs} fügt man ein Kapitel in dem Anhang hinzu z. B. Interview Transkripte
\anhang{Gesetzlicher Anforderungskatalog an einen Messengerdienst für den Austausch von Gesundheitsdaten}

\begin{longtable}{p{0.6cm}|p{6cm}|p{2cm}|p{6cm}}
    \hline
    Nr. & Anforderung                                                                                                                                                                                                                                                                                          & Relevants          & Einordnung für den Messenger                                                                                                                                                                                                                                                                                                                                                                                                                                                                                                                                        \\ \hline
    1      & „Jegliche Erhebung, Verarbeitung und Nutzung personenbezogener oder personenbeziehbarer Daten - insbesondere durch den Einsatz von Datenaustauschplattformen, die an das Internet angebunden sind - bedarf einer datenschutzrechtlich wirksamen Einwilligung des Betroffenen.“                       & relevant           & Anforderung 1 und 3 sind relevant, da sie für den Messenger bedeuten, dass dem Krankenhauspersonal die Möglichkeit gegeben werden muss, der Verarbeitung ihrer Nutzerdaten durch die Plattform zuzustimmen. Das gleiche gilt auch für die Patienten, deren Einwilligung über die für das Krankenhaus gültigen AGBs eingeholt werden muss, welche auch die allgemeine Verarbeitung der Patientendaten inerhalb des Krankenhauses beschreiben.                                                                                                                        \\ \hline
    3      & „Eine Einwilligung muss für den Betroffenen jederzeit mit Wirkung für die Zukunft widerrufbar sein.“                                                                                                                                                                                                 & relevant           & Anforderung 1 und 3 sind relevant, da sie für den Messenger bedeuten, dass dem Krankenhauspersonal die Möglichkeit gegeben werden muss, der Verarbeitung ihrer Nutzerdaten durch die Plattform zuzustimmen. Das gleiche gilt auch für die Patienten, deren Einwilligung über die für das Krankenhaus gültigen AGBs eingeholt werden muss, welche auch die allgemeine Verarbeitung der Patientendaten innerhalb des Krankenhauses beschreiben.                                                                                                                        \\ \hline
    4      & „Jede Datenaustauschplattform muss Datenschutzhinweise veröffentlichen und darin die getroffenen Datensicherheitslösungen allgemeinverständlich beschreiben.“                                                                                                                                        & relevant           & Der Messenger muss die Möglichkeit bieten, die Datenschutzhinweise Anforderung 4-8 entsprechend den Nutzern zur Verfügung zu stellen. Die eigentliche Formulierung der Datenschutzhinweise ist nicht Teil dieser Arbeit, die technische Umsetzung, um diese auf die geforderte Weise zur Verfügung zu stellen jedoch schon.                                                                                                                                                                                                                                         \\ \hline
    5      & „Datenschutzhinweise müssen unmittelbar von der Startseite der Datenaustauschplattform aus aufrufbar bzw. erreichbar sein.“                                                                                                                                                                          & relevant           &                                                                                                                                                                                                                                                                                                                                                                                                                                                                                                                                                                     \\ \hline
    6      & „Datenschutzhinweise müssen für den Betroffenen jederzeit abrufbar sein.“                                                                                                                                                                                                                            & relevant           &                                                                                                                                                                                                                                                                                                                                                                                                                                                                                                                                                                     \\ \hline
    7      & „Bei nachträglicher Änderung ist der Betroffene zu informieren und sein Einverständnis erneut einzuholen.“                                                                                                                                                                                           & relevant           &                                                                                                                                                                                                                                                                                                                                                                                                                                                                                                                                                                     \\ \hline
    8      & „Wird die Einwilligung des Betroffenen elektronisch eingeholt, so muss der Vorgang protokolliert werden und der Inhalt der Einwilligung für den Betroffenen jederzeit abrufbar sein.“                                                                                                                & relevant           &                                                                                                                                                                                                                                                                                                                                                                                                                                                                                                                                                                     \\ \hline
    9      & „Personenbezogene oder personenbeziehbare Daten müssen pseudonymisiert werden, soweit dies nach dem Verwendungszweck möglich ist und keinen im Verhältnis zu dem angestrebten Schutzzweck unverhältnismäßigen Aufwand erfordert.“                                                                    & relevant           & Der Messengerdienst sollte den Nutzern aufgrund dieser Anforderung funktionen zur Verfügung stellen, Daten wie zum Beispiel Bilddateien, die über den Dienst geteilt werden sollen auf eine einfache Art und Weise zu pseudonymisieren.                                                                                                                                                                                                                                                                                                                            \\ \hline
    13     & „Der Betroffene muss der Nutzung von Cookies explizit zustimmen.“                                                                                                                                                                                                                                    & teilweise relevant & Die gesetzliche Anforderungen 13-15 an den Umgang mit Cookies ist vor allem für die Umsetzung, der im Funktionsumfang festgehaltenen Anforderung, des Webzugangs zum Messengerdienst relevant. Der Messengerdienst muss dementsprechend technisch die Möglichkeit bieten, über mögliche Cookies zu informieren und dem Nutzer die Zustimmung zur Nutzung von Cookies ermöglichen.                                                                                                                                                                                    \\ \hline
    14     & „Die Verwendung von Cookies durch Drittanbieter muss auf die Erstellung anonymer Nutzungsstatistiken beschränkt sein.“                                                                                                                                                                               & teilweise relevant &                                                                                                                                                                                                                                                                                                                                                                                                                                                                                                                                                                     \\ \hline
    15     & „Das Handling der Drittanbieter-Cookies muss durch den Betroffenen steuer- und nachvollziehbar sein.“                                                                                                                                                                                                & teilweise relevant &                                                                                                                                                                                                                                                                                                                                                                                                                                                                                                                                                                     \\ \hline
    18     & „Die Erhebung, Verarbeitung und Nutzung von Daten zur Erstellung von Nutzungsstatistiken von Webportalen muss dokumentiert werden.“                                                                                                                                                                  & teilweise relevant & Aus Gründen der Privatsphäre ist von einem individuellen Tracken des Nutzerverhaltens auf der Plattform ab zu sehen, da es sich hierbei nicht um eine relevante Funktion für den Einsatz des Messengerdienstes handelt. Bei der nicht personenbezogenen Auswertungen des Nutzerverhaltens, zum Beispiel zur Überprüfung der Annahme des Dienstes seitens der Nutzer, sind Anforderungen 18-20 zu beachten. Allgemein fällt die Analyse nicht in die Zielsetzung der Arbeit und wird deshalb im Folgenden nicht weiter besprochen.                                  \\ \hline
    19     & „Die Erstellung von Nutzungsstatistiken von Datenaustauschplattformen muss auf Basis anonymisierter Daten erfolgen.“                                                                                                                                                                                 & teilweise relevant &                                                                                                                                                                                                                                                                                                                                                                                                                                                                                                                                                                     \\ \hline
    20     & „Eine Analyse des Nutzerverhaltens (einschließlich Geolokalisierung) auf der Basis gekürzter IP-Adressen ist ohne Einwilligung des Betroffenen möglich, wenn durch die Kürzung der IP-Adresse ein Personenbezug ausgeschlossen wurde.“                                                               & teilweise relevant &                                                                                                                                                                                                                                                                                                                                                                                                                                                                                                                                                                     \\ \hline
    31     & „Alle Personen, welche auf die gespeicherten Gesundheitsdaten zugreifen können, müssen vor dem erstmaligen Zugriff auf die Wahrung des Datengeheimnisses verpflichtet worden sein.“                                                                                                                  & relevant           & Für den Messengerdienst muss eine technische Lösung implementiert werden, welche es erlaubt, dem Nutzer eine solche Wahrung zu präsentieren.                                                                                                                                                                                                                                                                                                                                                                                                                       \\ \hline
    32     & „Ein anonymer Zugriff des Plattformbetreibers sowie der Nutzer der Plattform auf personenbezogene oder personenbeziehbare Gesundheitsdaten ist zu unterbinden.“                                                                                                                                      & relevant           & Es muss technisch sichergestellt werden, dass sämtliche Zugriffe auf den Dienst zu jeder Zeit einer realen Person zugeordnet werden können. Dies gilt auch für den direkten Zugriff auf die mit dem Dienst verbundenen Datenbanken durch Administratoren.                                                                                                                                                                                                                                                                                                           \\ \hline
    33     & „Daten über den Ablauf des Zugriffs oder der sonstigen Nutzung sind unmittelbar nach dem Zugriff bzw. ggfs. nach erfolgter Abrechnung der Dienstnutzung zu sperren und nur zu entsperren, wenn eine gesetzliche Bestimmung (z.B. Auskunftsersuchen eines Betroffenen) dies erlaubt.“                 & relevant           & Der Messenger muss es erlauben, Daten zeitlich begrenzt zu teilen, damit sichergestellt werden kann, dass  Daten nur für die Dauer ihres benötigten Verwendungszwecks über den Messenger abrufbar sind.                                                                                                                                                                                                                                                                                                                                                             \\ \hline
    34     & „Die Nutzung von Telemedien muss gegen die Kenntnisnahme unberechtigter Dritter geschützt werden.“                                                                                                                                                                                                   & relevant           & Eine Anmeldung und Verifizierung des Nutzers muss vor dem Zugriff auf den Messenger erfolgen. Ebenso muss der Zugriff auf den dem Messenger zugeordnete Software Code und entsprechenden Datenbanken geschützt werden. Dies muss durch eine technische Lösung sichergestellt werden.                                                                                                                                                                                                                                                                                \\ \hline
    39     & „Ereignisse, die potenziell dazu führen können, dass ein unberechtigter Zugriff auf personenbezogene oder personenbeziehbare Daten erfolgen könnte, sind zu protokollieren.“                                                                                                                         & relevant           & Es muss sichergestellt werden, dass mögliche, unberechtigte Zugriffe in den Logs des Dienstes festgehalten werden und entsprechend archiviert werden.                                                                                                                                                                                                                                                                                                                                                                                                               \\ \hline
    41     & „Der Betroffene muss die Möglichkeit haben, jederzeit Einblick in alle zu seiner Person gespeicherten Daten zu erhalten. Dies umfasst auch die Möglichkeit, Änderungen seiner gespeicherten Daten nachzuvollziehen.“                                                                                 & relevant           & Anforderung 41-43 betreffen im Kontext des Messengers nur die Daten der eigentlichen Nutzer, da keine langfristige Speicherung der Gesundheitsdaten über den Messenger vorgesehen ist. Die geforderten Möglichkeiten im Bezug auf Patientendaten müssen bei den Systemen vorgenommen werden, welche zur langfristigen Speicherung der Patientendaten vorgesehen sind. Was die eigentlichen Nutzerdaten betrifft sind technische Lösungen zu implementieren, welche die Umsetzung der beschriebenen Anforderungen erlaubt. \\ \hline
    42     & „Der Betroffene muss die Möglichkeit haben, einen Ausdruck oder einen für ihn verwertbaren Export aller zu seiner Person gespeicherten Daten zu erhalten.“                                                                                                                                           & relevant           &                                                                                                                                                                                                                                                                                                                                                                                                                                                                                                                                                                     \\ \hline
    43     & „Es muss eine Möglichkeit geben, auf Aufforderung des Betroffenen Daten zu seiner Person zu korrigieren.“                                                                                                                                                                                            & relevant           &                                                                                                                                                                                                                                                                                                                                                                                                                                                                                                                                                                     \\ \hline
    46     & „Es muss eine Löschfunktion implementiert werden, welche eine Rekonstruktion gelöschter Informationen ausschließt.“                                                                                                                                                                                  & relevant           & Eine permanente Löschung betroffener Daten muss über den Messenger technisch möglich sein.                                                                                                                                                                                                                                                                                                                                                                                                                                                                          \\ \hline
    58     & „Da im Internet eine potenziell größere Gefährdung für einen unbefugten Zugriff auf personenbezogene Daten existiert, muss mindestens eine 2-Faktor-Authentifizierung erfolgen.“                                                                                                                     & relevant           & Die Möglichkeit einer 2 Faktor Authentifikation muss in den Messengerdienst implementiert werden.                                                                                                                                                                                                                                                                                                                                                                                                                                                                  \\ \hline
    59     & „Werden statische Passwörter zur Authentisierung eingesetzt, so müssen die Empfehlungen des BSI bzgl. der Generierung und des Umgangs eingehalten werden. D.h. es müssen technische und organisatorische Maßnahmen zu der Einhaltung der Empfehlungen des BSI getroffen werden.“                     & relevant           & Die Mindestanforderungen an die Komplexität, der für die Nutzung des Messengers zu erstellenden Passwörtern, muss durch eine technische Lösung festlegbar sein.                                                                                                                                                                                                                                                                                                                                                                                                       \\ \hline
    60     & „Authentisierungsgeheimnisse dürfen nur gesichert in Netzwerken übertragen werden, d.h. es muss eine verschlüsselte Datenübertragung entsprechend dem Stand der Technik eingesetzt werden.“                                                                                                          & relevant           & Es muss technisch sicher gestellt werden, dass nur über eine gesicherte Netzwerkverbindung mit dem Dienst kommuniziert werden kann.                                                                                                                                                                                                                                                                                                                                                                                                                                 \\ \hline
    62     & „Nach wiederholter fehlerhafter Authentisierung muss der Zugang gesperrt werden.“                                                                                                                                                                                                                    & relevant           & Eine technische Möglichkeit muss implementiert werden, welche es erlaubt, nach zu häufiger, fehlerhafter Authentifikation betroffene Zugänge zu sperren.                                                                                                                                                                                                                                                                                                                                                                                                            \\ \hline
    63     & „Die Sitzung muss gesperrt oder beendet werden, wenn der Anwender eine definierte Zeitspanne in der Sitzung keine Aktivitäten durchführte (Sitzungs-Zeitlimit, Session Timeout).“                                                                                                                    & relevant           & Eine technische Möglichkeit muss implementiert werden, welche es erlaubt, eine Zeitspanne zu definieren, nach welcher eine erneute Authentifizierung vom Nutzer gegenüber des Messengers notwendig ist. Dies betrifft vor allem den Zugriff auf das Webinterface des Messengers.                                                                                                                                                                                                                                                                                    \\ \hline
    64     & „Ein Prozess zur Rücksetzung bzw. Entsperrung von gesperrten Zugangskennungen ist einzurichten, zu beschreiben und anzuwenden.“                                                                                                                                                                      & relevant           & Eine technische Möglichkeit muss implementiert werden, welche die Umsetzung von Anforderung 64 ermöglicht.                                                                                                                                                                                                                                                                                                                                                                                                                                                          \\ \hline
    66     & „Benutzerkennungen, welche über einen definierten Zeitraum nicht benutzt wurden, sind zu sperren bzw. auf inaktiv zu setzen.“                                                                                                                                                                        & relevant           & Eine technische Möglichkeit muss implementiert werden, welche die Umsetzung von Anforderung 66 ermöglicht.                                                                                                                                                                                                                                                                                                                                                                                                                                                          \\ \hline
    71     & „Eine Kombination von Rollen bzw. Zugriffsrechten für eine Person, welche der Person mehr Rechte auf Datenzugriffe erteilt, als für ihre Aufgabe nötig ist, ist zu verhindern. Es sind technische und organisatorische Maßnahmen zu ergreifen, um dies sicherzustellen.“                             & relevant           & Für den Messenger muss ein Anforderung 71 entsprechendes Rollenkonzept technisch umgesetzt werden.                                                                                                                                                                                                                                                                                                                                                                                                                                                                  \\ \hline
    73     & „Protokolldaten sind gegen unbefugten Zugriff in geeigneter Weise entsprechend dem Stand der Technik zu schützen.“                                                                                                                                                                                   & relevant           & Die vom Messenger erstellten Protokolle sind dem Stand der Technik entsprechend zu schützen.                                                                                                                                                                                                                                                                                                                                                                                                                                                                    \\ \hline
    81     & „Die Übertragung personenbezogener oder personenbeziehbarer Gesundheitsdaten zwischen Clients und Servern wie auch zwischen Servern selbst muss entsprechend dem jeweiligen Stand der Technik generell verschlüsselt erfolgen.“                                                                      & relevant           & Eine Verschlüsselung, wie sie in Anforderung 81 definiert, ist muss für den Messenger implantiert werden.                                                                                                                                                                                                                                                                                                                                                                                                                                                             \\ \hline
    86     & „Gesundheitsdaten sind entsprechend dem Stand der Technik verschlüsselt in der Datenbank zu speichern, so dass bei administrativen Zugriffen Wartungspersonal keinen unbefugten Zugriff auf die gespeicherten Daten erhalten kann.“                                                                  & relevant           & Eine Verschlüsselung, wie sie in Anforderung 86 definiert, ist muss für den Messenger implantiert werden.                                                                                                                                                                                                                                                                                                                                                                                                                                                             \\ \hline
    87     & „Nur Personen, die laut Berechtigungskonzept zu einer Eingabe berechtigt sind, dürfen personenbezogene oder personenbeziehbare Daten in eine Datenaustauschplattform eingeben.“                                                                                                                      & relevant           & Für den Messenger muss ein Anforderung 87 und 88 entsprechendes Rollenkonzept technisch umgesetzt werden.                                                                                                                                                                                                                                                                                                                                                                                                                                                           \\ \hline
    88     & „Nur Personen, die laut Berechtigungskonzept zu einem Datenimport berechtigt sind, dürfen einen Import personenbezogener oder personenbeziehbarer Daten in eine Datenaustauschplattform durchführen oder veranlassen.“                                                                               & relevant           &                                                                                                                                                                                                                                                                                                                                                                                                                                                                                                                                                                     \\ \hline
    93     & „Es muss ein Backup-Konzept vorhanden sein, welches gewährleistet, dass die Daten nach einem Vorfall in angemessener Zeit wieder zur Verfügung gestellt werden können. In diesem Backup-Konzept muss berücksichtigt werden, dass nur berechtigte Personen Zugriff auf Backup-Daten erlangen können.“ & teilweise relevant & Die technische Möglichkeit der Erstellung regelmäßiger Backups muss gewährleistet sein.                                                                                                                                                                                                                                                                                                                                                                                                                                                                             \\ \hline
\caption{Gesetzlicher Anforderungskatalog an einen Messengerdienst für den Austausch von Gesundheitsdaten}    
\end{longtable}

\anhang{Gesetzlicher Anforderungskatalog an einen Messengerdienst für den Austausch von Gesundheitsdaten, erweitert mit den Ansätzen zur Umsetzung durch das Matrix Framework}

\begin{longtable}{p{0.6cm}|p{4cm}|p{5cm}|p{5cm}}
    \hline
Nr. & Anforderung                                                                                                                                                                                                                                                                                          & Umsetzung in Matrix          & Einordnung für den Messenger                                                                                                                                                                                                                                                                                                                                                                                                                                                                                                                                        \\ \hline
1 &
  „Jegliche Erhebung, Verarbeitung und Nutzung personenbezogener oder personenbeziehbarer Daten - insbesondere durch den Einsatz von Datenaustauschplattformen, die an das Internet angebunden sind - bedarf einer datenschutzrechtlich wirksamen Einwilligung des Betroffenen.“ &
  Matrix erlaubt es technisch Policys anzulegen, denen der User zustimmen muss, bevor er den Dienst nutzen kann. Die rechtlich wirksame Einwilligung des Nutzers kann somit, wie in der Anforderung beschrieben, als Policy angelegt werden. &
  Anforderung 1 und 3 sind relevant, da sie für den Messenger bedeuten, dass dem Krankenhauspersonal die Möglichkeit gegeben werden muss, der Verarbeitung ihrer Nutzerdaten durch die Plattform zuzustimmen. Das gleiche gilt auch für die Patienten, deren Einwilligung über die für das Krankenhaus gültigen AGBs eingeholt werden muss, welche auch die Vorgaben zur allgemeinen Verarbeitung der Patientendaten innerhalb des Krankenhauses regeln." \\
  \\ \hline
  3 &
  „Eine Einwilligung muss für den Betroffenen jederzeit mit Wirkung für die Zukunft widerrufbar sein.“ &
  Matrix erlaubt es technisch Policys anzulegen, denen der User zustimmen muss, bevor er den Dienst nutzen kann. Die rechtlich wirksame Einwilligung des Nutzers kann somit, wie in der Anforderung beschriebenm als Policy angelegt werden. &
  Anforderung 1 und 3 sind relevant, da sie für den Messenger bedeuten, dass dem Krankenhauspersonal die Möglichkeit gegeben werden muss, der Verarbeitung ihrer Nutzerdaten durch die Plattform zuzustimmen. Das gleiche gilt auch für die Patienten, deren Einwilligung über die für das Krankenhaus gültigen AGBs eingeholt werden muss, welche auch die die Vorgaben zur allgemeinen Verarbeitung der Patientendaten innerhalb des Krankenhauses regeln. \\
  \\ \hline
  4 &
  „Jede Datenaustauschplattform muss Datenschutzhinweise veröffentlichen und darin die getroffenen Datensicherheitslösungen allgemeinverständlich beschreiben.“ &
  "Matrix erlaubt es technisch Policys anzulegen, denen der User zustimmen muss, bevor er den Dienst nutzen kann. Die Datenschutzhinweise können somit, wie in der Anforderung beschrieben, als Policy angelegt werden. &
  Der Messenger muss die Möglichkeit bieten, die Datenschutzhinweise Anforderung 4-8 entsprechend den Nutzern zur Verfügung zu stellen. Die eigentliche Formulierung der Datenschutzhinweise ist nicht Teil dieser Arbeit, die technische Umsetzung, um diese auf die geforderte Weise zur Verfügung zu stellen jedoch schon. \\
  \\ \hline
  5 &
  „Datenschutzhinweise müssen unmittelbar von der Startseite der Datenaustauschplattform aus aufrufbar bzw. erreichbar sein.“ &
  Da es sich bei dem Matrix Frontend um Opensource-Software handelt, geschrieben in HTML und JavaScript, ist das Einfügen eines entsprechenden Verweises auf die Datenschutzhinweise auf der Startseite des Messengers problemlos möglich. &
  \\ \hline
6 &
  „Datenschutzhinweise müssen für den Betroffenen jederzeit abrufbar sein.“ &
  Da es sich bei dem Matrix Frontend um Opensource Software handelt, geschrieben in HTML und JavaScript, ist das Einfügen eines entsprechenden Verweises auf die Datenschutzhinweise auf der Startseite des Messenger problemlos möglich &
  \\ \hline
7 &
  „Bei nachträglicher Änderung ist der Betroffene zu informieren und sein Einverständnis erneut einzuholen.“ &
  Matrix erlaubt es technisch Policys anzulegen, denen der User zustimmen muss, bevor er den Dienst nutzen kann. Es ist auch möglich, den Nutzer, wie in der Anforderung beschrieben, nachträglich über Änderungen der Datenschutzhinweisen zu informieren und ein erneutes Einverständnis einzuholen. Die Datenschutzhinweise können somit, wie in der Anforderung beschrieben, als Policy angelegt werden. &
  \\ \hline
8 &
  „Wird die Einwilligung des Betroffenen elektronisch eingeholt, so muss der Vorgang protokolliert werden und der Inhalt der Einwilligung für den Betroffenen jederzeit abrufbar sein.“ &
  Matrix erlaubt den Nutzern über eine vom Dienst zur Verfügung gestellte URL jederzeit alle hinterlegten Policys einzusehen. Zudem verfolgt Matrix, ob der Nutzer den Policys zugestimmt hat. &
  \\ \hline
9 &
  „Personenbezogene oder personenbeziehbare Daten müssen pseudonymisiert werden, soweit dies nach dem Verwendungszweck möglich ist und keinen im Verhältnis zu dem angestrebten Schutzzweck unverhältnismäßigen Aufwand erfordert.“ &
  Im Kontext des Messengers ist eine Pseudonomisierung vor allem im Zusammenhang mit Fotos relevant, zum Beispiel von Patientenakten oder Testergebnissen. Matrix selbst bietet keine direkte Möglichkeit, Teilbereiche von Bildern zu psudonomisieren, da es sich aber sowohl beim Web Interface als auch der App um Opensource Code handelt, ist es durchaus möglich, dies in Form eines Plug-Ins zu ergänzen. Bis diese Funktion direkt in den Messenger integriert ist, sollten Bilder über die Foto App des Smartphones oder den Computer, wenn möglich vor dem Versenden, pseudonymisert werden. &
  Der Messengerdienst sollte den Nutzern aufgrund dieser Anforderung Funktionen zur Verfügung stellen, Daten wie zum Beispiel Bilddateien, die über den Dienst geteilt werden sollen, auf eine einfache Art und Weise zu pseudonymisieren. \\
  \\ \hline
13 &
  „Der Betroffene muss der Nutzung von Cookies explizit zustimmen.“ &
  Matrix erlaubt es technisch Policys anzulegen, denen der User zustimmen muss, bevor er den Dienst nutzen kann. Die rechtlich wirksame Einwilligung des Nutzers kann somit, wie in der Anforderung beschrieben, als Policy angelegt werden. &
  Die gesetzliche Anforderungen 13-15 an den Umgang mit Cookies ist vor allem für die Umsetzung der im Funktionsumfang festgehaltenen Anforderung, des Webzugangs zum Messengerdienst relevant. Der Messengerdienst muss dementsprechend technisch die Möglichkeit bieten, über mögliche Cookies zu informieren und dem Nutzer die Zustimmung zur Nutzung von Cookies ermöglichen. \\
  \\ \hline
14 &
  „Die Verwendung von Cookies durch Drittanbieter muss auf die Erstellung anonymer Nutzungsstatistiken beschränkt sein.“ &
  Matrix erlaubt es technisch Policys anzulegen, denen der User zustimmen muss, bevor er den Dienst nutzen kann. Die rechtlich wirksame Einwilligung des Nutzers kann somit, wie in der Anforderung beschrieben, als Policy angelegt werden. &
  \\ \hline
15 &
  „Das Handling der Drittanbieter-Cookies muss durch den Betroffenen steuer- und nachvollziehbar sein.“ &
  Matrix erlaubt es technisch Policys anzulegen, denen der User zustimmen muss, bevor er den Dienst nutzen kann. Die rechtlich wirksame Einwilligung des Nutzers kann somit, wie in der Anforderung beschrieben, als Policy angelegt werden. &
  \\ \hline
18 &
  „Die Erhebung, Verarbeitung und Nutzung von Daten zur Erstellung von Nutzungsstatistiken von Webportalen muss dokumentiert werden.“ &
  Die Erstellung von Statistiken auf Basis der Nutzeraktivitäten des Messenger ist nicht Bestandteil der Zielsetzung dieser Arbeit. Matrix beitet allerdings dennoch die Möglichkeit, eine entsprechende Funktion zu implementieren. &
  Aus Gründen der Privatsphäre ist von einem individuellen Tracken des Nutzerverhaltens auf der Plattform ab zu sehen, da es sich hierbei nicht um eine relevante Funktion für den Einsatz des Messengerdienstes handelt. Bei der nicht personenbezogenen Auswertungen des Nutzerverhaltens, zum Beispiel zur Überprüfung der Annahme des Dienstes seitens der Nutzer, sind Anforderungen 18-20 zu beachten. Allgemein fällt die Analyse nicht in die Zielsetzung der Arbeit und wird deshalb im Folgenden nicht weiter besprochen. \\
  \\ \hline
19 &
  „Die Erstellung von Nutzungsstatistiken von Datenaustauschplattformen muss auf Basis anonymisierter Daten erfolgen.“ &
  Die Erstellung von Statistiken auf Basis der Nutzeraktivitäten des Messenger ist nicht Bestandteil der Zielsetzung dieser Arbeit. Matrix beitet allerdings dennoch die Möglichkeit, eine entsprechende Funktion zu implementieren. &
  \\ \hline
20 &
  „Eine Analyse des Nutzerverhaltens (einschließlich Geolokalisierung) auf der Basis gekürzter IP-Adressen ist ohne Einwilligung des Betroffenen möglich, wenn durch die Kürzung der IP-Adresse ein Personenbezug ausgeschlossen wurde. &
  Die Erstellung von Statistiken auf Basis der Nutzeraktivitäten des Messenger ist nicht Bestandteil der Zielsetzung dieser Arbeit. Matrix beitet allerdings dennoch die Möglichkeit, eine entsprechende Funktion zu implementieren. &
  \\ \hline
31 &
  „Alle Personen, welche auf die gespeicherten Gesundheitsdaten zugreifen können, müssen vor dem erstmaligen Zugriff auf die Wahrung des Datengeheimnisses verpflichtet worden sein.“ &
  Matrix erlaubt es technisch Policys anzulegen, denen der User zustimmen muss, bevor er den Dienst nutzen kann. Die rechtlich wirksame Einwilligung des Nutzers kann somit, wie in der Anforderung beschrieben, als Policy angelegt werden." &
  Für den Messengerdienst muss eine technische Lösung implementiert werden, welche es erlaubt, dem Nutzer eine solche Warnung zu präsentieren. \\
  \\ \hline
  32 &
  „Ein anonymer Zugriff des Plattformbetreibers sowie der Nutzer der Plattform auf personenbezogene oder personenbeziehbare Gesundheitsdaten ist zu unterbinden.“ &
  Um einen anonymen Zugriff auf den Dienst vorzubeugen, darf der Zugriff nur über einen, in der Nutzerverwaltung des Krankenhauses hinterlegten Account auf den Messenger geschehen. Dies ist in Matrix umsetzbar. Zudem sollte über die Nutzerverwaltung das Loginverhalten der Nutzer überwacht werden, und gegebenenfalls eine 2 Faktor Authentifikation vom Nutzer gefordert werden. Diese Funktionen müssen über die angebundene Nutzerverwaltung umgesetzt werden. Auf der Server Seite muss der Zugriff auf die einzelnen Komponenten des Messengers ebenfalls überwacht werden. Eine personengebundene Zuordnung kann zum Beispiel über personenspezifische SSH Keys überwacht werden und Zugriffs-Logs getrackt werden. &
  Es muss technisch sichergestellt werden, dass sämtliche Zugriffe auf den Dienst zu jeder Zeit einer realen Person zugeordnet werden können. Dies gilt auch für den direkten Zugriff auf die mit dem Dienst verbundenen Datenbanken durch Administratoren. \\
  \\ \hline
  33 &
  „Daten über den Ablauf des Zugriffs oder der sonstigen Nutzung sind unmittelbar nach dem Zugriff bzw. ggfs. nach erfolgter Abrechnung der Dienstnutzung zu sperren und nur zu entsperren, wenn eine gesetzliche Bestimmung (z.B. Auskunftsersuchen eines Betroffenen) dies erlaubt.“ &
  Diese Anforderung betrifft mehrere Funktionen, die mit Matrix umsetzbar sind: Um sicher zu stellen, dass keine sensiblen Daten länger als notwendig auf dem Messenger gespeichert sind, bietet Matrix die Funktion, ausgewählte Räume zu erstellen, mit einer zeitlich eingeschränkten Speicherung des Chat Verlaufs, beispielsweise 48 Stunden. Diese Funktion muss für alle Räume aktiviert werden, in denen Nutzer sensible Daten austauschen wollen. Hierbei ist es noch einmal wichtig anzumerken, dass der Messenger nur zum schnellen Austausch von Gesundheitsdaten vorgesehen ist. Falls Daten längerfristig gespeichert werden sollen, müssen diese in ein anderes System, welches für diese Aufgabe ausgelegt ist, überführt werden, zum Beispiel das Krankenhausinformationssystem. Der Zugriff auf den Messenger darf nur solange möglich sein, wie der Mitarbeiter für das Krankenhaus arbeitet. Aus diesem Grund darf ein Login nur über das Nutzerverzeichnis des Krankenhauses möglich sein. Wenn der Nutzer durch das Offboarding aus diesem System entfernt wird, kann er sich nicht mehr im Messengerdienst anmelden. &
  Der Messenger muss es erlauben, Daten zeitlich begrenzt zu teilen, damit sichergestellt werden kann, dass Daten nur für die Dauer ihres benötigten Verwendungszwecks über den Messenger abrufbar sind. Zudem muss der Zugriff auf das System gesperrt werden, sobald ein Mitarbeiter das Krankenhaus verlässt. Ebenso müssen Mitarbeiter aus Gruppen entfernt werden, in denen personenbezogene sensible Daten ausgetauscht werden, welche für die Person nicht, oder nicht mehr relevant sind. \\
  \\ \hline
  34 &
  „Die Nutzung von Telemedien muss gegen die Kenntnisnahme unberechtigter Dritter geschützt werden.“ &
  Um einen anonymen Zugriff auf den Dienst vorzubeugen, darf der Zugriff nur über einen, in der Nutzerverwaltung des Krankenhauses hinterlegten Account auf den Messenger geschehen. Zudem sollte über die Nutzerverwaltung das Loginverhalten der Nutzer überwacht werden, und gegebenenfalls eine 2 Faktor Authentifikation vom Nutzer gefordert werden. Auf der Serverseite muss der Zugriff auf die einzelnen Komponenten des Messengers ebenfalls überwacht werden. Eine personengebundene Zuordnung kann zum Beispiel über personenspezifische SSH Keys überwacht werden und Zugriffs-Logs getrackt werden. &
  Eine Anmeldung und Verifizierung des Nutzers muss vor dem Zugriff auf den Messenger erfolgen. Ebenso muss der Zugriff, auf den dem Messenger zugeordnete Software Code und entsprechenden Datenbanken geschützt werden. Dies muss durch eine technische Lösung sichergestellt werden. \\
  \\ \hline
  39 &
  „Ereignisse, die potenziell dazu führen können, dass ein unberechtigter Zugriff auf personenbezogene oder personenbeziehbare Daten erfolgen könnte, sind zu protokollieren.“ &
  Diese Anforderung bezieht sich nicht direkt auf den Messenger. Zugriffe müssen durch eine entsprechende Funktion in der Nutzerverwaltung des Krankenhauses, über die der Login auf den Messenger erfolgt, festgehalten werden. Ebenso müssen die Logs der Server auf denen Matrix betrieben wird entsprechend konfiguriert werden. &
  Es muss sichergestellt werden, dass mögliche, unberechtigte Zugriffe in den Logs des Dienstes festgehalten werden und entsprechend archiviert werden. \\
  \\ \hline
41 &
  „Der Betroffene muss die Möglichkeit haben, jederzeit Einblick in alle zu seiner Person gespeicherten Daten zu erhalten. Dies umfasst auch die Möglichkeit, Änderungen seiner gespeicherten Daten nachzuvollziehen.“ &
  Da alle über Matrix versendeten Daten in der für das Krankenhaus vorgesehenen Konfiguration End zu End verschlüsselt sein werden, handelt es sich bei den gespeicherten personenbezogenen Daten ausschließlich um die in den Nutzerprofilen angegebenen Informationen. Diese können selbstständig vom Nutzer eingesehen und angepasst werden. Zudem ist es aber auch für den Support des Dienstes möglich, durch einen entsprechenden Query in der an Matrix angebundenen Datenbank alle personenbezogenen Daten des entsprechenden Nutzerprofils abzufragen, um sie entsprechend an den Nutzer weiterzuleiten. &
  Anforderung 41-43 betreffen im Kontext des Messengers nur die Daten der eigentlichen Nutzer, da keine langfristige Speicherung der Gesundheitsdaten über den Messenger vorgesehen ist. Die geforderten Möglichkeiten in Bezug auf Patientendaten müssen bei den Systemen vorgenommen werden, welche zur langfristigen Speicherung der Patientendaten vorgesehen sind. Was die eigentlichen Nutzerdaten betrifft sind technische Lösungen zu implementieren, welche die Umsetzung der beschriebenen Anforderungen erlaubt. \\
  \\ \hline
42 &
  „Der Betroffene muss die Möglichkeit haben, einen Ausdruck oder einen für ihn verwertbaren Export aller zu seiner Person gespeicherten Daten zu erhalten.“ &
  Da alle über Matrix versendeten Daten in der für das Krankenhaus vorgesehenen Konfiguration End zu End verschlüsselt sein werden, handelt es sich bei den über die Nutzer des Dienstes gespeicherten personenbezogenen Daten ausschließlich um die im Nutzerprofil angegebenen Informationen. Diese können selbstständig vom Nutzer eingesehen und angepasst werden. Zudem ist es aber auch für den Support des Dienstes möglich, durch einen entsprechenden Query in der an Matrix angebundenen Datenbank alle personenbezogenen Daten des entsprechenden Nutzerprofils abzufragen, um sie entsprechend an den Nutzer weiterzu eiten. &
  \\ \hline
43 &
  „Es muss eine Möglichkeit geben, auf Aufforderung des Betroffenen Daten zu seiner Person zu korrigieren.“ &
  Da alle über Matrix versendeten Daten in der für das Krankenhaus vorgesehenen Konfiguration End zu End verschlüsselt sein werden, handelt es sich bei den über die Nutzer des Dienstes gespeicherten personenbezogenen Daten ausschließlich um die im Nutzerprofil angegebenen Informationen. Diese können selbstständig vom Nutzer eingesehen und angepasst werden. Zudem ist es aber auch für den Support des Dienstes möglich, durch einen entsprechenden Query in der an Matrix angebundenen Datenbank alle personenbezogenen Daten des entsprechenden Nutzerprofils abzufragen, um sie entsprechend an den Nutzer weiterzuleiten. &
  \\ \hline
46 &
  „Es muss eine Löschfunktion implementiert werden, welche eine Rekonstruktion gelöschter Informationen ausschließt.“ &
  Matrix erlaubt die Löschung von Nutzern. Dabei können alle im Profil des Nutzers hinterlegten Informationen sowie alle vom Nutzer versendeten Nachrichten gelöscht werden. &
  Eine permanente Löschung betroffener Daten muss über den Messenger technisch möglich sein. \\
  \\ \hline
58 &
  „Da im Internet eine potenziell größere Gefährdung für einen unbefugten Zugriff auf personenbezogene Daten existiert, muss mindestens eine 2-Faktor-Authentifizierung erfolgen.“ &
  Diese Anforderung betrifft Matrix nicht direkt. Der Login erfolgt über die Nutzerverwaltung des Krankenhauses. Somit muss dieses in der Lage sein 2 Faktor Authentifikation zu unterstützen. &
  Die Möglichkeit einer 2 Faktor Authentifikation muss in den Messengerdienst implementiert werden. \\
  \\ \hline
59 &
  „Werden statische Passwörter zur Authentisierung eingesetzt, so müssen die Empfehlungen des BSI bzgl. der Generierung und des Umgangs eingehalten werden. D.h. es müssen technische und organisatorische Maßnahmen zu der Einhaltung der Empfehlungen des BSI getroffen werden.“ &
  Im Konzept werden keine Nutzer von Matrix selbst angelegt, sondern nur über die Nutzerverwaltung des Krankenhauses. Die Komplexität des Nutzerpassworts muss also dort festgelegt werden. &
  Die Mindestanforderungen an die Komplexität, der für die Nutzung des Messengers zu erstellenden Passwörtern, muss durch eine technische Lösung festlegbar sein. \\
  \\ \hline
60 &
  „Authentisierungsgeheimnisse dürfen nur gesichert in Netzwerken übertragen werden, d.h. es muss eine verschlüsselte Datenübertragung entsprechend dem Stand der Technik eingesetzt werden.“ &
  Matrix nutzt zur Kommunikation zwischen Frontend und Homeserver http APIs. Durch die Verwendung von Hypertext Transfer Protocol Secure (HTTPS), einer Erweiterung des http Protokolls, geschieht die Kommunikation verschlüsselt. Dies entspricht dem aktuellen Stand der Technik. &
  Es muss technisch sichergestellt werden, dass nur über eine gesicherte Netzwerkverbindung mit dem Dienst kommuniziert werden kann. \\
  \\ \hline
62 &
  „Nach wiederholter fehlerhafter Authentisierung muss der Zugang gesperrt werden.“ &
  Im Konzept erfolgt die Authentifikation von Nutzern gegenüber Matrix nicht durch Matrix selbst, sondern nur über die Nutzerverwaltung des Krankenhauses. In dieser müssen entsprechende Einstellungen vorgenommen werden, um widerholten, fehlerhaften Login versuche vorzubeugen. &
  Eine technische Möglichkeit muss implementiert werden, welche es erlaubt, nach zu häufiger, fehlerhafter Authentifikation betroffene Zugänge zu sperren. \\
  \\ \hline
63 &
  „Die Sitzung muss gesperrt oder beendet werden, wenn der Anwender eine definierte Zeitspanne in der Sitzung keine Aktivitäten durchführte (Sitzungs-Zeitlimit, Session Timeout).“ &
  Nach der Authentifikation gegenüber dem Homeserver erhält der Nutzer ein Token mit einer zeitlich begrenzten Gültigkeit in Form eines Cookies. Nach Ablauf des Tokens muss sich der Nutzer erneut gegenüber dem Homeserver authentisieren. Die Zeitspanne, in der ein Token gültig ist, kann im Homeserver konfiguriert werden. &
  Eine technische Möglichkeit muss implementiert werden, welche es erlaubt, eine Zeitspanne zu definieren, nach welcher eine erneute Authentifizierung vom Nutzer gegenüber des Messengers notwendig ist. Dies betrifft vor allem den Zugriff auf das Webinterface des Messengers. \\
  \\ \hline
64 &
  „Ein Prozess zur Rücksetzung bzw. Entsperrung von gesperrten Zugangskennungen ist einzurichten, zu beschreiben und anzuwenden.“ &
  "Im Konzept erfolgt die Authentifikation von Nutzern gegenüber Matrix nicht durch Matrix selbst, sondern nur über die Nutzerverwaltung des Krankenhauses. In dieser müssen entsprechende Einstellungen vorgenommen werden, um die Entsperrung und Rücksetzung von Nutzerkonten zu ermöglichen. &
  Eine technische Möglichkeit muss implementiert werden, welche die Umsetzung von Anforderung 64 ermöglicht. \\
  \\ \hline
66 &
  „Benutzerkennungen, welche über einen definierten Zeitraum nicht benutzt wurden, sind zu sperren bzw. auf inaktiv zu setzen.“ &
  Im Konzept erfolgt die Authentifikation von Nutzern gegenüber Matrix nicht durch Matrix selbst, sondern nur über die Nutzerverwaltung des Krankenhauses. In dieser müssen entsprechende Einstellungen vorgenommen werden, welche Nutzer, die sich über einen längeren Zeitraum nicht auf das System eingeloggt haben, sperrt. &
  Eine technische Möglichkeit muss implementiert werden, welche die Umsetzung von Anforderung 66 ermöglicht. \\
  \\ \hline
71 &
  „Eine Kombination von Rollen bzw. Zugriffsrechten für eine Person, welche der Person mehr Rechte auf Datenzugriffe erteilt, als für ihre Aufgabe nötig ist, ist zu verhindern. Es sind technische und organisatorische Maßnahmen zu ergreifen, um dies sicherzustellen.“ &
  Dank der Administrations Tools von Matrix kann das Hinzufügen und Entfernen aus einer Gruppe von Nutzern mit Admin Rechten für die entbrechenden selbstständig durchgeführt werden. Dies gibt den Administratoren viel Flexibilität, da sie diesen Prozess selbstständig durchführen können, es muss allerdings auch darauf geachtet werden, dass auch wenn Personen innerhalb des Krankenhauses Positionen wechseln das hinzufügen oder entfernen aus Gruppen Teil des Off-/Onboarung Prozesses ist. &
  Für den Messenger muss ein Anforderung 71 entsprechendes Rollenkonzept technisch umgesetzt werden. \\
  \\ \hline
73 &
  „Protokolldaten sind gegen unbefugten Zugriff in geeigneter Weise entsprechend dem Stand der Technik zu schützen.“ &
  "Die vom Messenger erstellten Protokolldaten werden auf dem Server, auf welchem der Matrix Homeserver betrieben wird gespeichert. Der Zugriff auf diesen Server muss entsprechend geschützt werden. &
  Die vom Messenger erstellten Protokolle sind dem Stand der Technik entsprechend zu schützen. \\
  \\ \hline
81 &
  „Die Übertragung personenbezogener oder personenbeziehbarer Gesundheitsdaten zwischen Clients und Servern wie auch zwischen Servern selbst muss entsprechend dem jeweiligen Stand der Technik generell verschlüsselt erfolgen.“ &
  Matrix nutzt zur Kommunikation zwischen Frontend und Homeserver http APIs. Durch die Verwendung von Hypertext Transfer Protocol Secure (HTTPS), einer Erweiterung des http Protokolls, geschieht die Kommunikation verschlüsselt. Dies entspricht dem aktuellen Stand der Technik. &
  Eine Verschlüsselung wie sie in Anforderung 81 definiert ist muss für den Messenger implantiert werden. \\
  \\ \hline
86 &
  „Gesundheitsdaten sind entsprechend dem Stand der Technik verschlüsselt in der Datenbank zu speichern, so dass bei administrativen Zugriffen Wartungspersonal keinen unbefugten Zugriff auf die gespeicherten Daten erhalten kann.“ &
  Matrix erlaubt eine Anforderung 86 entsprechende Verschlüsselung der Daten. Wie Matrix seine Daten verschlüsselt wird in Kapitel 3.1.2 genauer erläutert. &
  Eine Verschlüsselung wie sie in Anforderung 86 definiert ist muss für den Messenger implantiert werden. \\
  \\ \hline
87 &
  „Nur Personen, die laut Berechtigungskonzept zu einer Eingabe berechtigt sind, dürfen personenbezogene oder personenbeziehbare Daten in eine Datenaustauschplattform eingeben.“ &
  Anforderung 87 muss auf organisatorischer Ebene umgesetzt werden. Falls es dennoch notwendig ist, einzelnen Personen auf einer technischen Ebene das Teilen von Inhalten in bestimmten Räumen zu verbieten, ist dies als Administrator des Raumes möglich. &
  Anforderung 87 und 88 sind vor allem für Systeme gedacht, welche für die langfristige Speicherung von Gesundheitsdaten gedacht sind. Um sicher zu stellen, dass nur dann Patientendaten über den Messenger geteilt werden, wenn es wirklich sinnvoll ist, kann es dennoch hilfreich sein, auf organisatorischer Ebene nur bestimmten Personen zu erlauben Patientendaten über den Messenger zu teilen. \\
  \\ \hline
88 &
  „Nur Personen, die laut Berechtigungskonzept zu einem Datenimport berechtigt sind, dürfen einen Import personenbezogener oder personenbeziehbarer Daten in eine Datenaustauschplattform durchführen oder veranlassen.“ &
  Anforderung 88 muss auf organisatorischer Ebene umgesetzt werden. Falls es dennoch notwendig ist, einzelnen Personen auf einer technischen Ebene das Teilen von Inhalten in bestimmten Räumen zu verbieten, ist dies als Administrator des Raumes möglich. &
  \\ \hline
93 &
  „Es muss ein Backup-Konzept vorhanden sein, welches gewährleistet, dass die Daten nach einem Vorfall in angemessener Zeit wieder zur Verfügung gestellt werden können. In diesem Backup-Konzept muss berücksichtigt werden, dass nur berechtigte Personen Zugriff auf Backup-Daten erlangen können.“ &
  Matrix erlaubt es die Chatverläufe auf einem SQL Server zu speichern, von welchem Backups erstellt werden können. Somit können alle über den Messenger verarbeiteten Daten gesichert werden. &
  Die technische Möglichkeit der Erstellung regelmäßiger Backups muss gewährleistet sein.
  \\ \hline
  \caption{Gesetzlicher Anforderungskatalog an einen Messengerdienst für den Austausch von Gesundheitsdaten, erweitert mit den Ansätzen zur Umsetzung durch das Matrix Framework} 
\end{longtable}

\anhang{Muster homeserver.yaml}

Bei dem hier hinterlegten Dokument handelt es sich um eine Beispielkonfiguration der homeserver.yaml, über welche der Homeserver einer Matrix Instanz konfiguriert werden kann.

\begin{lstlisting}
    # vim:ft=yaml
    # PEM encoded X509 certificate for TLS.
    # You can replace the self-signed certificate that synapse
    # autogenerates on launch with your own SSL certificate + key pair
    # if you like.  Any required intermediary certificates can be
    # appended after the primary certificate in hierarchical order.
    tls_certificate_path: "/etc/matrix-synapse/homeserver.tls.crt"
    
    # PEM encoded private key for TLS
    tls_private_key_path: "/etc/matrix-synapse/homeserver.tls.key"
    
    # PEM dh parameters for ephemeral keys
    tls_dh_params_path: "/etc/matrix-synapse/homeserver.tls.dh"
    
    # Don't bind to the https port
    no_tls: False
    
    # List of allowed TLS fingerprints for this server to publish along
    # with the signing keys for this server. Other matrix servers that
    # make HTTPS requests to this server will check that the TLS
    # certificates returned by this server match one of the fingerprints.
    #
    # Synapse automatically adds the fingerprint of its own certificate
    # to the list. So if federation traffic is handled directly by synapse
    # then no modification to the list is required.
    #
    # If synapse is run behind a load balancer that handles the TLS then it
    # will be necessary to add the fingerprints of the certificates used by
    # the loadbalancers to this list if they are different to the one
    # synapse is using.
    #
    # Homeservers are permitted to cache the list of TLS fingerprints
    # returned in the key responses up to the "valid_until_ts" returned in
    # key. It may be necessary to publish the fingerprints of a new
    # certificate and wait until the "valid_until_ts" of the previous key
    # responses have passed before deploying it.
    #
    # You can calculate a fingerprint from a given TLS listener via:
    # openssl s_client -connect $host:$port < /dev/null 2> /dev/null |
    #   openssl x509 -outform DER | openssl sha256 -binary | base64 | tr -d '='
    # or by checking matrix.org/federationtester/api/report?server_name=$host
    #
    tls_fingerprints: []
    # tls_fingerprints: [{"sha256": "<base64_encoded_sha256_fingerprint>"}]
    
    
    ## Server ##
    
    # When running as a daemon, the file to store the pid in
    pid_file: "/var/run/matrix-synapse.pid"
    
    # CPU affinity mask. Setting this restricts the CPUs on which the
    # process will be scheduled. It is represented as a bitmask, with the
    # lowest order bit corresponding to the first logical CPU and the
    # highest order bit corresponding to the last logical CPU. Not all CPUs
    # may exist on a given system but a mask may specify more CPUs than are
    # present.
    #
    # For example:
    #    0x00000001  is processor #0,
    #    0x00000003  is processors #0 and #1,
    #    0xFFFFFFFF  is all processors (#0 through #31).
    #
    # Pinning a Python process to a single CPU is desirable, because Python
    # is inherently single-threaded due to the GIL, and can suffer a
    # 30-40% slowdown due to cache blow-out and thread context switching
    # if the scheduler happens to schedule the underlying threads across
    # different cores. See
    # https://www.mirantis.com/blog/improve-performance-python-programs-restricting-single-cpu/.
    #
    # cpu_affinity: 0xFFFFFFFF
    
    # Whether to serve a web client from the HTTP/HTTPS root resource.
    web_client: False
    
    # The root directory to server for the above web client.
    # If left undefined, synapse will serve the matrix-angular-sdk web client.
    # Make sure matrix-angular-sdk is installed with pip if web_client is True
    # and web_client_location is undefined
    # web_client_location: "/path/to/web/root"
    
    # The public-facing base URL for the client API (not including _matrix/...)
    # public_baseurl: https://example.com:8448/
    
    # Set the soft limit on the number of file descriptors synapse can use
    # Zero is used to indicate synapse should set the soft limit to the
    # hard limit.
    soft_file_limit: 0
    
    # The GC threshold parameters to pass to `gc.set_threshold`, if defined
    # gc_thresholds: [700, 10, 10]
    
    # Set the limit on the returned events in the timeline in the get
    # and sync operations. The default value is -1, means no upper limit.
    # filter_timeline_limit: 5000
    
    # Whether room invites to users on this server should be blocked
    # (except those sent by local server admins). The default is False.
    # block_non_admin_invites: True
    
    # Restrict federation to the following whitelist of domains.
    # N.B. we recommend also firewalling your federation listener to limit
    # inbound federation traffic as early as possible, rather than relying
    # purely on this application-layer restriction.  If not specified, the
    # default is to whitelist everything.
    #
    # federation_domain_whitelist:
    #  - lon.example.com
    #  - nyc.example.com
    #  - syd.example.com
    
    # List of ports that Synapse should listen on, their purpose and their
    # configuration.
    listeners:
      # Main HTTPS listener
      # For when matrix traffic is sent directly to synapse.
      -
        # The port to listen for HTTPS requests on.
        port: 8448
    
        # Local addresses to listen on.
        # On Linux and Mac OS, `::` will listen on all IPv4 and IPv6
        # addresses by default. For most other OSes, this will only listen
        # on IPv6.
        bind_addresses:
          - '::'
          - '0.0.0.0'
    
        # This is a 'http' listener, allows us to specify 'resources'.
        type: http
    
        tls: true
    
        # Use the X-Forwarded-For (XFF) header as the client IP and not the
        # actual client IP.
        x_forwarded: false
    
        # List of HTTP resources to serve on this listener.
        resources:
          -
            # List of resources to host on this listener.
            names:
              - client     # The client-server APIs, both v1 and v2
              - webclient  # The bundled webclient.
    
            # Should synapse compress HTTP responses to clients that support it?
            # This should be disabled if running synapse behind a load balancer
            # that can do automatic compression.
            compress: true
    
          - names: [federation]  # Federation APIs
            compress: false
    
        # optional list of additional endpoints which can be loaded via
        # dynamic modules
        # additional_resources:
        #   "/_matrix/my/custom/endpoint":
        #     module: my_module.CustomRequestHandler
        #     config: {}
    
      # Unsecure HTTP listener,
      # For when matrix traffic passes through loadbalancer that unwraps TLS.
      - port: 8008
        tls: false
        bind_addresses: ['::', '0.0.0.0']
        type: http
    
        x_forwarded: false
    
        resources:
          - names: [client, webclient]
            compress: true
          - names: [federation]
            compress: false
    
      # Turn on the twisted ssh manhole service on localhost on the given
      # port.
      # - port: 9000
      #   bind_addresses: ['::1', '127.0.0.1']
      #   type: manhole
    
    
    # Database configuration
    database:
      # The database engine name
      name: "sqlite3"
      # Arguments to pass to the engine
      args:
        # Path to the database
        database: "/var/lib/matrix-synapse/homeserver.db"
    
    # Number of events to cache in memory.
    event_cache_size: "10K"
    
    
    # A yaml python logging config file
    log_config: "/etc/matrix-synapse/log.yaml"
    
    
    
    ## Ratelimiting ##
    
    # Number of messages a client can send per second
    rc_messages_per_second: 0.2
    
    # Number of message a client can send before being throttled
    rc_message_burst_count: 10.0
    
    # The federation window size in milliseconds
    federation_rc_window_size: 1000
    
    # The number of federation requests from a single server in a window
    # before the server will delay processing the request.
    federation_rc_sleep_limit: 10
    
    # The duration in milliseconds to delay processing events from
    # remote servers by if they go over the sleep limit.
    federation_rc_sleep_delay: 500
    
    # The maximum number of concurrent federation requests allowed
    # from a single server
    federation_rc_reject_limit: 50
    
    # The number of federation requests to concurrently process from a
    # single server
    federation_rc_concurrent: 3
    
    
    
    # Directory where uploaded images and attachments are stored.
    media_store_path: "/var/lib/matrix-synapse/media"
    
    # Media storage providers allow media to be stored in different
    # locations.
    # media_storage_providers:
    # - module: file_system
    #   # Whether to write new local files.
    #   store_local: false
    #   # Whether to write new remote media
    #   store_remote: false
    #   # Whether to block upload requests waiting for write to this
    #   # provider to complete
    #   store_synchronous: false
    #   config:
    #     directory: /mnt/some/other/directory
    
    # Directory where in-progress uploads are stored.
    uploads_path: "/var/lib/matrix-synapse/uploads"
    
    # The largest allowed upload size in bytes
    max_upload_size: "10M"
    
    # Maximum number of pixels that will be thumbnailed
    max_image_pixels: "32M"
    
    # Whether to generate new thumbnails on the fly to precisely match
    # the resolution requested by the client. If true then whenever
    # a new resolution is requested by the client the server will
    # generate a new thumbnail. If false the server will pick a thumbnail
    # from a precalculated list.
    dynamic_thumbnails: false
    
    # List of thumbnail to precalculate when an image is uploaded.
    thumbnail_sizes:
    - width: 32
      height: 32
      method: crop
    - width: 96
      height: 96
      method: crop
    - width: 320
      height: 240
      method: scale
    - width: 640
      height: 480
      method: scale
    - width: 800
      height: 600
      method: scale
    
    # Is the preview URL API enabled?  If enabled, you *must* specify
    # an explicit url_preview_ip_range_blacklist of IPs that the spider is
    # denied from accessing.
    url_preview_enabled: False
    
    # List of IP address CIDR ranges that the URL preview spider is denied
    # from accessing.  There are no defaults: you must explicitly
    # specify a list for URL previewing to work.  You should specify any
    # internal services in your network that you do not want synapse to try
    # to connect to, otherwise anyone in any Matrix room could cause your
    # synapse to issue arbitrary GET requests to your internal services,
    # causing serious security issues.
    #
    # url_preview_ip_range_blacklist:
    # - '127.0.0.0/8'
    # - '10.0.0.0/8'
    # - '172.16.0.0/12'
    # - '192.168.0.0/16'
    # - '100.64.0.0/10'
    # - '169.254.0.0/16'
    #
    # List of IP address CIDR ranges that the URL preview spider is allowed
    # to access even if they are specified in url_preview_ip_range_blacklist.
    # This is useful for specifying exceptions to wide-ranging blacklisted
    # target IP ranges - e.g. for enabling URL previews for a specific private
    # website only visible in your network.
    #
    # url_preview_ip_range_whitelist:
    # - '192.168.1.1'
    
    # Optional list of URL matches that the URL preview spider is
    # denied from accessing.  You should use url_preview_ip_range_blacklist
    # in preference to this, otherwise someone could define a public DNS
    # entry that points to a private IP address and circumvent the blacklist.
    # This is more useful if you know there is an entire shape of URL that
    # you know that will never want synapse to try to spider.
    #
    # Each list entry is a dictionary of url component attributes as returned
    # by urlparse.urlsplit as applied to the absolute form of the URL.  See
    # https://docs.python.org/2/library/urlparse.html#urlparse.urlsplit
    # The values of the dictionary are treated as an filename match pattern
    # applied to that component of URLs, unless they start with a ^ in which
    # case they are treated as a regular expression match.  If all the
    # specified component matches for a given list item succeed, the URL is
    # blacklisted.
    #
    # url_preview_url_blacklist:
    # # blacklist any URL with a username in its URI
    # - username: '*'
    #
    # # blacklist all *.google.com URLs
    # - netloc: 'google.com'
    # - netloc: '*.google.com'
    #
    # # blacklist all plain HTTP URLs
    # - scheme: 'http'
    #
    # # blacklist http(s)://www.acme.com/foo
    # - netloc: 'www.acme.com'
    #   path: '/foo'
    #
    # # blacklist any URL with a literal IPv4 address
    # - netloc: '^[0-9]+\.[0-9]+\.[0-9]+\.[0-9]+$'
    
    # The largest allowed URL preview spidering size in bytes
    max_spider_size: "10M"
    
    
    
    
    ## Captcha ##
    # See docs/CAPTCHA_SETUP for full details of configuring this.
    
    # This Home Server's ReCAPTCHA public key.
    recaptcha_public_key: "YOUR_PUBLIC_KEY"
    
    # This Home Server's ReCAPTCHA private key.
    recaptcha_private_key: "YOUR_PRIVATE_KEY"
    
    # Enables ReCaptcha checks when registering, preventing signup
    # unless a captcha is answered. Requires a valid ReCaptcha
    # public/private key.
    enable_registration_captcha: False
    
    # A secret key used to bypass the captcha test entirely.
    #captcha_bypass_secret: "YOUR_SECRET_HERE"
    
    # The API endpoint to use for verifying m.login.recaptcha responses.
    recaptcha_siteverify_api: "https://www.google.com/recaptcha/api/siteverify"
    
    
    ## Turn ##
    
    # The public URIs of the TURN server to give to clients
    turn_uris: []
    
    # The shared secret used to compute passwords for the TURN server
    turn_shared_secret: "YOUR_SHARED_SECRET"
    
    # The Username and password if the TURN server needs them and
    # does not use a token
    #turn_username: "TURNSERVER_USERNAME"
    #turn_password: "TURNSERVER_PASSWORD"
    
    # How long generated TURN credentials last
    turn_user_lifetime: "1h"
    
    # Whether guests should be allowed to use the TURN server.
    # This defaults to True, otherwise VoIP will be unreliable for guests.
    # However, it does introduce a slight security risk as it allows users to
    # connect to arbitrary endpoints without having first signed up for a
    # valid account (e.g. by passing a CAPTCHA).
    turn_allow_guests: False
    
    
    ## Registration ##
    
    # Enable registration for new users.
    enable_registration: False
    
    # The user must provide all of the below types of 3PID when registering.
    #
    # registrations_require_3pid:
    #     - email
    #     - msisdn
    
    # Mandate that users are only allowed to associate certain formats of
    # 3PIDs with accounts on this server.
    #
    # allowed_local_3pids:
    #     - medium: email
    #       pattern: ".*@matrix\.org"
    #     - medium: email
    #       pattern: ".*@vector\.im"
    #     - medium: msisdn
    #       pattern: "\+44"
    
    # If set, allows registration by anyone who also has the shared
    # secret, even if registration is otherwise disabled.
    # registration_shared_secret: <PRIVATE STRING>
    
    # Set the number of bcrypt rounds used to generate password hash.
    # Larger numbers increase the work factor needed to generate the hash.
    # The default number is 12 (which equates to 2^12 rounds).
    # N.B. that increasing this will exponentially increase the time required
    # to register or login - e.g. 24 => 2^24 rounds which will take >20 mins.
    bcrypt_rounds: 12
    
    # Allows users to register as guests without a password/email/etc, and
    # participate in rooms hosted on this server which have been made
    # accessible to anonymous users.
    allow_guest_access: False
    
    # The list of identity servers trusted to verify third party
    # identifiers by this server.
    trusted_third_party_id_servers:
        - matrix.org
        - vector.im
        - riot.im
    
    # Users who register on this homeserver will automatically be joined
    # to these rooms
    #auto_join_rooms:
    #    - "#example:example.com"
    
    
    ## Metrics ###
    
    # Enable collection and rendering of performance metrics
    enable_metrics: False
    
    ## API Configuration ##
    
    # A list of event types that will be included in the room_invite_state
    room_invite_state_types:
        - "m.room.join_rules"
        - "m.room.canonical_alias"
        - "m.room.avatar"
        - "m.room.name"
    
    
    # A list of application service config file to use
    app_service_config_files: []
    
    
    # macaroon_secret_key: <PRIVATE STRING>
    
    # Used to enable access token expiration.
    expire_access_token: False
    
    ## Signing Keys ##
    
    # Path to the signing key to sign messages with
    signing_key_path: "/etc/matrix-synapse/homeserver.signing.key"
    
    # The keys that the server used to sign messages with but won't use
    # to sign new messages. E.g. it has lost its private key
    old_signing_keys: {}
    #  "ed25519:auto":
    #    # Base64 encoded public key
    #    key: "The public part of your old signing key."
    #    # Millisecond POSIX timestamp when the key expired.
    #    expired_ts: 123456789123
    
    # How long key response published by this server is valid for.
    # Used to set the valid_until_ts in /key/v2 APIs.
    # Determines how quickly servers will query to check which keys
    # are still valid.
    key_refresh_interval: "1d" # 1 Day.
    
    # The trusted servers to download signing keys from.
    perspectives:
      servers:
        "matrix.org":
          verify_keys:
            "ed25519:auto":
              key: "Noi6WqcDj0QmPxCNQqgezwTlBKrfqehY1u2FyWP9uYw"
    
    
    
    # Enable SAML2 for registration and login. Uses pysaml2
    # config_path:      Path to the sp_conf.py configuration file
    # idp_redirect_url: Identity provider URL which will redirect
    #                   the user back to /login/saml2 with proper info.
    # See pysaml2 docs for format of config.
    #saml2_config:
    #   enabled: true
    #   config_path: "/home/erikj/git/synapse/sp_conf.py"
    #   idp_redirect_url: "http://test/idp"
    
    
    
    # Enable CAS for registration and login.
    #cas_config:
    #   enabled: true
    #   server_url: "https://cas-server.com"
    #   service_url: "https://homeserver.domain.com:8448"
    #   #required_attributes:
    #   #    name: value
    
    
    # The JWT needs to contain a globally unique "sub" (subject) claim.
    #
    # jwt_config:
    #    enabled: true
    #    secret: "a secret"
    #    algorithm: "HS256"
    
    
    
    # Enable password for login.
    password_config:
       enabled: true
       # Uncomment and change to a secret random string for extra security.
       # DO NOT CHANGE THIS AFTER INITIAL SETUP!
       #pepper: ""
    
    
    
    # Enable sending emails for notification events
    # Defining a custom URL for Riot is only needed if email notifications
    # should contain links to a self-hosted installation of Riot; when set
    # the "app_name" setting is ignored.
    #
    # If your SMTP server requires authentication, the optional smtp_user &
    # smtp_pass variables should be used
    #
    #email:
    #   enable_notifs: false
    #   smtp_host: "localhost"
    #   smtp_port: 25
    #   smtp_user: "exampleusername"
    #   smtp_pass: "examplepassword"
    #   require_transport_security: False
    #   notif_from: "Your Friendly %(app)s Home Server <noreply@example.com>"
    #   app_name: Matrix
    #   template_dir: res/templates
    #   notif_template_html: notif_mail.html
    #   notif_template_text: notif_mail.txt
    #   notif_for_new_users: True
    #   riot_base_url: "http://localhost/riot"
    
    
    # password_providers:
    #     - module: "ldap_auth_provider.LdapAuthProvider"
    #       config:
    #         enabled: true
    #         uri: "ldap://ldap.example.com:389"
    #         start_tls: true
    #         base: "ou=users,dc=example,dc=com"
    #         attributes:
    #            uid: "cn"
    #            mail: "email"
    #            name: "givenName"
    #         #bind_dn:
    #         #bind_password:
    #         #filter: "(objectClass=posixAccount)"
    
    
    
    # Clients requesting push notifications can either have the body of
    # the message sent in the notification poke along with other details
    # like the sender, or just the event ID and room ID (`event_id_only`).
    # If clients choose the former, this option controls whether the
    # notification request includes the content of the event (other details
    # like the sender are still included). For `event_id_only` push, it
    # has no effect.
    
    # For modern android devices the notification content will still appear
    # because it is loaded by the app. iPhone, however will send a
    # notification saying only that a message arrived and who it came from.
    #
    #push:
    #   include_content: true
    
    
    # spam_checker:
    #     module: "my_custom_project.SuperSpamChecker"
    #     config:
    #         example_option: 'things'
    
    
    # Whether to allow non server admins to create groups on this server
    enable_group_creation: false
    
    # If enabled, non server admins can only create groups with local parts
    # starting with this prefix
    # group_creation_prefix: "unofficial/"
    
    
    
    # User Directory configuration
    #
    # 'search_all_users' defines whether to search all users visible to your HS
    # when searching the user directory, rather than limiting to users visible
    # in public rooms.  Defaults to false.  If you set it True, you'll have to run
    # UPDATE user_directory_stream_pos SET stream_id = NULL;
    # on your database to tell it to rebuild the user_directory search indexes.
    #
    #user_directory:
    #   search_all_users: false
\end{lstlisting}

\anhang{Interviewleitfaden}

\begin{longtable}{p{12cm}|p{3cm}}
  \hline
  \textbf{1. Fragen zur IT Infrastruktur des Krankenhauses}                                                                                                  \\ \hline
  \endfirsthead
  \textbf{Fragen}                                                                                                                                           &   \textbf{Antworten}   \\ \hline
(1) Wird die gesamte IT des Krankenhauses über das Krankenhaus eigene Rechenzentrum betrieben?  Wenn Ja, ist dies der Regelfall für Krankenhäuser? &     \\ \hline
(2) Verfügt das Krankenhauspersonal über Diensthandys/ Dienstcomputer o. a. digitale Kommunikationsgeräte?                                         &          \\ \hline
(3) Ist im gesamten Krankenhaus ein Internetzugriff über WLAN/ mobiles Netz möglich?                                                               &       \\ \hline
(4) Gibt es ein krankenhausinternes, digitales Nutzerverzeichnis mit Unterstützung für eine Schnittstelle wie SAML, OIDC, etc.?                   &          \\ \hline
  \hline
  \textbf{2. Fragen zur digitalen Verarbeitung und dem Austausch von Patientendaten}                                                                                                  \\ \hline
  \textbf{Fragen}                                                                                                                                           &   \textbf{Antworten}   \\ \hline
(1) Welche Rechte überträgt der Patient an das Krankenhaus zur Verarbeitung seiner Daten? &     \\ \hline
(2) Welche Regelungen gelten beim Austausch von Patientendaten innerhalb des Krankenhauses?                                       &          \\ \hline
(3) Welche Regelungen gelten beim Austausch von Patientendaten mit anderen Krankenhäusern?                                                             &       \\ \hline
(4) Ist das Krankenhaus zur Datenhaltung der Patientendaten auf eigenen Servern verpflichtet?                  &          \\ \hline
\hline
\textbf{3. Fragen zur digitalen Kommunikation in Krankenhäusern}                                                                                                  \\ \hline
  \textbf{Fragen}                                                                                                                                           &   \textbf{Antworten}   \\ \hline
(1) Welches sind die primären digitalen Kommunikationsmethoden innerhalb des Krankenhauses?  &     \\ \hline
(2)	Welche digitalen Kommunikationsmethoden werden genutzt, wenn Gesundheitsdaten verschickt werden müssen?                                       &          \\ \hline
(3)	Sehen Sie Verbesserungspotential, wenn es um die digitale Kommunikation innerhalb des Krankenhauses geht? Wenn ja, können Sie Beispiele nennen?                                                            &       \\ \hline
(4)	Sehen Sie einen Bedarf für Krankenhäuser, neben den bereits bestehenden, digitalen Kommunikationsmethoden, einen Slack/ WhatsApp ähnlichen Dienst zu verwenden?                                                          &       \\ \hline
(5)	Wo sehen Sie die größten Hürden bei der Einführung einer solchen Kommunikationsplattform im Krankenhaus?                &          \\ \hline
\caption{Interviewleitfaden} 
\end{longtable}

\anhang{Experteninterview Georg Woditsch}

Experteninterview
\begin{itemize}
  \item Datum: 19.10.2020
  \item Ort: Telefoninterview
\end{itemize}

Experte
\begin{itemize}
  \item Name: Georg Woditsch 
  \item Position / Arbeit: Leitung Klinische Systeme an der Uni Klinik Münster
\end{itemize}

\textbf{Frage: }Wird die gesamte IT des Krankenhauses über das Krankenhaus eigene Rechenzentrum betrieben?  Wenn Ja, ist dies der Regelfall für Krankenhäuser?

\textbf{Georg Woditsch:}Das ist in der Tat absolut State of the Art. Ich will nicht behaupten, dass das in drei bis fünf Jahren noch so ist. Aber aktuell ist es tatsächlich so, dass alle im Krankenhaus erzeugten Daten und auch die Befunde wo das Krankenhaus die Aufbewarungspflicht hat in der Regel im Krankenhaus selber verbleiben, auch technisch im Krankenhaus verbleiben.
Denn wenn man kann  Daten auslagern will, zum Beispiel in ein großes Microsoft oder Amazon Cloud Center, also eine abgesicherte Cloud, dann hat man immer die Problematik, dass irgendeiner der amerikanischen Dienstleister irgendwo in Unterverträgen seine Finger im Spiel hat. Und durch die Auflösung des Patriot Act sagen die alle da hätte ein Amerikaner Zugriff auf die Patientenakte und kann die unverschlüsselt an einer bestimmten Stelle sehen. Das wollen wir nicht. Ich halte die Problematik nicht so groß, aber so ist es einfach. Das ist so die Grundhaltung. Es gibt allerdings eine große Ausnahme. Da sind wir selbst nicht von betroffen.

Das ist die Firma Teletax. Das ist ein großer Cloud Packs Dienstleister oder Archivierungsdienstleister für radiologische Bilddaten. Der das tatsächlich in einem eigenen gehosteten Rechenzentrum zur Verfügung stellt. Und da gibt es einige gesundheitliche Einrichtung, die ihre Daten dort speichern. Ansonsten sind Gesundheitsdaten in ihrer Vollständigkeit immer im Krankenhaus gespeichert.

\textbf{Frage: }Auch wenn die Gesetzeslage so geschrieben ist, dass es dem Krankenhaus sehr ans Herz gelegt wird, die Daten selber zu hosten, wissen Sie, ob es etwas gibt, was direkt dagegen spricht, Gesundheitsdaten verschlüsselt beim Cloud Betreiber zu zu hosten oder so abzulegen?

\textbf{Georg Woditsch: }Also wenn ich ehrlich bin, dann weiß ich zumindest in Nordrhein-Westfalen nicht, was offiziell rechtlich dagegen sprechen sollte. Dieses Beispiel, was man da anführen kann, ist tatsächlich die Asklepios-Kliniken-Gruppe , für welchen ich 10 Jahre gearbeitet habe.

Deshalb kann ich das auch sagen. Die machen auch kein Geheimnis draus, weil da ein Großteil der Daten in einem Rechenzentrum in Hamburg liegt. Allerdings gehört das Rechenzentrum dem Asklepios Konzern. In dem Zuge haben wir aber mit den Landes Datenschutz Ämtern verschiedene Bundesländer in Deutschland intensiv gesprochen. Im Grunde genommen gab es irgendwo einen wirklich rechtlichen Verweis darauf, dass Daten nicht ausgelagert werden dürfen. Es gibt einen Zusammenhang, der ist aber meines Wissens juristisch mittlerweile aufgelöst und zwar ist das der Paragraf 213 aus dem Strafgesetzbuch, da geht es umd as Berufsgeheimnis des Arztes, dass er seine Daten nicht in Hände legen darf von sozusagen unbeteiligten oder nicht an seinem Prozess Beteiligten. So ganz, ganz grob steht das da drin. Zu dem Paragraphen gab es jahrelang keine Rechtsprechung, was dazu geführt hat, dass er so ausgelegt wurde, dass ein Mitarbeiter eines Rechenzentrum keinen Zugriff auf die Daten erlangen darf, weil es ansonsten so ausgelegt wurde, dass er sich persönlich haftbar mache. Deswegen hat man das an vielen Stellen umspielt. Das ist der einzige juristische Zusammenhang und es gibt einen zweiten. Der ist aber rein auf das Land Bayern bezogen. Das ist, glaube ich der Paragraph 28 im Krankenhausgesetz Bayern. Da steht nämlich drin, dass Gesundheitsdaten eines Krankenhauses in Bayern nur in einem anderen Krankenhaus gelagert werden dürfen. Heißt also das Rechenzentrum, wo ich da bin, muss das Rechenzentrum eines krankenhauses sein.

\textbf{Frage: } Verfügt das gesamte Krankenhauspersonal über Diensthandys,  Dienst Laptops oder allgemein digitale Kommunikationsgeräte?

\textbf{Georg Woditsch: } Also fürs UKM gesprochen? Nein. Hier haben gerade mal die Ärzte ein Telefon in der Tasche. Und nur etwa 10 Prozent der Ärzte haben überhaupt ein Smartphone.
Ansonsten gibt es für das UKM und einen Großteil der Krankenhäuser, die mir bekannt sind keine Mobile Device Pflicht. Wobei die Wahlstrategie es ist, jedem Mitarbeiter die Möglichkeit zu geben. Das gleiche gilt für Laptops etc.. Es gibt für verschiedene Stationen, verschiedene Funktionsbereiche mobile Endgeräte, sowohl Laptops, als auch mal ein Tablet oder auch mal Dienstgruppenhandy. Aber nicht Flächenübergreifend.

Das ist auch eines der Hauptprobleme, warum ich glaubte, dass ein Messenger lange, lange, lange ohne Erfolg war. Denn dafür müsste jeder daran teilnehmen können. Wir müssten also im Krankenhaus jedem Mitarbeiter den Zugriff darauf sicherstellen können. Das können wir momentan nicht. Wenn wir das hochskalieren haben wir an der Uniklinik Münster 10 000 Mitarbeiter. Das heißt wir müssten 10 000 Endgerät bereitstellt und damit ist es ja nicht getan. Der zweite Teil ist, die Nutzerverwaltung und die Zugriffsvergabe. Es gibt ja die SAP Systeme, Buchungssystem, das Krankenhaus Informationssystem, einzelne kleine Systeme, z.B. zur Zeiterfassung. Was viele gemeinsam haben ist ein privates Telefon, da haben die Whatsapp drauf,so plakativ wie es ist. Und da hat auch jeder seine Gruppen, weil das alles nur Sinn macht, wenn alle daran teilnehmen. In dem Moment, wo ein oder zwei an einer Gruppe nicht teilnehmen nützt die Gruppe nichts. Da haben sie dann einen Informationsverlust. Und das ist exakt der Punkt, warum ich glaube, dass es gerade in der Uni Klinik unglaublich schwer wird, das umzusetzen. Man kann bestimmt Prozesse damit nach vorne bringen. Man müsste aber den Schritt gehen Endgeräte zu kaufen, die alle drei Jahre neu gekauft werden.

\textbf{Frage: } Würden Sie sagen, dass es eine Bewegung in die Richtung geben wird?

\textbf{Georg Woditsch: } Sagen wir so, ich bin der festen Überzeugung, dass das zwingend notwendig ist. Und bin auch ein Kämpfer dafür. Ich glaube, dass wir mehr Flexibilisierung an der Stelle brauchen. Ich glaube, man wird so vorgehen müssen, das man Geräte an einzelnen Stellen ausgibt und misst, ob ein Zeitersparniss festgestellt wird und man auf Basis dieser Erfahrungen dann weiter ausbauen kann. Man wird immer Mitarbeiter dabei haben, die nicht mit dem Smartphone umgehen wollen, oder die nicht umgehen können, die telefonieren wollen, die tippen wollen, die diktieren wollen. Da gibt's ganz viel. Aber ich glaube, dass sich das nach und nach ergibt, denn die Generation die jetzt sozusagen nachrückt, ist mobiles Arbeiten gewöhnt. Wir haben heute eigentlich wöchentlich Anfragen von Ärzten. Zum Beispiel fragen die: Warum kann ich mein Protokol nicht einfach kurz in mein Handy reintippen, warum muss ich für die Protokolle umständliche Software auf meinem PC benutzenl.

\textbf{Frage: }Da haben Sie schon die perfekte Überleitung zum nächsten Punkt, den ich auf meiner Liste habe:  WLAN / Mobiles Netz im Krankenhaus. Netzwerk Abdeckung ist natürlich eine Grundlage, damit so ein Messenger funktioniert. Wenn man nur in der Hälfte der Räume überhaupt eine Netzwerkverbindung hat, ist das ganze System hinfällig.

\textbf{Georg Woditsch: } Die Lösung wird, denke ich, eine Kombination aus Mobilen Daten,gerade mit 5G und Wlan sein. Ich glaube, dadurch wird es möglich sein, dass man eigene Schwächen im eigenen Wifi Netzwerk findet und darüber kompensieren kann. Ich glaube aber auch nicht, dass es unbedingt notwendig ist, dass in jedem Raum, in jeder Ecke voller Empfang ist, um eine vernünftige Kommunikation zu ermöglichen. Das ist auf jeden Fall ein Theama was in Zukunft noch sehr ausgebaut wird.

\textbf{Frage: } In der nächsten Frage geht darum, wie die Nutzer im Messenger registriert werden. Gibt es ein digitales Krankenhaus Nutzern Verzeichnis, was eben als Schnittstelle genutzt werden könnte für die Authentifikation gegenüber anderer Systeme und Schnittstellen wie wie SAML oder OIDC bereit stellt?

\textbf{Georg Woditsch: } Ja, es gibt eine zentrale Nutzerverwaltung, die wir auch zur authentifikation Nutzen. Wir haben ein mehr schlecht als recht aufgebautes Active Directory. Krankenhausintern werden sie damit also keine Probleme haben. Es bleibt aber eine händische Aufgabe diese zu pflegen. Das Problem ist die Kommunikation in das private Leben der Menschen. Wenn sie zum Beispiel an einem Chat mit ihrer dienstlichen Nummer Teilnehmen. Dann wird der Name der Person angezeigt werden. Das muss dann auch noch mit dem Betriebsrat abgestimmt werden, was auch noch mal eine riesige Hürde ist.

\textbf{Frage: } Welche Rechte übergibt der Patient an das Krankenhaus zur Verarbeitung seiner Daten?

\textbf{Georg Woditsch: } Ah ja, das ist von Krankenhaus zu Krankenhaus unterschiedlich, aber die meisten Kliniken haben eine Form von AGBs. Der Patient unterschreibt, dass er mit der Verarbeitung und Weitergabe seiner Daten einverstanden ist. Das heißt für den Kontext der Klinik: Sie müssen nur sicher stellen, dass nur die Personen Zugriff auf die Daten haben, die auch wirklich darauf zugreiffen dürfen. Innheralb des Krankenhauses gibt es aber ansonsten keine Einschränkungen.

In dem Moment, wo sie sich mit anderen Kliniken austauschen dürfen sie ein Konsil stellen. Das heißt Sie können einen explizieten Arzt oder eine Gruppe fragen: Wir haben einen Patienten mit folgenden Symthomen, könnten Sie dazu einmal Feedback geben? Was sie nicht dürfen ist die Daten einfach an Dritte weitergeben. So nach dem Motto, hier liebe Klinik in Göttingen, hier sind die Patientendaten.  Das geht dann eben nicht, aber der Patient unterschreibt in der Regel am Anfang, dass seine Daten bei Bedarf elektronisch bearbeitet werden dürfen und solange diese in der Klinik bleiben ist alles gut. In dem Moment, wo die wieder extern gespeichert würden, ist das wieder etwas anderes. Dann müssten sie gucken, was wird genau extern gespeichert. Das ist auch  der Grund, warum die Kliniken gesagt haben, dass man einen Messenger braucht der die Daten nicht automatisch extern speichert, sondern maximal die Kommunikation zwischen zwei Anwendern. Die Daten müssen im Krankenhaus bleiben. Ein Röntgenbild mit dem Namen des Patienten dürfe also nicht einfach nach außen geschickt werden.

\textbf{Frage: }  Was ist der primäre Weg der digitalen Kommunikation im Krankenhaus? Ist es E-mail oder andere Services?

\textbf{Georg Woditsch: } Also im Kontext der Pfleger ist es das Krankenhaus-Informationssystem, wo die Daten elektronisch gespeichert, verarbeitet und den richtigen Nutzergruppen zur Verfügung gestellt werden? Dafür ist das Krankenhausinformationssystem gedacht. 90 Prozent der elektonischen Dokumente befinden sich im Krankenhausinformationssystem. Für die oragnisatorischen Prozesse gibt es aber eine Million Systeme.

Da gibt wirklich alles, da gibt es diejenigen die per Chat hin und her schreiben, diejenigen, die mit Microsoft Teams arbeiten. Da gibt es Email, was ich mittlerweile sehr ablehne, es ist ein überholtes Medium für die interne Kommunikation, Zeitaufwendig, trifft nie den richtigen. Für Ärtze ist das Telefonieren auch sehr wichtig, dass darf man nicht vergessen, da man sich so schnell austauschen kann. Die haben einen Patienten vor sich und dann haben die eine Frage und dann wollen sie die kurz mit Kollegen klären und wollen nicht lange rum tippen. Die Telfonie ist eine ganz, ganz wichtige Kommunikation. Da wäre aber auch ein Messenger gut, über den man sehen kann, wer online ist und dem dann kurz "Zeit?" schreiben kann, um dann zu telefonieren.

\textbf{Frage: } Dann denke ich, dass Videotelefonie auch ein spannedes Thema sein kann, oder?

\textbf{Georg Woditsch: } Ja, definitiv.

\textbf{Frage: } Sie hatten ja vorhin gesagt, dass es für die organisatorische Kommunikation eine Million unterschiedliche Möglichkeiten gibt. Sie sehen also durchaus einen Bedarf danach, das zu zentralisieren und einen Dienst vorzugeben? Also eben durch eine Plattform, die allgemein genutzt wird.

\textbf{Georg Woditsch: } Jaein, als ein kleines Krankenhaus mit 200 300 Betten kann man sich auf 1,2 Systeme reduzieren.In einem komplexen Laden wie der Uniklinik ist das nicht möglich, weil einfach viel zu viele unterschiedliche Gruppen da sind, die sie abdecken müssten. Die beiden Sachen, die da helfen, sind zum einen Spielregeln haben, die oft nicht eingehalten werden, weil jeder seinen Prozess selber regelt. Und das andere sind Hilfsmittel, die sozusagen von sich aus überzeugen. Die Leute suchen sich immer den bequemsten Weg zur Kommunikation, mit wem sie auch kommunizieren, wo ihnen am schnellsten geholfen wird, weil Ärzte, wie gesagt, daruf eingestellt sind, in der Regel Patienten zu behandeln und Pfleger in der Regel auch totalen Prozessdruck haben. Aus dem Grund ist es so, dass sich glaube, dass sich nur durchsetzen wird, was für alle Beteiligten am schnellsten und am einfachsten geht, wo alle mitarbeiten dürfen.

Wir haben momentan viele verschiedene Systeme: Microsoft Teams, Webex. Wir haben die Anforderungen auf Collaborations Plattformen zu arbeiten, wir haben die Anforderungen in Systemen zu dokumentieren. Ich glaube, dass sich als Standart für die Kommunikation das durchsetzt, was am zugänglichsten ist. Es muss immer noch möglich sein, alle Prozesse abzubilden.

Wenn das Produkt gut genug ist und leicht genug zu handhaben ist, dann wird sich das auch durchsetzen. Dann werden die Leute nach und nach damit arbeiten und dann werden sie versuchen, weil die Menschen ja bequem sind, mehr als einen Prozess darüber abzubilden, z.B die medizinische Dokumentation.

Also muss das erste Ziel nicht unbedingt ein gewaltiger Funktionsumfang sein, sondern wie setzte ich  die Einstiegshürde möglichst gering, um die Anwender davon zu überzeugen, dass sie mit meinem Produkt arbeiten wollen.

Weil ich glaube, wir machen uns heute Gedanken darüber, was man noch alles an Funktionen und Systemen anbinden kann, aber wir wissen ja gar nicht, wie sieht die Prozesse aussehen, wenn wir erst einmal einen Teil mobilisiert und digitalisiert haben. Da kommen ganz neue Anforderungen. 

Aber ich glaube, das wird noch lange brauchen. Seit zwei Jahren diskutieren wir da rum. Ich bin von der Notwendigkeit eines solchen Dinestes aber vollkommen überzeugt.