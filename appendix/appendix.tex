\chapter*{Anhang}
\addcontentsline{toc}{chapter}{Anhang}

\lstset{language=TeX,
    morekeywords={anhang, anhangteil}
}

% Definition des Anhangverzeichnis
\section*{Anhangverzeichnis}
\vspace{-8em}
\abstaendeanhangverzeichnis
\listofanhang
\clearpage

% Konfiguration der speziellen Kopfzeile für den Anhang
\spezialkopfzeile{Anhang}

% Hauptteil des Anhangs
% Mit \anhang{Abschnitt des Anhangs} fügt man ein Kapitel in dem Anhang hinzu z. B. Interview Transkripte
\anhang{Interview Transkripte}

\begin{longtable}{p{1cm}|p{4cm}|p{2cm}|p{5cm}}
    \hline
    Nr. & Anforderung                                                                                                                                                                                                                                                                                          & Relevants          & Einordnung für den Messanger                                                                                                                                                                                                                                                                                                                                                                                                                                                                                                                                        \\ \hline
    1      & „Jegliche Erhebung, Verarbeitung und Nutzung personenbezogener oder personenbeziehbarer Daten - insbesondere durch den Einsatz von Datenaustauschplattformen, die an das Internet angebunden sind - bedarf einer datenschutzrechtlich wirksamen Einwilligung des Betroffenen.“                       & relevant           & Anforderung 1 und 3 sind relevant, da sie für den Messanger bedeuten, dass dem Krankenhauspersonal die Möglichkeit gegeben werden muss, der Verarbeitung ihrer Nutzerdaten durch die Plattform zuzustimmen. Das gleiche gilt auch für die Patienten, deren Einwilligung über die für das Krankenhaus gültigen AGBs eingeholt werden muss, welche auch die allgemeine Verarbeitung der Patientendaten inerhalb des Krankenhauses beschreiben.                                                                                                                        \\ \hline
    3      & „Eine Einwilligung muss für den Betroffenen jederzeit mit Wirkung für die Zukunft widerrufbar sein.“                                                                                                                                                                                                 & relevant           & Anforderung 1 und 3 sind relevant, da sie für den Messanger bedeuten, dass dem Krankenhauspersonal die Möglichkeit gegeben werden muss, der Verarbeitung ihrer Nutzerdaten durch die Plattform zuzustimmen. Das gleiche gilt auch für die Patienten, deren Einwilligung über die für das Krankenhaus gültigen AGBs eingeholt werden muss, welche auch die allgemeine Verarbeitung der Patientendaten inerhalb des Krankenhauses beschreiben.                                                                                                                        \\ \hline
    4      & „Jede Datenaustauschplattform muss Datenschutzhinweise veröffentlichen und darin die getroffenen Datensicherheitslösungen allgemeinverständlich beschreiben.“                                                                                                                                        & relevant           & Der Messanger muss die Möglichkeit bieten, die Datenschutzhinweise Anforderung 4-8 entsprechend den Nutzern zur Verfügung zu stellen. Die eigentliche Formulierung der Datenschutzhinwiese ist nicht Teil dieser Arbeit, die technische Umsetzung, um diese auf die geforderte Weise zur Verfügung zu stellen jedoch schon.                                                                                                                                                                                                                                         \\ \hline
    5      & „Datenschutzhinweise müssen unmittelbar von der Startseite der Datenaustauschplattform aus aufrufbar bzw. erreichbar sein.“                                                                                                                                                                          & relevant           &                                                                                                                                                                                                                                                                                                                                                                                                                                                                                                                                                                     \\ \hline
    6      & „Datenschutzhinweise müssen für den Betroffenen jederzeit abrufbar sein.“                                                                                                                                                                                                                            & relevant           &                                                                                                                                                                                                                                                                                                                                                                                                                                                                                                                                                                     \\ \hline
    7      & „Bei nachträglicher Änderung ist der Betroffene zu informieren und sein Einverständnis erneut einzuholen.“                                                                                                                                                                                           & relevant           &                                                                                                                                                                                                                                                                                                                                                                                                                                                                                                                                                                     \\ \hline
    8      & „Wird die Einwilligung des Betroffenen elektronisch eingeholt, so muss der Vorgang protokolliert werden und der Inhalt der Einwilligung für den Betroffenen jederzeit abrufbar sein.“                                                                                                                & relevant           &                                                                                                                                                                                                                                                                                                                                                                                                                                                                                                                                                                     \\ \hline
    9      & „Personenbezogene oder personenbeziehbare Daten müssen pseudonymisiert werden, soweit dies nach dem Verwendungszweck möglich ist und keinen im Verhältnis zu dem angestrebten Schutzzweck unverhältnismäßigen Aufwand erfordert.“                                                                    & relevant           & Der Messanger Dienst sollte den Nutzern aufgrund dieser Anforderung funktionen zur Verfügung stellen, Daten wie zum Beispiel Bilddateien, die über den Dienst geteilt werden sollen auf eine einfache Art und Weise zu pseudonymisieren.                                                                                                                                                                                                                                                                                                                            \\ \hline
    13     & „Der Betroffene muss der Nutzung von Cookies explizit zustimmen.“                                                                                                                                                                                                                                    & teilweise relevant & Die gesetzliche Anforderungen 13-15 an den Umgang mit Cookies ist vorallem für die Umsetzung der im Funktionsumfang festgehaltenen Anforderung des Webzugangs zum Messanger Dienst relevant. Der Messanger Dienst muss dementsprechend technisch die Möglichkeit bieten, über mögliche Cookies zu informieren und dem Nutzer die Zustimmung zur Nutzung von Cookies ermöglichen.                                                                                                                                                                                    \\ \hline
    14     & „Die Verwendung von Cookies durch Drittanbieter muss auf die Erstellung anonymer Nutzungsstatistiken beschränkt sein.“                                                                                                                                                                               & teilweise relevant &                                                                                                                                                                                                                                                                                                                                                                                                                                                                                                                                                                     \\ \hline
    15     & „Das Handling der Drittanbieter-Cookies muss durch den Betroffenen steuer- und nachvollziehbar sein.“                                                                                                                                                                                                & teilweise relevant &                                                                                                                                                                                                                                                                                                                                                                                                                                                                                                                                                                     \\ \hline
    18     & „Die Erhebung, Verarbeitung und Nutzung von Daten zur Erstellung von Nutzungsstatistiken von Webportalen muss dokumentiert werden.“                                                                                                                                                                  & teilweise relevant & Aus Gründen der Privatsphäre ist von einem individuellen Tracken des Nutzerverhaltens auf der Plattform ab zu sehen, da es sich hierbei nicht um eine relevante Funktion für den Einsatz des Messanger Dienstes handelt. Bei der nicht personenbezogenen Auswertungen des Nutzerverhaltens, zum Beispiel zur Überprüfung der Annahme des Dienstes seitens der Nutzer, sind Anforderungen 18-20 zu beachten. Allgemein fällt die Analyse nicht in die Zielsetzung der Arbeit und wird deshalb im Folgenden nicht weiter besprochen.                                  \\ \hline
    19     & „Die Erstellung von Nutzungsstatistiken von Datenaustauschplattformen muss auf Basis anonymisierter Daten erfolgen.“                                                                                                                                                                                 & teilweise relevant &                                                                                                                                                                                                                                                                                                                                                                                                                                                                                                                                                                     \\ \hline
    20     & „Eine Analyse des Nutzerverhaltens (einschließlich Geolokalisierung) auf der Basis gekürzter IP-Adressen ist ohne Einwilligung des Betroffenen möglich, wenn durch die Kürzung der IP-Adresse ein Personenbezug ausgeschlossen wurde.“                                                               & teilweise relevant &                                                                                                                                                                                                                                                                                                                                                                                                                                                                                                                                                                     \\ \hline
    31     & „Alle Personen, welche auf die gespeicherten Gesundheitsdaten zugreifen können, müssen vor dem erstmaligen Zugriff auf die Wahrung des Datengeheimnisses verpflichtet worden sein.“                                                                                                                  & relevant           & Für den Messanger Dienst muss eine technische Lösung implementiert werden, welche es erlaubt, dem Nutzer eine solche Wahrung zu präsentieren.                                                                                                                                                                                                                                                                                                                                                                                                                       \\ \hline
    32     & „Ein anonymer Zugriff des Plattformbetreibers sowie der Nutzer der Plattform auf personenbezogene oder personenbeziehbare Gesundheitsdaten ist zu unterbinden.“                                                                                                                                      & relevant           & Es muss technisch sichergestellt werden, dass sämtliche Zugriffe auf den Dienst zu jeder Zeit einer realen Person zugeordnet werden können. Dies gilt auch für den direkten Zugriff auf die mit dem Dienst verbundenen Datenbanken durch Administratoren.                                                                                                                                                                                                                                                                                                           \\ \hline
    33     & „Daten über den Ablauf des Zugriffs oder der sonstigen Nutzung sind unmittelbar nach dem Zugriff bzw. ggfs. nach erfolgter Abrechnung der Dienstnutzung zu sperren und nur zu entsperren, wenn eine gesetzliche Bestimmung (z.B. Auskunftsersuchen eines Betroffenen) dies erlaubt.“                 & relevant           & Der Messanger muss es erlauben, Daten zeitlich begrenzt zu Teilen, damit sichergestellt werden kann, dass  Daten nur für die Dauer ihres benötigten Verwendungszwecks über den Messenger abrufbar sind.                                                                                                                                                                                                                                                                                                                                                             \\ \hline
    34     & „Die Nutzung von Telemedien muss gegen die Kenntnisnahme unberechtigter Dritter geschützt werden.“                                                                                                                                                                                                   & relevant           & Eine Anmeldung und Verifizierung des Nutzers muss vor dem Zugriff auf den Messanger erfolgen. Ebenso muss der Zugriff auf den dem Messanger zugeordnete Software Code und entsprechenden Datenbanken geschützt werden. Dies muss durch eine technische Lösung sichergestellt werden.                                                                                                                                                                                                                                                                                \\ \hline
    39     & „Ereignisse, die potenziell dazu führen können, dass ein unberechtigter Zugriff auf personenbezogene oder personenbeziehbare Daten erfolgen könnte, sind zu protokollieren.“                                                                                                                         & relevant           & Es muss sichergestellt werden, dass mögliche, unberechtigte Zugriffe in den Logs des Dienstes festgehalten werden und entsprechend archiviert werden.                                                                                                                                                                                                                                                                                                                                                                                                               \\ \hline
    41     & „Der Betroffene muss die Möglichkeit haben, jederzeit Einblick in alle zu seiner Person gespeicherten Daten zu erhalten. Dies umfasst auch die Möglichkeit, Änderungen seiner gespeicherten Daten nachzuvollziehen.“                                                                                 & relevant           & Anforderung 41-43 betreffen im Kontext des Messangers nur die Daten der eigentlichen Nutzer, da keine langfristige Speicherung der Gesundheitsdaten über den Messanger vorgesehen ist, die geforderten Möglichkeiten im Bezug auf Patientendaten müssen bei den Systemen vorgenommen werden, welche zur langfristigen Speicherung der Patientendaten vorgesehen sind. Was die eigentlichen Nutzerdaten betrifft sind technische Lösungen zu implementieren, welche die Umsetzung der beschriebenen Anforderungen erlaubt. \\ \hline
    42     & „Der Betroffene muss die Möglichkeit haben, einen Ausdruck oder einen für ihn verwertbaren Export aller zu seiner Person gespeicherten Daten zu erhalten.“                                                                                                                                           & relevant           &                                                                                                                                                                                                                                                                                                                                                                                                                                                                                                                                                                     \\ \hline
    43     & „Es muss eine Möglichkeit geben, auf Aufforderung des Betroffenen Daten zu seiner Person zu korrigieren.“                                                                                                                                                                                            & relevant           &                                                                                                                                                                                                                                                                                                                                                                                                                                                                                                                                                                     \\ \hline
    46     & „Es muss eine Löschfunktion implementiert werden, welche eine Rekonstruktion gelöschter Informationen ausschließt.“                                                                                                                                                                                  & relevant           & Eine permanente Löschung betroffener Daten muss über den Messanger technisch möglich sein.                                                                                                                                                                                                                                                                                                                                                                                                                                                                          \\ \hline
    58     & „Da im Internet eine potenziell größere Gefährdung für einen unbefugten Zugriff auf personenbezogene Daten existiert, muss mindestens eine 2-Faktor-Authentifizierung erfolgen.“                                                                                                                     & relevant           & Die Möglichkeit einer 2 Faktor Authentifikation muss in den Messanger Dienst implementiert werden.                                                                                                                                                                                                                                                                                                                                                                                                                                                                  \\ \hline
    59     & „Werden statische Passwörter zur Authentisierung eingesetzt, so müssen die Empfehlungen des BSI bzgl. der Generierung und des Umgangs eingehalten werden. D.h. es müssen technische und organisatorische Maßnahmen zu der Einhaltung der Empfehlungen des BSI getroffen werden.“                     & relevant           & Die Mindestanforderungen an die Komplexität der für die Nutzung des Messangers zu erstellenden Passwörtern muss durch eine technische Lösung festlegbar sein.                                                                                                                                                                                                                                                                                                                                                                                                       \\ \hline
    60     & „Authentisierungsgeheimnisse dürfen nur gesichert in Netzwerken übertragen werden, d.h. es muss eine verschlüsselte Datenübertragung entsprechend dem Stand der Technik eingesetzt werden.“                                                                                                          & relevant           & Es muss technisch sicher gestellt werden, dass nur über eine gesicherte Netzwerkverbindung mit dem Dienst kommuniziert werden kann.                                                                                                                                                                                                                                                                                                                                                                                                                                 \\ \hline
    62     & „Nach wiederholter fehlerhafter Authentisierung muss der Zugang gesperrt werden.“                                                                                                                                                                                                                    & relevant           & Eine technische Möglichkeit muss implementiert werden, welche es erlaubt, nach zu häufiger, fehlerhafter Authentifikation betroffene Zugänge zu sperren.                                                                                                                                                                                                                                                                                                                                                                                                            \\ \hline
    63     & „Die Sitzung muss gesperrt oder beendet werden, wenn der Anwender eine definierte Zeitspanne in der Sitzung keine Aktivitäten durchführte (Sitzungs-Zeitlimit, Session Timeout).“                                                                                                                    & relevant           & Eine technische Möglichkeit muss implementiert werden, welche es erlaubt, eine Zeitspanne zu definieren, nach welcher eine erneute Authentifizierung vom Nutzer gegenüber des Messangers notwendig ist. Dies betrifft vor allem den Zugriff auf das Webinterface des Messangers.                                                                                                                                                                                                                                                                                    \\ \hline
    64     & „Ein Prozess zur Rücksetzung bzw. Entsperrung von gesperrten Zugangskennungen ist einzurichten, zu beschreiben und anzuwenden.“                                                                                                                                                                      & relevant           & Eine technische Möglichkeit muss implementiert werden, welche die Umsetzung von Anforderung 64 ermöglicht.                                                                                                                                                                                                                                                                                                                                                                                                                                                          \\ \hline
    66     & „Benutzerkennungen, welche über einen definierten Zeitraum nicht benutzt wurden, sind zu sperren bzw. auf inaktiv zu setzen.“                                                                                                                                                                        & relevant           & Eine technische Möglichkeit muss implementiert werden, welche die Umsetzung von Anforderung 64 ermöglicht.                                                                                                                                                                                                                                                                                                                                                                                                                                                          \\ \hline
    71     & „Eine Kombination von Rollen bzw. Zugriffsrechten für eine Person, welche der Person mehr Rechte auf Datenzugriffe erteilt, als für ihre Aufgabe nötig ist, ist zu verhindern. Es sind technische und organisatorische Maßnahmen zu ergreifen, um dies sicherzustellen.“                             & relevant           & Für den Messanger muss ein Anforderung 71 entsprechendes Rollenkonzept technisch umgesetzt werden.                                                                                                                                                                                                                                                                                                                                                                                                                                                                  \\ \hline
    73     & „Protokolldaten sind gegen unbefugten Zugriff in geeigneter Weise entsprechend dem Stand der Technik zu schützen.“                                                                                                                                                                                   & relevant           & Die vom Messanger erstellten Protokollanten sind dem Stand der Technik entsprechend zu schützen.                                                                                                                                                                                                                                                                                                                                                                                                                                                                    \\ \hline
    81     & „Die Übertragung personenbezogener oder personenbeziehbarer Gesundheitsdaten zwischen Clients und Servern wie auch zwischen Servern selbst muss entsprechend dem jeweiligen Stand der Technik generell verschlüsselt erfolgen.“                                                                      & relevant           & Eine Verschlüsselung wie sie in Anforderung 81 definiert ist muss für den Messanger implantiert werden.                                                                                                                                                                                                                                                                                                                                                                                                                                                             \\ \hline
    86     & „Gesundheitsdaten sind entsprechend dem Stand der Technik verschlüsselt in der Datenbank zu speichern, so dass bei administrativen Zugriffen Wartungspersonal keinen unbefugten Zugriff auf die gespeicherten Daten erhalten kann.“                                                                  & relevant           & Eine Verschlüsselung wie sie in Anforderung 86 definiert ist muss für den Messanger implantiert werden.                                                                                                                                                                                                                                                                                                                                                                                                                                                             \\ \hline
    87     & „Nur Personen, die laut Berechtigungskonzept zu einer Eingabe berechtigt sind, dürfen personenbezogene oder personenbeziehbare Daten in eine Datenaustauschplattform eingeben.“                                                                                                                      & relevant           & Für den Messanger muss ein Anforderung 87 und 88 entsprechendes Rollenkonzept technisch umgesetzt werden.                                                                                                                                                                                                                                                                                                                                                                                                                                                           \\ \hline
    88     & „Nur Personen, die laut Berechtigungskonzept zu einem Datenimport berechtigt sind, dürfen einen Import personenbezogener oder personenbeziehbarer Daten in eine Datenaustauschplattform durchführen oder veranlassen.“                                                                               & relevant           &                                                                                                                                                                                                                                                                                                                                                                                                                                                                                                                                                                     \\ \hline
    93     & „Es muss ein Backup-Konzept vorhanden sein, welches gewährleistet, dass die Daten nach einem Vorfall in angemessener Zeit wieder zur Verfügung gestellt werden können. In diesem Backup-Konzept muss berücksichtigt werden, dass nur berechtigte Personen Zugriff auf Backup-Daten erlangen können.“ & teilweise relevant & Die technische Möglichkeit der Erstellung regelmäßiger Backups muss gewehrleistet sein.                                                                                                                                                                                                                                                                                                                                                                                                                                                                             \\ \hline
\caption{longtable}    
\end{longtable}

    % Einfügen von weiteren Unterabschnitten, um den Anhang zu gliedern
    \input{appendix/interview_1.tex}
    \clearpage