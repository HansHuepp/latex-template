\chapter{Thematische Grundlagen}\label{chapter:hintergrund}
Um einen Einstiegspunkt in die Thematik der technische Datenschutz und Nutzeranforderungen an Messenger Dienste im Krankenhausbereich zu ermöglichen, müssen zunächst die einzelnen Anforderungssteller erörtert werden, sowie die Problemstellung mit herkömmlichen Messenger Systemen und der Ist Zustand innerhalb von Krankenhäusern im Bezug auf digitale Kommunikation.

\section{Übersicht der Anforderungssteller}\label{section:ueda}
Im folgenden Kapitel werden die Stakeholder, welche bei der Entwicklung eines Messenger-Dienstes im Krankenhausbereich zu beachten sind gelistet. Diese werden in späteren Kapiteln einzeln betrachtet und ihr Einfluss auf die Konzeption des Dienstes weiter analysiert und ausgewertet. Die Akteure: Hersteller einer Datenaustauschplattform, Betreiber einer Datenaustauschplattform, Datenlieferant, Nutzer einer Datenaustauschplattform und Betroffener bei Nutzung einer Datenaustauschplattform

Datenschutz-Grundverordnung (DSGVO): Die DSGVO ist eine Verordnung der Europäischen Union, welche die private wie auch die öffentliche Verarbeitung personenbezogener Daten EU-weit vereinheitlichend reglementiert. Dadurch soll einerseits der Schutz personenbezogener Daten innerhalb der Europäischen Union sichergestellt und andererseits der freie Datenverkehr innerhalb des Europäischen Binnenmarktes gewährleistet werden. Die Einhaltung der DSGVO Verordnungen ist eine der obersten Prioritäten bei der Entwicklung des Messenger Dienstes. Ein Verstoß wird mit erheblichen Bußgeldern geahndet. Da sich die DSGVO auf den Verarbeitung von personenbezogenen Daten in ihrer gesamtheit bezieht, sind nicht alle Regelungen gleichermaßen relevant für die Entwicklung des Messengers. Besonders zu beachtende Regelungen sind "das Recht auf Löschung", "Das Recht auf Einverständis" und "das Recht auf Datenübertragung". Alle relevanten DSGVO Anforderungen werden in Kapitel x genauer betrachtet und analysiert.

Länderrecht: Neben der europaweit geltenden DSGVO gibt es innerhalb der einzelnen Bundesländer zusätzliche Datenschutzbestimmungen, welche im jeweiligen Bundesrecht verankert sind und in einzelnen Fällen Einfluss auf die Verarbeitung von Gesundheitsdaten haben.

Die  Aufteilung  der  Zuständigkeiten zwischen  dem Bund  und  den  Ländern  werden  in  den  Artikeln  70  bis 75 Grundgesetz (GG) behandelt („Föderalismus“). Entsprechend Artikel 74 GG existiert im Bereich des    Gesundheitswesens    eine    konkurrierende    Gesetzgebung,    sodass    die    länderspezifische    Betrachtung      bei      einer      Datenerhebung      in      Deutschland      nicht      zu      umgehen      ist.      Datenaustauschplattformen  werden  häufig  von  ausgegründeten  Unternehmen,  die  als  GmbH  oder  gGmbH  arbeiten,  betrieben.  Für  diese  gilt  als  datenschutzrelevante  gesetzliche  Vorschrift  das  Bundesdatenschutzgesetz  (BDSG).  Gerade  für  Krankenhäuser,  die  als  wichtige  Akteure  bei  der  Bereitstellung   von   Gesundheitsdaten   anzusehen   sind,   gelten   landes-   und   kirchenrechtliche   Bestimmungen,   die   zum   Teil   sehr   detailliert   beschreiben,   was   mit   den Daten der   Patienten   geschehen  muss  oder  auch  darf.  Für  Sozialdaten  (Legaldefinition  siehe  §67  Abs.  1  SGB  X)  gilt  der  Sozialdatenschutz,  der  überwiegend  im  Zehnten  Buch  Sozialgesetzbuch  (Sozialverwaltungsverfahren  und Sozialdatenschutz SGB X6) beschrieben wird.

Krankenhäuser: Bei den Krankenhäusern handelt es sind logischerweise um einen der relevantesten Stakeholder, da sie diejenigen sind, welche einen solchen Dienst betreiben würden und damit der direkte Kunde des Messengers. Für Krankenhäuser sind neben der Sicherheit und Robustheit des Dienstes vorallem Punkte wie Hosting und Integration in die bestehende IT Infrastruktur des Krankenhauses relevant. 
Es ist zu erwarten, dass das selbstständige Hosten des Dienstes, aufgrund der sensibilität der über den Dienst transferierten Daten, für Krankenhäuser die bevorzugte Art und Weise des Betriebs ist. Zudem sollte es möglich sein bestehende Nutzerverzeichnisse und Nutzerverwaltungen in den Dienst zu integrieren.

Krankenhauspersonal: Das Krankenhauspersonal ist ebenso von enormer Wichtigkeit, da der Dienst nur dann Erfolg hat, wenn er vom Personal angenommen und genutzt wird und zu einer Verbesserung der krankenhausinternen Kommunikation beitragen kann. Aus diesem Grund spielen UI und UX eine entscheidende Rolle und Personal muss in die Entwicklung des Dienstes eingebunden werden.

Patienten: Damit der Messenger Dienst von Krankenhauspersonal genutzt werden darf, muss der Patient die mögliche Verarbeitung der Daten freigeben.

\section{Übersicht des Ist-Zustands im medizinischen Sektor}\label{section:ueda}
Das folgende Kapitel beschäftigt sich mit dem vorhandenen Angebot an Messenger Diensten, um zu überprüfen ob es nicht bereits für den gebrauch in Krankenhäusern passende Dienste auf dem Markt gibt. 

-Open Source , -Selbstgehostet, -End zu End Verschlüsselt, -Integrierbar in bestehende IT Infrastrktur (Login, Sollte auf Computern laufen), -Hoher grad der Individualisierbarkeit und Administration (Chat Löschung nach 48 Stunden, Geschlossene und Offene Gruppen)

Whatsapp: Whatsapp ist einer der populärsten Messenger Dienste Deutschlands. Der dem Facebook Konzern angehörige Dienst ist vor allem für den privaten gebrauch ausgelegt, bietet aber dennoch einen großen Funktionsumfang, mit der Integration von Audio und Video telefonaten, der Bedienung des Dienstes außerhalb der App über ein Web Interface und der Möglichkeit Nachrichten End zu End zu verschlüsseln.

//100 Punkte aus PDF auf relevnate Punkte kürzen. Dann Services in Tabelle vergleichen.

-Speicherung von Daten direkt auf dem Handy 
-Keine Administration möglich 
-An die Telefonnummer gebunden,
-Kein Selfhosting möglich 
-Online Speicherung der Daten intransparent 
-Programmcode nicht einsehbar
-Keine zwingende End zu End Verschlüsselung
-Eigentum eines amerikanischen Unternehmens

Slack 
-Kein Selfhosting möglich 
-Programmcode nicht einsehbar
-Keine zwingende End zu End Verschlüsselung
-Eigentum eines amerikanischen Unternehmens

Microsoft Teams
-Kein Selfhosting möglich 
-Programmcode nicht einsehbar
-Keine zwingende End zu End Verschlüsselung
-Eigentum eines amerikanischen Unternehmens

Siilo 
-Kein Selfhosting möglich 
-Programmcode nicht einsehbar

Die Tabelle zeigt, dass d

\section{Einschränkungen}\label{section:einschraenkungen}
Die Implementierung eines aus rechtlicher wie auch aus technischer Sicht so komplexen Systems wie eines Messenger Dienstes in die bestehende IT Infrastruktur eines Krankenhauses ist ein sehr umfangreicher Prozess. Zu der Integrierung des Dienstes zählen zudem auch Dinge wie die Erstellung von Onboarding Plänen für die Mitarbeiter des Krankenhauses, Notfall Konzepte für den Ausfall des Service oder der Aufbau eines Supports für den Messenger Dienst. Die komplette Abbildung der Entwicklung und Integrierung des Messenger Dienstes würde somit den Umfang dieser wissenschaftlichen Arbeit überschreiten. Aus diesem Grund fokussiert sich der Inhalt dieser Arbeit ausschließlich auf die technische Umsetzung der gesetzlichen Anforderungen an einen solchen Dienstes, welcher dessen Betrieb unter den Faktoren des derzeitig gültigen Datenschutz erlaubt.


-Beinhaltet kein Datenschutzkonzept 
- Keine vorgehen bei Implementierung 
- UI wird nur eingeschränkt betrachtet 
- Funktion für das anoymisieren von Bildern entwickeln, nicht aber plan für die Ärzte zu, briefen 
- System für AGBs wird in die App implementiert, AGBs werden aber nicht geschrieben 
- Es wird nur in dieser Arbeit behandelt, wenn es eine Anpassung der Software zur Folge hat