\begin{abstract}
\thispagestyle{kapitelkopfzeile}
\textbf{\LaTeX-Vorlage für Projekt-, Seminar- und Bachelorarbeiten}

Bei dem vorliegenden Dokument handelt es sich um eine Vorlage, die
für Projekt-, Seminar- und Bachelorarbeiten im Studiengang
Wirtschaftsinformatik der DHBW Stuttgart verwendet werden kann.

Sie setzt die technischen Vorgaben der Zitierrichtlinien\footnote{Sie finden diese unter \enquote{Prüfungsleistungen} im Studierendenportal (\url{https://studium.dhbw-stuttgart.de/winf/pruefungsleistungen/}).} des Studiengangs
(Stand: 01/2020) um.

\emph{Hinweise:} Bitte lesen Sie sich die Zitierrichtlinien unbedingt genau durch. Dieses Dokument ersetzt keine Anleitung oder Einführung in \LaTeX,
für die Nutzung sind daher gewisse Vorkenntnisse unerlässlich. Ein Einstieg in 
\LaTeX\ ist aber weniger schwierig, als es vielleicht auf den ersten Blick scheint
und lohnt sich für das Verfassen wissenschaftlicher Arbeiten in jedem Fall.\footnote{%
so auch \url{http://www.spiegel.de/netzwelt/tech/textsatz-keine-angst-vor-latex-a-549509.html}} 
Als Hilfestellung beim Schreiben eines Dokuments habe ich einen zweiseitigen kompakten \LaTeX-Spickzettel erstellt, der über Moodle verfügbar ist.

Ihre Rückmeldungen und Anregungen zu dieser Vorlage nehme ich gerne per E-Mail an die Adresse
\url{tobias.straub@dhbw-stuttgart.de} entgegen.

--- Prof. Dr. Tobias Straub

\vspace{5em}

\begin{center}\small
\begin{tabular}{ccl}
\multicolumn{3}{c}{\textbf{Versionshistorie}}\\
\hline
1.0	& 05.02.2015 & erste Fassung \\
\hline
1.1 & 16.02.2015 & siehe~\ref{anhang:ReleaseNotes11} \\
\hline
1.2 & 20.04.2015 & siehe~\ref{anhang:ReleaseNotes12} \\
\hline
1.3 & 20.02.2016 & siehe~\ref{anhang:ReleaseNotes13} \\
\hline
1.4 & 24.07.2017 & siehe~\ref{anhang:ReleaseNotes14} \\
\hline
1.5 & 07.01.2018 & siehe~\ref{anhang:ReleaseNotes15} \\
\hline
1.6 & 07.04.2018 & siehe~\ref{anhang:ReleaseNotes16} \\
\hline
1.7 & 12.02.2019 & siehe~\ref{anhang:ReleaseNotes17} \\
\hline
1.8 & 10.02.2020 & siehe~\ref{anhang:ReleaseNotes18} \\
\end{tabular}
\end{center}

\end{abstract}

