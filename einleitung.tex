\chapter{Einleitung}\label{chapter:einleitung}

Messanger Dienste haben über die letzten zehn Jahre massiv an bedeutung gewonnen. Dienste wie Facebook Messanger, Whatsapp und WeChat verzeichnen monatlich milliarden aktiver Nutzer.\footnote{Vgl. \cite{Ballve2014}} Eine vergleichbare Entwicklung lässt sich auch im Business Umfeld beobachten. Hier haben sich Services wie Slack und Microsoft Teams zu essentiellen Kommunikations-Tools für Unternehmen entwickelt, während herkömmliche Kommunikationsdienste wie E-Mail und SMS immer mehr an bedeutung verlieren.\footnote{Vgl. \cite{Richter2019}}
Gründe hierfür sind neben der leichten Bedienbarkeit vorallem der große Funktionsumfang, der es erlaubt, neben Textnachrichten auch Bilder, Videos und Sprachnachrichten auszutauschen, Sprach- und Videoanrufe durchzuführen und wahlweise mit einzelnen Teilnehmern oder in der Gruppe zu kommunizieren.

Trotz der Vielzahl an Gründen für moderne Kommunikationsplattformen in Unternehmen, gibt es dennnoch eine Reihe von Bereichen, in denen die Verwendung der etablierten Dienste nur sehr eingeschränkt möglich ist, nämlich Bereiche in denen mit besonders schützenswerten, persönlichen oder sensiblen Daten gearbeitet wird. Der Grund für diese Einschränkungen liegt bei den europaweiten, bundes und länderspezifischen Datenschutzgesetzen, welche das Verarbeiten und Teilen dieser Art Daten stark reglementieren. Diesen besonders hohen Datenschutzanforderungen können verbreitete Systeme wie Teams oder Slack nicht gerecht werden.\footnote{Vgl. \cite{Datenschutzkonferenz2019}}

Ein Paradebeispiel für einen Sektor, der auf schnelle Kommunikation angewiesen ist und mit besonders schützenswerten, persönlichen und sensiblen Daten arbeitet, ist der medizinische Sektor, insbesondere Krankenhäuser. Studien zeigen, dass die Kommunikation zwischen Personal innerhalb eines Krankenhauses einen direkten Eiluss auf die qualität der Behandlung haben kann und somit den Krankheitsverlauf und die Behandlungszeit von Patienten beeinflussen kann. Krankenhäuser sind somit stark daran interessiert, die interne Kommunikation so gut wie möglich zu gestalten.\footnote{Vgl. \cite{G.Murphy2010}} Hierbei kann ein wie bereits beschriebens, professionelles Messaging System einen positiven Einfluss haben.

Die Nachfrage nach einer solchen Kommunikationsplattform zeigt sich im medizinischen Sektor deutlich. Eine Forderung der Bundesärztekammer sprach sich im Juli 2020 deutlich für die Notwendigkeit und Entwicklung einer solchen Lösung aus.\footnote{Vgl. \cite{Bundesaerztekammer2019}} Ebenso wurde das Thema Krankenhaus Messanger bereits vom Ärzteblatt\footnote{Vgl. \cite{Giesselmann2018}}, wie auch der deutschen DSK Datenschutzkonferenz behandelt,\footnote{Vgl. \cite{Datenschutzkonferenz2019}} je mit dem Verweis auf die Probleme verbreiteter Messenger und dem Nutzen einer solchen Plattform für Krankenhäuser im allgemeinen.

Das Forschungsziel lautet also die konzeptionelle und prototypische Entwicklung eines Datenschutzkonformen und auf die Anforderungen des medizinischen Sektors zugeschnitte Kommunikationsplattform. 

Um diese Zielsetzung zu erreichen, wird in einem ersten Schritt ein Anforderungskatalog anhand der aktuellen Gesetzteslage entwickelt. Darauf aufbauend werden in einem zweiten Schritt leitfadengestützte Experteninterviews mit Ärzten und Datenschutzbeauftragenten von Krankenhäusern geführt, um den Anforderungskatalog um die speziell für den medizinischen Sektor benötigen Funktionen der Plattform zu erweitern. Die Ergebnisse dieser Gespräche werden wörtlich transkribiert und anschließend mithilfe der qualitativen Inhaltsanalyse nach Mayring ausgewertet. Im dritten Schritt wird anhand des Anforderungskatalogs zuerst eine konzeptionelle und anschließend eine prototypische Umsetzung des Messanger Systems entwickelt. 

Die vorliegende Arbeit ist wie folgt aufgebaut: Das erste Kapitel dieser Arbeit dient der Einleitung in die Thematik und beinhaltet die Problemstellung und Zielsetzung der Arbeit. Zudem wird der Ist-Zustand des medizinischen Sektors analysiert und die die Konzeption des Messanger beeinflusseden Akteure aufgelistet. Das zweite Kapitel betrachtet die aktuelle Gesätzeslage. Das dritte Kapitel betrachtet die technischen Anforderungen an das Messangersystems. Im vierten Kaptitel wird auf dieser Basis eine konzeptionelle Umsetzung entwickelt, welche in Kapitel 5 in einen Prototyp überführt wird.

Die Schlussbetrachtung enthält ein Fazit, das die Hauptergebnisse der Arbeit zusammenfasst und reflektiert. Abschließend wird außerdem ein Ausblick auf mögliche in Zukunft zu untersuchende Aspekte gewährt, welche sich als Konsequenz aus dieser Arbeit ergeben.
