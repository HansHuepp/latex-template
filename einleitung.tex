\chapter{Einleitung}\label{chapter:einleitung}

Messengerdienste haben über die letzten zehn Jahre massiv an Bedeutung gewonnen. Dienste wie Facebook Messenger, WhatsApp und WeChat verzeichnen monatlich Milliarden aktiver Nutzer.\footnote{Vgl. \cite{Ballve2014}} Eine vergleichbare Entwicklung lässt sich auch im Business Umfeld beobachten. Hier haben sich Services wie Slack und Microsoft Teams zu essenziellen Kommunikations-Tools für Unternehmen entwickelt, während herkömmliche Kommunikationsdienste wie E-Mail und SMS immer mehr an Bedeutung verlieren.\footnote{Vgl. \cite{Rcihter2019}}
Gründe hierfür sind, neben der leichten Bedienbarkeit, vor allem der große Funktionsumfang, der es erlaubt, neben Textnachrichten auch Bilder, Videos und Sprachnachrichten auszutauschen, Sprach- und Videoanrufe durchzuführen und wahlweise mit einzelnen Teilnehmern oder in der Gruppe zu kommunizieren.

Trotz der Vielzahl an Gründen für moderne Kommunikationsplattformen in Unternehmen, gibt es dennnoch eine Reihe von Bereichen, in denen die Verwendung dieser Dienste nur sehr eingeschränkt möglich ist, nämlich Bereiche, in denen mit besonders schützenswerten, persönlichen oder sensiblen Daten gearbeitet wird. Der Grund für diese Einschränkungen liegt bei den europaweiten, bundes- und länderspezifischen Datenschutzgesetzen, welche das Verarbeiten und Teilen dieser Art Daten stark reglementieren. Diesen besonders hohen Datenschutzanforderungen können verbreitete Systeme wie Teams oder Slack nicht gerecht werden.\footnote{Vgl. \cite[S. 1 ff.]{Datenschutzkonferenz2019}}

Ein Paradebeispiel für einen Sektor, der auf schnelle Kommunikation angewiesen ist und mit besonders schützenswerten, persönlichen und sensiblen Daten arbeitet, ist der medizinische Sektor, insbesondere Krankenhäuser. Studien zeigen, dass die Kommunikation zwischen dem Personal innerhalb eines Krankenhauses einen direkten Einfluss auf die Qualität der Behandlung haben kann und somit den Krankheitsverlauf und die Behandlungszeit von Patienten beeinflusst.\footnote{Vgl. \cite[S. 1292 f.]{G.Murphy2010}} Krankenhäuser sind somit stark daran interessiert, die interne Kommunikation so gut wie möglich zu gestalten. Hierbei kann ein, wie bereits beschriebenes, professionelles Messeging System einen positiven Einfluss haben, da durch Gruppen-, Video- und Audiochats die Möglichkeit Dokumente direkt über das Handy mit Mitarbeitern des Krankenhauses zu teilen und die vielen weiteren Funktionen einer modernen Kommunikationsplattform, eine neue Art der Kommunikation innerhalb des Krankenhauses ermöglicht.

Die Nachfrage nach einer solchen Kommunikationsplattform zeigt sich im medizinischen Sektor deutlich. So sprach sich die Bundesärztekammer im Juli 2020 für die Notwendigkeit der Entwicklung einer solchen Lösung aus.\footnote{Vgl. \cite[S. 2]{Bundesaerztekammer2020}} Ebenso wurde das Thema Krankenhaus Messenger bereits vom Ärzteblatt\footnote{Vgl. \cite{Giesselmann2018}}, wie auch der deutschen DSK Datenschutzkonferenz behandelt,\footnote{Vgl. \cite[S. 1 ff.]{Datenschutzkonferenz2019}} je mit dem Verweis auf die Probleme verbreiteter Messenger und dem Nutzen einer solchen Plattform für Krankenhäuser im Allgemeinen.

Das Forschungsziel lautet dementsprechend die konzeptionelle und prototypische Entwicklung einer datenschutzkonformen Kommunikationsplattform, welche vom Krankenhauspersonal für organisatorische und Gesundheitsdaten betreffende Kommunikation genutzt werden kann und darf. 

Um diese Zielsetzung zu erreichen, wird ein Anforderungskatalog für die Umsetzung des Dienstes entwickelt. Dies geschieht durch die Analyse und Auswertung von Gesetzestexten und Fachliteratur. Gestützt und ergänzt werden die Befunde mit einem leitfadengestützten Experteninterview mit der Abteilungsleitung für Klinische Systeme des Universitätsklinikums Münster, Georg Woditsch. Die Ergebnisse der Recherchen und des Interviews wurden mit Hilfe der qualitativen Inhaltsanalyse nach Mayring ausgewertet.\footnote{Vgl. \cite[S. 35]{Mayring2002}} Anhand der gewonnen Informationen wird ein entsprechender Anforderungskatalog für Messanger erstellt. Abschließend soll anhand des Anforderungskatalogs eine konzeptionelle Umsetzung des Messenger Systems entwickelt werden.

Die Implementierung eines, aus rechtlicher wie auch aus technischer Sicht so komplexen Systems, wie eines Messengerdienstes, in die bestehende IT-Infrastruktur eines Krankenhauses, ist ein sehr umfangreicher Prozess. Zu der Integration des Dienstes zählen, neben der eigentlichen technischen Umsetzung, auch Prozesse wie die Erstellung von Onboarding Plänen für die Mitarbeiter des Krankenhauses, Notfall Konzepte für den Ausfall des Service oder der Aufbau eines Supports für den Messengerdienst. Zudem sollten für das Krankenhauspersonal hilfreiche, in den bestehenden Diensten nicht vorhandene, Funktionen ergänzt werden können, welche zum Beispiel durch Design Thinking Workshops erörtert werden könnten.\footnote{Vgl. Experteninterview Georg Woditsch} Die komplette Abbildung der Entwicklung und Integration des Messengerdienstes würde somit den Umfang dieser wissenschaftlichen Arbeit überschreiten. Aus diesem Grund fokussiert sich der Inhalt dieser Arbeit ausschließlich auf die technische Umsetzung der gesetzlichen Anforderungen an einen Dienst, welcher organisatorische und Gesundheitsdaten betreffende Kommunikation unter dem derzeitig gültigen Datenschutzrecht erlaubt. Faktoren, die für die Umsetzung und Inbetriebnahme des Messengerdienstes notwendig sind, aber die eigentliche Zielsetzung überschreiten, sind nicht Bestandteil dieser Arbeit und müssten dementsprechend bei einer Weiterführung der Forschung behandelt werden.
